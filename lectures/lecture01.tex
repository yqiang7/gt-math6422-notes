\chapter{Jan.~13 --- Overview and Review}

\section{Course Overview}

\begin{remark}
  This course will cover the following
  topics:
  \begin{enumerate}[(i)]
    \item vector bundles and line
      bundles in algebraic geometry;
    \item coherent sheaves;
    \item differentials;
    \item sheaf cohomology: in
      particular, we will see that
      $H^k_{\mathrm{dR}}(X^{\mathrm{an}}, \C) = \bigoplus_{i + j = k} H^i(X, \wedge^j T_X^*)$);
    \item the Riemann-Roch theorem:
      if $\omega = f\, dz$ is a rational
      $1$-form on a smooth
      projective curve $C$, then
      \[
        (\# \text{ zeroes of } \omega)
        - (\# \text{ poles of } \omega)
        = 2 \genus(C) - 2;
      \]
    \item surfaces and toric varieties;
    \item schemes: for example,
      $\Spec \Z$ has points
      corresponding to the primes $p$
      and $0$.
  \end{enumerate}
\end{remark}

\section{Review of Algebraic Geometry I}

\begin{remark}
  Let $k = \overline{k}$ be an
  algebraically closed field.
\end{remark}

\begin{remark}[Hilbert's Nullstellensatz]
  There is a correspondence
  \begin{align*}
    \text{closed subvarieties of $\Affine^n$}
    &\longleftrightarrow
    \text{radical ideals in $k[x_1, \dots, x_n]$} \\
    Z &\longmapsto I(Z) \\
    V(J) &\mathrel{\reflectbox{\ensuremath{\longmapsto}}} J.
  \end{align*}
  Under this correspondence,
  $Z$ being
  irreducible (resp. a point)
  corresponds to $I(Z)$ being
  prime (resp. maximal).
\end{remark}

\begin{remark}[Zariski topology on $\Affine^n$]
  The closed sets in $\Affine^n$ are
  of the form $V(J)$, and this induces
  a Zarisiki topology on any subset
  of $\Affine^n$.
\end{remark}

\begin{remark}[Embedded affine varieties]
  Let $J \le k[x_1, \dots, x_n]$.
  Then we can associate to $J$ a
  ringed space $(X, \OO_X)$
  by setting $X := V(J) \subseteq \Affine^n$
  with the Zariski topology, and
  $\OO_X$ to be the sheaf of regular functions on $X$, i.e.
  for $U \subseteq X$ open, we have
  \[
    \OO_X(U)
    := \{
      \varphi : U \to k \mid
      \varphi \text{ is regular}
    \}.
  \]
  Here, $\varphi : U \to k$ is
  \emph{regular} if for each $p \in U$,
  there exists an open set
  $p \in U_p \subseteq U$ and
  $f, g \in k[x_1, \dots, x_n]$
  such that $\varphi(x) = f(x) / g(x)$
  for all $x \in U_p$.
\end{remark}

\begin{remark}[Coordinate ring]
  The \emph{coordinate ring} of $X$ is
  \[
    A(X) :=
    \OO_X(X)
    \cong k[x_1, \dots, x_n] / I(X).
  \]
  We also get a version of
  Hilbert's Nullstellensatz for
  $A(X)$:
  \[
    \text{closed subsets of $X$}
    \longleftrightarrow
    \text{radical ideals of $A(X)$}.
  \]
\end{remark}

\begin{remark}[Distinguished open sets]
  The \emph{distinguished open sets} of $X$
  are
  \[
    D(f)
    := \{x \in X : f(x) \ne 0\}
    = X \setminus V(f).
  \]
  These for a basis for $X$ as we vary
  $f \in A(X)$.
\end{remark}

\begin{definition}
  An \emph{affine variety} is a
  ringed space $(X, \OO_X)$ (here
  $\OO_X$ is a sheaf of $k$-valued
  functions) which is
  isomorphic to an embedded affine
  variety.
\end{definition}

\begin{example}
  If $(X, \OO_X)$ is an affine variety
  and $f \in \OO_X(X)$, then
  \[
    (D(f), \OO_X|_{D(f)})
  \]
  is again an affine variety. To
  see this, we may assume that
  $X = V(J) \subseteq \Affine^n$
  with $J \le k[x_1, \dots, x_n]$
  a radical ideal. Now we can define
  a map
  \begin{align*}
    D(f) &\longrightarrow
    V(J, fy - 1) \subseteq \Affine^n_{x_i} \times \Affine^1_y \\
    x &\longmapsto (x, 1 / f(x)),
  \end{align*}
  which one can check is an isomorphism.
  Now that this also shows
  \[
    \OO_X(D(f))
    = A(D(f))
    \cong \frac{k[x_1, \dots, x_n, y]}{(J, fy - 1)}
    \cong \frac{(k[x_1, \dots, x_n] / J)[y]}{(\overline{f} y - 1)}
    \cong A(X)_f.
  \]
\end{example}

\begin{theorem}
  There is an equivalence of
  categories
  \begin{align*}
    \Phi :
    \mathrm{Aff\text{-}var}
    &\longrightarrow
  \mathrm{Red\text{-}f.g.\text{-}} k \mathrm{\text{-}alg}^{\mathrm{op}} \\
  (X, \OO_X)
    &\longmapsto A(X).
  \end{align*}
  This implies the following:
  \begin{enumerate}
    \item There is a bijection
      \begin{align*}
        \Hom_{\mathrm{aff}\text{-}\mathrm{var}}(X, Y)
        &\longrightarrow
        \Hom_{k\text{-}\mathrm{alg}}(A(Y), A(X)) \\
        f &\longmapsto f^*.
      \end{align*}
    \item For any reduced finitely
      generated $k$-algebra $A$,
      there exists an affine variety
      with $A \cong A(X)$.
  \end{enumerate}
\end{theorem}

\begin{remark}
  How can we explicitly
  define the inverse
  functor
  $
  \mathrm{Red\text{-}f.g.\text{-}} k \mathrm{\text{-}alg}^{\mathrm{op}}
    \rightarrow
  \mathrm{Aff\text{-}var}
  $?
  We can define this as
  $A \mapsto (X, \OO_X)$, where
  $X$ is the set of maximal ideals
  of $A$. Think about what
  $\OO_X$ should be.
\end{remark}

\begin{remark}[Varieties]
  A \emph{variety} $(X, \OO_X)$
  is a ringed space such that
  \begin{itemize}
    \item there exists a finite open
      cover of $X$ by affine varieties,
    \item the diagonal $\Delta_X$ is
      closed in $X \times X$.
  \end{itemize}
\end{remark}

\begin{example}
  The following are examples of
  varieties:
  \begin{itemize}
    \item affine varieties,
    \item open or closed subsets of
      varieties,
    \item $\PP^n = (\Affine^{n + 1} \setminus \{0\}) / k^\times$.
  \end{itemize}
\end{example}

\begin{remark}[Projective spaces]
  Recall that $\PP^n$ has an
  open cover by
  \[
    U_i
    = \{[x] \in \PP^n : x_i \ne 0\}
    \cong \Affine^n.
  \]
  A basis for $\PP^n$ by distinguished
  open sets is given by
  \[
    D(f)
    = \{[x] \in \PP^n : f(x) \ne 0\}
  \]
  with $f \in k[x_0, \dots, x_n]$
  homogeneous.
\end{remark}

\section{Vector and Line Bundles}

\begin{definition}
  Let $X$ be a variety. A
  \emph{vector bundle (of rank $m$)}
  on $X$ is a variety
  $\EE$ with a morphism
  $p : \EE \to X$ such that
  \begin{enumerate}
    \item $\EE_x = p^{-1}(X)$ has the
      structure of a rank $m$
      vector space for every $x \in X$ 
      (i.e. $k^m$),
    \item for every $x \in X$,
      there exists an open neighborhood
      $x \in U \subseteq X$ 
      and an isomorphism
      $p^{-1}(U) \to U \times \Affine^m$
      such that for any $y \in U$,
      the map
      $\EE_y \to \{y\} \times \Affine^m$
      is an isomorphism of vector
      spaces, i.e.
      \[
        \begin{tikzcd}
          p^{-1}(U)
          \ar[rr, "\phi_U"] \ar[dr ] & &
          U \times \Affine^m \ar[dl] \\
          & U
        \end{tikzcd}
      \]
      commutes.
      We will call the map $\phi_U$ a
      \emph{trivialization.}
  \end{enumerate}
\end{definition}

\begin{definition}
  A \emph{line bundle} on $X$ is a
  rank $1$ vector bundle.
\end{definition}

\begin{remark}
  A different way to think about
  this is the following:
  \begin{enumerate}
    \item Given two trivializations
      $\phi_U : p^{-1}(U) \to U \times \Affine^m$
      and $\phi_V : p^{-1}(V) \to V \times \Affine^m$,
      we get a morphism
      \[
        \begin{tikzcd}
        (U \cap V) \times \Affine^m
        \ar[rr, "\phi_{U, V}"]
        \ar[dr, "\phi_V^{-1}", swap]
        & & (U \cap V) \times \Affine^m \\
        & p^{-1}(U \cap V) \ar[ur, "\phi_U", swap]
        \end{tikzcd}
      \]
      with $\phi_{U, V} = \phi_U \circ \phi_V^{-1}$.
      Observe
      $\phi_{U, V}(x, v) = (x, g_{U, V}(x) v)$
      for some
      $g_{U, V}(x) \in \GL(m, k)$.
      Furthermore,
      $g_{U, V} : U \cap V \to \GL(m, k)$
      is a morphism.
      We will call the
      $g_{U, V}$ \emph{transition functions}.

      In the special case where
      $m = 1$ (so
      $\EE$ is a line bundle
      and $\GL(1, k) = k^\times$),
      the map
      $g_{U, V} : U \cap V\to \GL(m, k)$
      is equivalent to the data of a
      non-vanishing function
      $g_{U, V} : U \cap V \to k$.
    \item The data of a vector bundle
      of rank $m$
      is equivalent to the data of
      \begin{itemize}
        \item an open cover
          $X = \bigcup_{i \in I} U_i$,
        \item and morphisms
          $g_{i, j} : U_i \cap U_j \to \GL(m, k)$
      \end{itemize}
      such that
      $g_{i, k} = g_{i, j} g_{j, k}$,
      $g_{i, j} = g_{j, i}^{-1}$,
      and $g_{i, i} = \id$.

      To recover the vector bundle,
      we can glue
      $\EE_i = U_i \times \Affine^m$
      for $i \in I$
      via
      \begin{align*}
        \EE_{i, j}
        = (U_i \cap U_j) \times \Affine^m
        &\longrightarrow
        \EE_{j, i}
        = (U_j \cap U_i) \times \Affine^m \\
        (x, v) &\longmapsto
        (x, g_{i, j}(x) v).
      \end{align*}
      One can check that this defines
      a vector bundle $\EE$ on $X$.
  \end{enumerate}
\end{remark}

\begin{example}[Trivial vector bundle]
  Define the vector bundle
  $\EE : X \times \Affine^m \to X$
  by $(x, v) \to x$.
  Given a cover $X = \bigcup_{i \in I} U_i$,
  we get $g_{i, j} : U_i \cap U_j \to \GL(m, k)$
  as $x \mapsto I_m$.
\end{example}

\begin{example}[Trivial line bundle]
  We will denote the trivial
  line bundle by
  $\mathbbm{1}_X : X \times \Affine^1 \to X$.
\end{example}

\begin{example}
  Let $X = \PP^n$ and
  $\mathbb{L} = \{(\ell, x) \in \PP^n \times \Affine^{n + 1} : x \in \ell\}$. Consider
  \[
    \begin{tikzcd}
      & \mathbb{L} \ar[dl, "p"] \ar[dr, "q"] \\
      \PP^n & & \Affine^{n + 1}
    \end{tikzcd}
  \] 
  The map $q : \mathbb{L} \to \Affine^{n + 1}$ is the blowup.
  We claim that $p : \mathbb{L} \to \PP^n$
  is a line bundle. We have:
  \begin{itemize}
    \item $p^{-1}([x]) = kx$,
      a $1$-dimensional vector space;
    \item let $U_i = \{[x] \in \PP^n : x_i \ne 0\}$,
      then we can define
      \begin{align*}
        p^{-1}(U_i)
        &\longrightarrow U_i \times \Affine^1 \\
        ([x], y)
        &\longmapsto ([x], y_i),
      \end{align*}
      which one can check is
      a trivialization.
  \end{itemize}
\end{example}
