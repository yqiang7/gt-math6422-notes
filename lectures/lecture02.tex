\chapter{Jan.~15 --- Vector and Line Bundles}

\section{Vector and Line Bundles, Continued}

\begin{example}[Tautological bundle]
  Let $X = \PP^n$ and
  $\mathbb{L} = \{(\ell, x) \in \PP^n \times \Affine^{n + 1} : x \in \ell\}$. Consider
  \[
    \begin{tikzcd}
      & \mathbb{L} \ar[dl, "p", swap] \ar[dr, "q"] \\
      \PP^n & & \Affine^{n + 1}
    \end{tikzcd}
  \] 
  The map $q : \mathbb{L} \to \Affine^{n + 1}$ is the blowup.
  We claim that $p : \mathbb{L} \to \PP^n$
  is a line bundle. We have:
  \begin{itemize}
    \item $p^{-1}([x]) = \{([x], cx) : c \in k\} \cong kx$,
      a $1$-dimensional vector space;
    \item let $U_i = \{[x] \in \PP^n : x_i \ne 0\}$,
      then we can define
      \begin{align*}
        p^{-1}(U_i)
        &\longrightarrow U_i \times \Affine^1 \\
        ([x], y)
        &\longmapsto ([x], y_i),
      \end{align*}
      which we claim is
      a trivialization. To see this,
      observe that for fixed
      $[x] \in \PP^n$, we have
      \begin{align*}
        \mathbb{L}_{[x]}
        = \{([x], cx) : c \in k\}
        &\longrightarrow
        \{[x]\} \times \Affine^1 \\
        ([x], cx)
        &\longmapsto
        ([x], cx_i),
      \end{align*}
      which is a vector space
      isomorphism.
  \end{itemize}
  We can also compute the
  transitions functions. Let
  $U_{i, j} = U_i \cap U_j$. We have
  \[
    \begin{tikzcd}[row sep=small]
      U_{i, j} \times \Affine^1
      \ar[r, "{\phi_j^{-1}}", swap]
      \ar[rr, bend left=15, "{\phi_{i, j}}"]
      & p^{-1}(U_{i, j})
      \ar[r, "{\phi_i}", swap]
      & U_{i, j} \times \Affine^1 \\
      ([x], t)
      \ar[r, mapsto] &
      ([x], (t x_0 / x_j, \dots, t x_n / x_j))
      \ar[r, mapsto] &
      ([x], tx_i / x_j).
    \end{tikzcd}
  \]
  Thus we see that
  $g_{i, j} = x_i / x_j$.
  This is called the \emph{tautological bundle},
  or $\OO_{\PP^n}(-1)$.
\end{example}

\begin{example}[Hyperplane bundle, or $\OO_{\PP^n}(1)$]
  Consider
  \begin{align*}
    \mathbb{L} := \PP^{n + 1} \setminus \{[0 : \cdots : 0 : 1]\}
    &\longrightarrow \PP^n \\
    {[x_0 : \cdots : x_n : x_{n + 1}]}
    &\longmapsto {[x_0 : \cdots : x_n]}.
  \end{align*}
  Then $\mathbb{L}$ is a line
  bundle with transition functions
  with respect to $\{U_i\}$ given
  by $g_{i, j} = x_j / x_i$ (HW).
\end{example}

\section{Operations on Vector Bundles}

\begin{remark}
  The philosophy is:
  Every natural operation of
  vector spaces gives one for
  vector bundles.
\end{remark}

\begin{example}[Direct sum]
  Let
  $p : \EE \to X$ and $q : \mathbb{F} \to X$
  be vector bundles of rank
  $e$ and $f$ on $X$, respectively.
  There exists trivializations
  with respect to a common open cover
  $\{U_i\}$ (just take intersections)
  with transition functions
  $g_{i, j}$ and $h_{i, j}$ for
  $\EE$ and $\mathbb{F}$, respectively.

  Then we define the vector bundle
  $\EE \oplus \mathbb{F}
    \longrightarrow X$
  as follows:
  \begin{itemize}
    \item As a set, it is
      $r : \EE \oplus \mathbb{F} = \{(x, u, v) : (x, u) \in \EE, (x, v) \in \mathbb{F}\} \to X$.
    \item We give $\EE \oplus \mathbb{F}$
      the structure of a variety by
      requiring that
      \begin{align*}
        r^{-1}(U_i)
        &\longrightarrow U_i \times \Affine^{e + f} \\
        (x, u, v)
        &\longmapsto
        (x, \pr_2(\phi^E_i(x, u)), \pr_2(\phi^F_i(x, v)))
      \end{align*}
      be an isomorphism,
      where $\phi^E_i$ and $\phi^F_i$
      are the trivializations of
      $\EE$ and $\mathbb{F}$,
      and $\pr_2$ is the
      second projection.
      This gives a variety structure on
      $r^{-1}(U_i)$, and one can show that
      these are consistent
      on $U_{i, j}$, so that this
      gives a variety structure on
      all of $\EE \oplus \mathbb{F}$.
  \end{itemize}
  Note that the transition functions
  for $\EE \oplus \mathbb{F}$
  with respect to $\{U_i\}$ are given
  by the block matrix
  \[
    \begin{bmatrix}
      g_{i, j} & 0 \\
      0 & h_{i, j}
    \end{bmatrix}
    : U_{i, j} \longrightarrow \GL(e + f, k).
  \]
\end{example}

\begin{example}
  Let $\EE$ and
  $\mathbb{F}$ be vector bundles on $X$
  of ranks $e$ and $f$, respectively.
  Then the following are
  also vector bundles on $X$:
  \begin{enumerate}
    \item $\Hom(\EE, \mathbb{F})$,
      of rank $ef$;
    \item $\EE^\vee = \Hom(\EE, \mathbbm{1}_X)$,
      of rank $e$;
    \item $\EE \otimes \mathbb{F}$,
      of rank $ef$;
    \item $\wedge^k \EE$
      and $\mathrm{Sym}^d \EE$.
  \end{enumerate}
\end{example}

\begin{remark}
  Let $\mathbb{L}, \mathbb{M}$ be line bundles
  on $X$ with trivializations
  on $\{U_i\}$ and transition functions
  $g_{i, j}, h_{i, j} \in \OO_X(U_{i, j})^\times$.
  In this case, we can describe
  operations on
  $\mathbb{L}, \mathbb{M}$
  more explicitly:
  \begin{enumerate}
    \item $\mathbb{L} \otimes \mathbb{M}$
      has transition functions
      $g_{i, j} h_{i, j}$;
    \item $\Hom(\mathbb{L}, \mathbb{M})$
      has transition functions
      $h_{i, j} / g_{i, j}$;
    \item $\mathbb{L}^\vee = \Hom(\mathbb{L}, \mathbbm{1}_X)$
      has transition functions
      $1 / g_{i, j}$;
    \item $\mathbb{L}^{\otimes m} =
      \begin{cases}
        \mathbb{L}^{\otimes m}, & \text{if } m > 0, \\
        \mathbbm{1}_X, & \text{if } m = 0, \\
        (\mathbb{L}^\vee)^{\otimes -m}, & \text{if } m < 0
      \end{cases}$
      has transition functions
      $g_{i, j}^m$.
  \end{enumerate}
\end{remark}

\begin{example}
  Define $\OO_{\PP^n}(m) := \OO_{\PP^n}(1)^{\otimes m}$
  with transition functions
  $(x_j / x_i)^m$ with respect
  to the standard open cover
  for $\PP^n$.
\end{example}

\section{Morphisms of Vector Bundles}

\begin{remark}
  Let $p : \EE \to X$ and
  $q : \mathbb{F} \to X$ be
  vector bundles on $X$, as before.
\end{remark}

\begin{definition}
  A \emph{morphism of vector
  bundles} $\EE \to \mathbb{F}$
  is a morphism of
  varieties
  \[
    \begin{tikzcd}
      \EE \ar[rr, "a"] \ar[dr, "p", swap] & & \mathbb{F} \ar[dl, "q"] \\
      & X
    \end{tikzcd}
  \]
  such that the diagram commutes
  and $a$ is linear on each fiber.
\end{definition}

\begin{remark}
  More concretely, given an
  open cover
  $\{U_i\}$ which trivializes
  both vector bundles, we have
  \[
    \begin{tikzcd}[row sep=small]
      p^{-1}(U_i) \ar[r, "a"]
      \ar[dd, "\phi_i", swap]
      \ar[dd, "\cong"]
      & q^{-1}(U_i)
      \ar[dd, "\psi_j"]
      \ar[dd, "\cong", swap]
      \\ \\
      U_i \times \Affine^e
      \ar[r] &U_i \times \Affine^f \\
      (x, v)
      \ar[r, mapsto]
             & (x, a_i(x) v)
    \end{tikzcd}
  \]
  such that $a_i : U_i \to \Hom(k^e, k^f)$
  is regular.
  On $U_{i, j}$, we have
  \[
    \begin{tikzcd}
      U_{i, j} \times \Affine^e
      \ar[r, "a_j"]
      \ar[d, "g_{i, j}", swap]
      & U_{i, j} \times \Affine^f
      \ar[d, "h_{i, j}"] \\
      U_{i, j} \times \Affine^e
      \ar[r, "a_i", swap]
      & U_{i, j} \times \Affine^f
    \end{tikzcd}
  \]
  So $h_{i, j} a_j = a_i g_{i, j}$,
  or equivalently,
  $a_i = h_{i, j} a_j g_{i, j}^{-1}$.

  As a special case when
  $e = f$,
  $a : \EE \to \mathbb{F}$
  is an isomorphism if and only if
  the $a_i$ are isomorphisms.
\end{remark}

\begin{remark}
  When is a line bundle
  $\mathbb{L}$ given by
  the trivialization data
  $\{U_i, g_{i, j}\}$ isomorphic
  to $\mathbbm{1}_X$?
  We have
  \begin{align*}
    \mathbb{L} \cong \mathbbm{1}_X
    &\iff
    \text{if and only if there exists
    an isomorphism }
    a : \mathbbm{1}_X \to \mathbb{L} \\
    &\iff
    \text{there exist }
    a_i \in \OO_X(U_i)^\times
    \text{ such that }
    (a_j / a_i)|_{U_{i, j}} = g_{i, j}.
  \end{align*}
\end{remark}

\begin{definition}
  Define the \emph{Picard group}
  of $X$ to be
  \[
    \Pic X
    := \{\text{line bundles on $X$}\} / {\cong}.
  \]
  This is a group with respect
  to $\otimes$ with
  $\mathbbm{1}_X$
  as the identity and
  $\mathbb{L}^\vee \otimes \mathbb{L} \cong \mathbbm{1}_X$.
\end{definition}

\section{Global Sections}

\begin{definition}
  A \emph{(global) section}
  of a vector bundle $p : \EE \to X$
  is a morphism
  $s : X \to \EE$
  such that $p \circ s = \id_X$.
  Note that for $x \in X$, we have
  $s(x) \in \EE_x$.
\end{definition}

\begin{example}[Zero section]
  Let $s : X \to \EE$ where
  $s(x)$ is the zero element
  in $\EE_x$.
\end{example}

\begin{example}
  Let $\EE = \mathbbm{1}_X$.
  Then sections $s : X \to X \times \Affine^1$
  of $\EE$ correspond to
  morphisms $X \to \Affine^1$,
  which correspond to
  regular functions $X \to k$.
\end{example}

\begin{remark}[Local description of sections]
  Let $\{U_i, g_{i, j}\}$ be the
  trivialization data for
  $\EE \to X$, and let
  $s : X \to \EE$ be a section.
  On $U_i$, we have:
  \[
    \begin{tikzcd}
      & U_i \times \Affine^e \\
      U_i \ar[r, "{s|_{U_i}}", swap] \ar[ur, "{x \mapsto (x, s_i(x))}"] & p^{-1}(U_i) \ar[u, "\phi_i", swap] \ar[u, "\cong"]
    \end{tikzcd}
  \]
  Note that $s_i : U_i \to k^e$
  is a regular function (i.e.
  regular on each coordinate).
  These maps must satisfy
  the compatibility condition
  $s_i = g_{i, j} s_j$, since
  we have the diagram:
  \[
    \begin{tikzcd}
      & & U_{i, j} \times \Affine^e \ar[dd, "{(x, v) \mapsto (x, g_{i, j}(v))}"] \\
      U_{i, j} \ar[urr, bend left=30, "{(x, s_j(x))}"] \ar[drr, bend right=30, swap, "{(x, s_i(x))}"] \ar[r, "s|_{U_{i, j}}"]& p^{-1}(U_{i, j}) \ar[ur, "\phi_j"] \ar[dr, swap, "\phi_i"] \\
               & & U_{i, j} \times \Affine^e
    \end{tikzcd}
  \]
\end{remark}

\begin{example}
  We can use the above compatibility
  condition to compute the
  global sections of $\OO_{\PP^1}(1)$.
  Write $\PP^1_{x_0 : x_1} = U_0 \cup U_1$.
  Given a section $s : \PP^1 \to \OO(1)$,
  we get regular functions
  \begin{align*}
    s_0 : U_0 &\longrightarrow k \\
    s_1 : U_1 &\longrightarrow k
  \end{align*}
  satisfying
  $(x_1 / x_0) s_1 = s_0$ $(*)$.
  We can write
  \[
    s_0 = \sum_{m \ge 0} a_m (x_1 / x_0)^m
    \quad \text{and}
    \quad
    s_1 = \sum_{m \ge 0} b_m (x_0 / x_1)^m
  \]
  with $a_m, b_m \in k$ (finitely
  many nonzero).
  Then $(*)$ implies that
  \[
    a_0 + a_1 (x_1 / x_0) + \cdots
    = (x_1 / x_0)(b_0 + b_1 (x_0 / x_1) + \cdots),
  \]
  so $a_0 = b_1$, $a_1 = b_0$, and
  all other terms are $0$. So
  we can relate $s$ to a linear form
  \[
    f = a_0 x_0 + a_1 x_1,
  \]
  where $s_0 = (1 / x_0) f$
  and $s_1 = (1 / x_1) f$.
\end{example}
