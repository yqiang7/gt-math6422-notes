\chapter{Jan.~20 --- Sections}

\section{Global Sections, Continued}

\begin{definition}
  Let $\Gamma(X, \EE) := \{\text{sections of } \EE \to X\}$,
  which
  has the structure of a
  $k$-vector space by
  \[
    (s + t)(x) = s(x) + t(x)
    \quad \text{and} \quad
    (cs)(x) = c s(x)
  \]
  for $s, t \in \Gamma(X, \EE)$ and
  $c \in k$.
\end{definition}

\begin{example}
  One can check that
  $\Gamma(\PP^n, \OO_{\PP^n}(d)) \cong k[x_0, \dots, x_n]_d$ (HW).
  For example, for $d < 0$, we have
  $\Gamma(\PP^n, \OO(d)) = \{0\}$, and
  for $d = 0$, we have
  \[
    \Gamma(\PP^n, \OO(0))
    = \Gamma(\PP^n, \mathbbm{1}_{\PP^n})
    = \OO_{\PP^n}(\PP^n)
    = k.
  \]
  For $d = 1$,
  we can define an isomorphism
  \begin{align*}
    k[x_0, \dots, x_n]_1
    &\overset{\cong}{\longrightarrow}
    \Gamma(\PP^n, \OO(1)) \\
    f &\longmapsto
    \text{section } s : \PP^n \to \OO(1)
    \text{ given by }
    s_i = f / x_i.
  \end{align*}
  Note that $(x_j / x_i) s_j = s_i$
  holds, so this is a section.
  An alternative perspective
  is that $s$ corresponds to
  \begin{align*}
    \PP^n &\longrightarrow \PP^{n + 1} \setminus \{[0 : \cdots : 0 : 1]\} \\
    x &\longmapsto [x_0 : \cdots : x_n : f(x)].
  \end{align*}
\end{example}

\section{Morphisms and Sections}

\begin{definition}
  Given a section $s : X \to \EE$,
  its \emph{vanishing locus} is
  \[
    V(s) := \{s = 0\}
    = \{x \in X : s(x) = 0\}.
  \]
  Using a trivializing cover, one
  can check that $V(s)$ is closed in
  $X$.
\end{definition}

\begin{example}
  For a section $s : \PP^n \to \OO(1)$
  corresponding to
  $f \in k[x_0, \dots, x_n]_1$,
  we have $V(s) = V_{\PP^n}(f)$.
\end{example}

\begin{remark}
  Recall that there is a bijection
  \begin{align*}
    \{\text{morphisms } X \to \Affine^n\}
    &\longleftrightarrow
    \{f_1, \dots, f_n \in \OO_X(X)\} \\
    [f : X \to \Affine^n]
    &\longmapsto
    [f_1 = f^* x_1, \dots, f_n = f^* x_i \in \OO_X(X)] \\
    [x \mapsto (f_1(x), \dots, f_n(x))] &\mathrel{\reflectbox{\ensuremath{\longmapsto}}} [f_1, \dots, f_n \in \OO_X(X)].
  \end{align*}
  We want a similar statement
  for $\PP^n$.
\end{remark}

\begin{definition}
  Given a line bundle
  $\LL \to X$ and
  $s_0, \dots, s_n \in \Gamma(X, \LL)$,
  they
  are \emph{nowhere vanishing}
  if
  \[V(s_0) \cap \cdots \cap V(s_n) = \varnothing.\]
\end{definition}

\begin{example}
  For $\OO(1)$, the sections
  $x_0, \dots, x_n \in \Gamma(\PP^n, \OO(1))$
  are nowhere vanishing.
\end{example}

\begin{remark}
  If $s_0, \dots, s_n \in \Gamma(X, \LL)$
  are nowhere vanishing, then we
  get a morphism
  \begin{align*}
    X &\longrightarrow \PP^n \\
    x &\longmapsto
    [s_0(x) : \cdots : s_n(x)].
  \end{align*}
  Note that $(s_0(x), \dots, s_n(x))$
  is a well-defined point in
  $\Affine^{n + 1}$ up to scaling.
  One can check that this map is a morphism
  by working locally.
\end{remark}

\begin{example}[Linear maps]
  Let $X = \PP^n$ and $\LL = \OO(1)$.
  \begin{enumerate}[(i)]
    \item $x_0, \dots, x_n \in \Gamma(\PP^n, \OO(1))$
      gives
      $\id : \PP^n \to \PP^n$.
    \item For $A \in \GL_{n + 1}(k)$,
      we get a map
      \begin{align*}
         \PP^n &\longrightarrow \PP^n \\
         [x] &\longmapsto
         [A x]
      \end{align*}
      given by
      $A x_0, \dots, A x_n \in \Gamma(\PP^n, \OO(1))$.
  \end{enumerate}
\end{example}

\begin{remark}
  Now given a morphism
  $X \to \PP^n$, we want to get
  a line bundle with sections.
\end{remark}

\begin{definition}[Pullback]
  Let $p : \EE \to X$ be a vector
  bundle and
  $f : Y \to X$ a morphism.
  Define
  \[
    f^* \EE
    = \{
      (e, y) : e \in \EE, y \in Y
      \text{ with } p(e) = f(y)
    \}
    \longrightarrow Y.
  \]
  One can show that this has
  the structure of a vector bundle
  in a natural way.
\end{definition}

\begin{remark}
  An alternative way to define the
  pullback is to choose
  trivialization data
  $(U_i, g_{i, j})$ for $\EE \to X$.
  Then we can define
  $f^* : \EE \to Y$ to be the
  vector bundle with
  trivialization data
  $(f^{-1}(U_i), f^* g_{i, j})$.
\end{remark}

\begin{remark}
  Now to go in reverse,
  given a morphism
  $X \to \PP^n$ and nowhere
  vanishing sections $x_0, \dots, x_n \in \Gamma(\PP^n, \OO(1))$, we get
  nowhere vanishing sections
  \[
    f^* x_0, \dots, f^* x_n \in \Gamma(X, f^* \OO(1)).
  \]
  We can define the pullback of a
  section in one of two ways:
  by
  $f^*(x_i)(a) = (x_i(f(a)), a) \in f^* \OO(1))$
  for $a \in X$ or
  by using trivializing covers.
\end{remark}

\begin{remark}
  Using the above, we get a bijection
  \begin{align*}
    \{{\text{morphisms } X \to \PP^n}\}
    &\longleftrightarrow
    \{
      \text{line bundles } \LL \to X
      \text{ with }
      s_0, \dots, s_n \in \Gamma(X, \LL)
      \text{ nowhere vanishing}
    \}.
  \end{align*}
  Note that we should consider the 
  right-hand side up to isomorphism
  of the line bundle.
  When do $\LL \to X$ and
  $s_0, \dots, s_n \in \Gamma(X, \LL)$
  give an injective morphism
  (or an embedding)?
\end{remark}

\begin{definition}
  Given a vector bundle $\EE \to X$,
  we get a sheaf of abelian groups
  $\mathcal{E}$ on $X$ by
  \[
    \mathcal{E}(U)
    := \{\text{sections of } p^{-1}(U) \to U\}
  \]
  for $U \subseteq X$ open.
  For $V \subseteq U \subseteq X$ open,
  the restriction map is given by
  \begin{align*}
    \mathcal{E}(U) &\longrightarrow \mathcal{E}(V) \\
    s &\longmapsto s|_V.
  \end{align*}
  \pagebreak
  We call $\mathcal{E}$ the
  \emph{sheaf of sections} of $\EE$.
  Also note that $\mathcal{E}(U)$ has the
  structure of an $\OO_X(U)$-module.
  We will see that this gives rise
  to the structure of an
  $\OO_X$-module.
\end{definition}

\section{Review of Sheaves}

\begin{definition}
  A \emph{presheaf} of abelian
  groups $\mathcal{F}$ on a topological
  space $X$ is the data of:
  \begin{itemize}
    \item for $U \subseteq X$ open,
      an abelian group $\mathcal{F}(U)$ (with $\mathcal{F}(\varnothing) = 0$),
    \item for $V \subseteq U \subseteq X$
      open, a group homomorphism
      $p_{V, U} : \mathcal{F}(U) \to \mathcal{F}(V)$.
  \end{itemize}
\end{definition}

\begin{remark}
  Note the following:
  \begin{enumerate}
    \item We may replace abelian groups in
      the above definition by
      rings, sets,
      $R$-modules, etc.
    \item We denote
      $\Gamma(U, \mathcal{F}) = \mathcal{F}(U)$,
      whose elements are called
      \emph{sections}.
    \item $s|_V := p_{V, U}(s)$
      is called the \emph{restriction}
      for $s \in \mathcal{F}(U)$ and
      $V \subseteq U \subseteq X$
      open.
    \item We may view $\mathcal{F}$
      as a functor
      $\mathrm{Open}_X \to \mathrm{Ab}\text{-}\mathrm{gps}$
      given by $U \mapsto \mathcal{F}(U)$.
  \end{enumerate}
\end{remark}

\begin{definition}
  For $\mathcal{F}$ a presheaf on
  $X$ and $x \in X$, the
  \emph{stalk} of $\mathcal{F}$ at $x$ is
  \[
    \mathcal{F}_x
    = \lim_{\substack{\longrightarrow \\ U \ni x \text{ open}}}
    \mathcal{F}(U)
    = \{(s, U) : s \in \mathcal{F}(U)\} / {\sim}.
  \]
\end{definition}

\begin{example}
  The following are examples of
  presheaves:
  \begin{enumerate}
    \item Let $M$ be a smooth manifold.
      Then
      \begin{itemize}
        \item $\OO_M =$ sheaf of
          smooth $\R$-valued functions on $M$,
        \item $\mathcal{E} =$
          sheaf of sections of a
          vector bundle $\EE \to M$.
      \end{itemize}
    \item Let $X$ be an algebraic
      variety, $\EE \to X$ a
      vector bundle, and
      $Z \subseteq X$ closed. Then
      \begin{itemize}
        \item $\OO_X$ and
          $\mathcal{E}$ are sheaves,
        \item $\mathcal{I}_Z =$
          ideal sheaf of $Z$, given by
          $\mathcal{I}_Z(U) = \{\varphi \in \OO_X(U) : \varphi|_Z = 0\}$.
      \end{itemize}
    \item Let $X$ be a topological
      space and $A$ an abelian group.
      \begin{itemize}
        \item $\underline{A}^{\mathrm{pre}}$
          given by
          $U \mapsto \{\text{constant functions } U \to A\}$, i.e.
          $\underline{A}^{\mathrm{pre}}(U) \cong A$ for
          $U \ne \varnothing$,
        \item $\underline{A}$
          given by
          $U \mapsto \{\text{locally constant functions $U \to A$}\}$,
        \item $i_p A =$ skyscraper
          sheaf, given by
          $U \mapsto
          \begin{cases}
            A & \text{if } p \in U, \\
            0 & \text{otherwise}.
          \end{cases}$
      \end{itemize}
  \end{enumerate}
\end{example}

\begin{definition}
  A presheaf $\mathcal{F}$ is a
  \emph{sheaf} if for any
  \begin{itemize}
    \item open set $U \subseteq X$,
    \item open cover
      $U = \bigcup_{i \in I} U_i$,
    \item and $s_i \in \mathcal{F}(U_i)$
      such that $s_i|_{U_{i, j}} = s_j|_{U_{i, j}}$
      for all $i, j \in I$,
  \end{itemize}
  then there exists a unique
  $s \in \mathcal{F}(U)$ such that
  $s|_{U_i} = s_i$ for every $i \in I$.
\end{definition}

\begin{remark}
  The presheaf
  $\underline{A}^{\mathrm{pre}}$
  is not a sheaf in general.
  All other examples above are sheaves.
\end{remark}

\pagebreak
\begin{definition}
  A \emph{morphism} of (pre)sheaves
  $\varphi : \mathcal{F} \to \mathcal{G}$
  on a topological space $X$ is
  the data of group homomorphisms
  $\varphi(U) : \mathcal{F}(U) \to \mathcal{G}(U)$
  for each $U \subseteq X$ open
  such that for all $V \subseteq U \subseteq X$, the following
  diagram commutes:
  \[
    \begin{tikzcd}
      \mathcal{F}(U) \ar[d, swap, "\mathrm{res}"] \ar[r, "{\varphi(U)}"]
      & \mathcal{G}(U) \ar[d, "\mathrm{res}"] \\
      \mathcal{F}(V) \ar[r, swap, "{\varphi(V)}"]
      & \mathcal{G}(V)
    \end{tikzcd}
  \]
\end{definition}

\begin{example}
  Let $X$ be a variety.
  \begin{enumerate}
    \item If
      $a : \EE \to \mathbb{F}$ is a morphism
      of vector bundles on $X$, then
      we get a morphism of sheaves
      $\mathcal{E} \to \mathcal{F}$
      by $s \mapsto a \circ s \in \mathcal{F}(U)$
      for $s \in \mathcal{E}(U)$.
    \item A closed subvariety
      $Z \subseteq X$ induces a
      morphism $\ell_Z \to \OO_X$
      given by inclusion.
  \end{enumerate}
\end{example}

\begin{remark}
  Given a morphism of (pre)sheaves
  and $\varphi : \mathcal{F} \to \mathcal{G}$
  and $p \in X$, we get an
  induced morphism
  \begin{align*}
    \mathcal{F}_p &\longrightarrow \mathcal{G}_p \\
    (s, U) &\longmapsto
    (\varphi(s), U).
  \end{align*}
\end{remark}
