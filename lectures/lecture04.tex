\chapter{Jan.~22 --- Sheaves}

\section{Sheafification}

\begin{theorem}[Sheafification]
  For a presheaf $\mathcal{F}$ on
  a topological space $X$, there
  exists a morphism
  to a sheaf $i : \mathcal{F} \to \mathcal{F}^+$
  such that for any morphism to a sheaf
  $g : \mathcal{F} \to \mathcal{G}$,
  there exists a unique morphism
  $g^+ : \mathcal{F}^+ \to \mathcal{G}$
  such that
  $g = g^+ \circ i$, i.e. the following
  diagram commutes:
  \[
    \begin{tikzcd}
      \mathcal{F} \ar[d, "i", swap] \ar[r, "g"] & \mathcal{G} \\
      \mathcal{F}^+ \ar[ur, swap, "g^+", dashed]
    \end{tikzcd}
  \]
  In the above, $\mathcal{F}^+$
  is called the \emph{sheafification}
  of $\mathcal{F}$, and the pair
  $(i, \mathcal{F}^+)$ is unique
  up to isomorphism (as a consequence
  of the universal property).
\end{theorem}

\begin{proof}
  We first define
  $\mathcal{F}^+(U) = \{t : U \to \bigsqcup_{p \in X} \mathcal{F}_p : \text{(1) and (2) hold}\}$, where
  \begin{enumerate}
    \item $t(p) \in \mathcal{F}_p$;
    \item for any $x \in X$, there
      is an open set
      $x \in V_x \subseteq U$
      with $s \in \mathcal{F}(V_x)$
      such that $t(p) = s_p$ for
      all $p \in V_x$.
  \end{enumerate}
  It is straightforward to see that
  $\mathcal{F}^+$ is a sheaf and
  that
  \begin{align*}
    \mathcal{F}(U) &\longrightarrow \mathcal{F}^+(U) \\
    s
    &\longmapsto (X \ni p \mapsto s_p \in \mathcal{F}_p).
  \end{align*}
  gives a morphism
  $i : \mathcal{F} \to \mathcal{F}^+$.
  Now we check the universal property.
  Given a morphism
  $g : \mathcal{F} \to \mathcal{G}$
  with $\mathcal{G}$ a sheaf, we
  need to define $g^+ : \mathcal{F}^+ \to \mathcal{G}$.
  Fix $t \in \mathcal{F}^+(U)$.
  By definition, there exists an open
  cover  $\{U_i\}$ of $U$ and
  $s_i \in \mathcal{F}(U_i)$ such that
  $t(p) = (s_i)_p \in \mathcal{F}_p$
  for all $p \in U_i$.
  Set $t_i' := g(t_i)$. Note that
  \[
    (t_i')_p = g_p(t_p)
    = (t_j')_p
    \in \mathcal{G}_p
  \]
  for every $p \in U_i \cap U_j$.
  Since $\mathcal{G}$ is a sheaf, we
  get $t_i'|_{U_i \cap U_j} = t_j'|_{U_i \cap U_j}$.
  Thus there exists a unique
  $t' \in \mathcal{G}(U)$
  such that $t'|_{U_i} = t_i$
  for every $i \in I$. Then
  we can set $g^+(t) = t'$.
  One can check as an exercise
  that this gives a morphism
  $\mathcal{F}^+ \to \mathcal{G}$
  satisfying the universal property.
\end{proof}

\begin{example}
  We have the following:
  \begin{enumerate}
    \item If $\mathcal{F}$ is a
      sheaf, then $\mathcal{F} \to \mathcal{F}^+$
      is an isomorphism.
    \item For an abelian group
      $A$ and topological space
      $X$, we have
      $(\underline{A}^{\mathrm{pre}})^+ \cong \underline{A}$.
  \end{enumerate}
\end{example}

\begin{remark}\label{rem:properties-of-sheafification}
  We have the following:
  \begin{enumerate}
    \item If $p \in X$, the induced
      morphism on stalks
      $i_p : \mathcal{F}_p \to \mathcal{F}_p^+$ is
      an isomorphism for all
      $p \in X$.

      To construct the inverse map,
      consider $\mathcal{F}^+_p \to \mathcal{F}_p$
      defined by $(t, U) \mapsto t_p$
      for $t \in \mathcal{F}^+(U)$
      Check as an exercise that this
      is well-defined and is inverse
      to $i_p$.
    \item If $\mathcal{F} \subseteq \mathcal{G}$
      is a subpresheaf (i.e.
      $\mathcal{F}(U) \subseteq \mathcal{G}(U)$
      and $\rho_{V, U}^{\mathcal{F}} = \rho_{V, U}^{\mathcal{G}}|_{\mathcal{F}(U)}$
      for all $V \subseteq U \subseteq X$)
      and $\mathcal{G}$ is a sheaf,
      then we could alternatively
      define $\mathcal{F}^+$ as
      \[
        \mathcal{F}^+(U)
        = \{
          s \in \mathcal{G}(U)
          : \text{for all } x \in X,\,
          \text{there exists }
          x \in U_x \subseteq U
          \text{ open such that }
          s|_{U_x} \in \mathcal{F}(U_x)
        \}.
      \]
  \end{enumerate}
\end{remark}

\section{Kernel, Image, Cokernel for Sheaves}

\begin{remark}
  We want the following notions for
  sheaves:
  \begin{itemize}
    \item kernel, image, cokernel;
    \item short exact sequences;
    \item injectivity and surjectivity.
  \end{itemize}
\end{remark}

\begin{example}
  We want the following to be
  short exact sequences of sheaves:
  \begin{itemize}
    \item for $X$ a variety and
      $Z \hookrightarrow X$ a
      closed subvariety,
      \[
      \begin{tikzcd}
        0 \ar[r] & \mathcal{I}_Z \ar[r] & \OO_X \ar[r] & i_x \OO_Z \ar[r] & 0
      \end{tikzcd}
      \]
    \item for $M$ a complex manifold
      (e.g. $\C^n$) with
      $\OO_M$ the sheaf of
      $\C$-valued holomorphic functions,
      \[
        \begin{tikzcd}
          0 \ar[r] & \underline{\Z} \ar[r, "2\pi i \times"] & \OO_M \ar[r, "\varphi \mapsto e^\varphi"] & \OO_M^\times \ar[r] & 0
        \end{tikzcd}
      \]
  \end{itemize}
\end{example}

\begin{remark}
  Let $\varphi : \mathcal{F} \to \mathcal{G}$
  be a morphism of sheaves on
  a topological space $X$.
\end{remark}

\begin{definition}
  The \emph{kernel} of $\varphi$
  is $(\ker \varphi)(U) = \ker(\varphi(U) : \mathcal{F}(U) \to \mathcal{G}(U))$.
\end{definition}

\begin{remark}[Properties of the kernel]
  It is straightforward to check that
  $\ker \varphi$ is a sheaf. Moreover:
  \begin{enumerate}
    \item $\ker \varphi$
      satisfies the following
      universal property: For any
      morphism to a sheaf $\alpha$
      such that $\varphi \circ \alpha = 0$,
      there exists a unique morphism
      $\alpha'$
      such that the following diagram commutes:
      \[
        \begin{tikzcd}
          \mathcal{F}' \ar[r, "\alpha"] \ar[rr, bend left=30, "0"] \ar[dr, "\alpha'", dashed, swap]& \mathcal{F} \ar[r, "\varphi"] & \mathcal{G} \\
                                                & \ker \varphi \ar[u]
        \end{tikzcd}
      \]
      To see this, use the universal
      property of the kernel in the
      category of abelian groups.
    \item Since filtered limits are
      exact, we have
      $(\ker \varphi_p) = (\ker \varphi)_p$
      for all $p \in X$.
  \end{enumerate}
\end{remark}

\begin{lemma}[Injectivity for sheaves]
  The following are equivalent:
  \begin{enumerate}
    \item $\varphi(U) : \mathcal{F}(U) \to \mathcal{G}(U)$
      is injective for all
      $U \subseteq X$ open;
    \item $\varphi_x : \mathcal{F}_x \to \mathcal{G}_x$
      is injective for all $x \in X$.
  \end{enumerate}
  We say that $\varphi$ is
  \emph{injective} if either of these
  equivalent conditions hold.
\end{lemma}

\begin{proof}
  $(1 \Rightarrow 2)$ This is clear.

  $(2 \Rightarrow 1)$ Fix $s \in \mathcal{F}(U)$
  with $\varphi(U)(s) = 0$. Then
  \[
    \varphi_p(s_p)
    = (\varphi(U)(S))_p
    = 0
  \]
  for all $p \in X$, so
  $s_p = 0$ for all $p \in U$, so
  $s = 0$ by homework from Algebraic
  Geometry I.
\end{proof}

\begin{example}[Subtleties for the image]
  Consider the following:
  \begin{enumerate}
    \item Let $\varphi : \OO_{\C^n} \xrightarrow{\exp} \OO_{\C^n}^\times$.
      Then $U \mapsto \im(\OO_{\C^n}(U) \to \OO_{\C^n}^\times(U))$
      is a presheaf but not a sheaf.
      This is because logarithms
      only exist locally.
    \item Define
      $\varphi : \OO_{\PP^n} \to i_{p_1} \underline{k} \oplus i_{p_2} \underline{k}$
      by $f \mapsto (f(p_1), f(p_2))$.
      Again
      $U \mapsto \im(\varphi(U))$
      is not a sheaf.
  \end{enumerate}
\end{example}

\begin{definition}
  Let $\widetilde{\im}\, \varphi = (U \mapsto \im(\varphi(U))$.
  This is a presheaf and
  $\widetilde{\im}\, \varphi \subseteq \mathcal{G}$.
  Then the \emph{image} of $\varphi$
  is $\im \varphi = (\widetilde{\im}\, \varphi)^+$.
  By Remark \ref{rem:properties-of-sheafification},
  we can equivalently define
  \[
    (\im \varphi)(U)
    = \{
      s \in \mathcal{G}(U)
      : \text{there exists cover }
      \{U_i\} \text{ of } U
      \text{ such that }
      s|_{U_i} \in \im(\mathcal{F}(U_i) \to \mathcal{G}(U_i))
    \}.
  \]
\end{definition}

\begin{remark}
  We have $\im(\varphi_x) \cong (\widetilde{\im}\, \varphi)_x \cong (\im \varphi)_x$,
  where the first isomorphism
  is because filtered direct limits
  are exact and the second isomorphism
  is because sheafification
  preserves stalks.
\end{remark}

\begin{definition}
  Let $\widetilde{\coker}(\varphi) = (U \mapsto \coker(\varphi(U)))$.
  Then the \emph{cokernel} of
  $\varphi$ is $\coker(\varphi) = (\widetilde{\coker}(\varphi))^+$.
\end{definition}

\begin{remark}
  We have the following:
  \begin{enumerate}
    \item $\coker(\varphi)_x \cong \coker(\varphi_x)$
      (similar to above).
    \item $\coker(\varphi)$
      satisfies the universal
      property of the cokernel:
      \[
        \begin{tikzcd}
          \mathcal{F} \ar[r, "\varphi"] \ar[rr, "0", bend left=30]
          & \mathcal{G} \ar[r, "p_0"] \ar[d]
          & \mathcal{G}' \\
          & \widetilde{\coker}(\varphi)
          \ar[r] \ar[ur, dashed] & \coker(\varphi) \ar[u, dashed]
        \end{tikzcd}
      \]
    \item For a subsheaf
      $\mathcal{F}' \subseteq \mathcal{F}$,
      we can define the \emph{quotient sheaf}
      $\mathcal{F}/\mathcal{F}' = \coker(\mathcal{F'} \hookrightarrow \mathcal{F})$.
    \item By the universal property
      of the cokernel, we get natural
      maps
      \[
        \begin{tikzcd}
          \ker \varphi \ar[r] & \mathcal{F} \ar[r] \ar[d] & \im \mathcal{F} \\
          & \mathcal{F} / {\ker \varphi} \ar[ur, dashed, "\alpha", swap]
        \end{tikzcd}
      \]
      As the following diagram commutes,
      \[
        \begin{tikzcd}
          (\mathcal{F} / {\ker \varphi})_p \ar[r, "\alpha_p"] \ar[d, "\cong"] & (\im \varphi)_p \ar[d, "\cong", swap] \\
          \mathcal{F}_p / {(\ker \varphi)_p} \ar[r, "\cong", swap] & \im(\varphi_p)
        \end{tikzcd}
      \]
      $\alpha_p$
      is an isomorphism for all
      $p \in X$. So by HW,
      $\alpha$ is an isomorphism.
      So $\mathcal{F} / {\ker \varphi} \cong \im \varphi$.
  \end{enumerate}
\end{remark}

\begin{lemma}[Surjectivity for sheaves]
  The following are equivalent:
  \begin{enumerate}
    \item $\coker \varphi = 0$;
    \item $\im \varphi = \mathcal{G}$;
    \item $\varphi_x : \mathcal{F}_x \to \mathcal{G}_x$
      is surjective for all $x \in X$.
  \end{enumerate}
  We say that $\varphi$ is
  \emph{surjective} if any of these
  equivalent conditions hold.
\end{lemma}

\begin{proof}
  $(3 \Leftrightarrow 1)$
  We have (3) if and only if
  $\coker(\varphi_x) = 0$
  for all $x \in X$, if and only if
  $(\coker \varphi)_x = 0$ for all
  $x \in X$, if and only if (1).

  $(3 \Leftrightarrow 2)$
  We have (3) if and only if
  $\coker(\varphi_x) = 0$ for all
  $x \in X$, if and only if
  $\im \varphi_x = \mathcal{G}_x$
  for all $x \in X$, if and only if
  $(\im \varphi)_x \rightarrow \mathcal{G}_x$
  is an isomorphism for all
  $x \in X$, if and only if (2) by HW.
\end{proof}

\begin{remark}
  Note that if $\varphi(U) : \mathcal{F}(U) \to \mathcal{G}(U)$
  is surjective for all $U \subseteq X$, then
  $\varphi$ is surjective.
  However, the converse is false
  in general.
\end{remark}

\begin{definition}
  A sequence of morphisms of
  sheaves
  \[
    \begin{tikzcd}
      \mathcal{F} \ar[r, "f"]
      & \mathcal{G} \ar[r, "g"]
      & \mathcal{H}
    \end{tikzcd}
  \]
  is \emph{exact} at $\mathcal{G}$
  if $\ker g = \im f$.
\end{definition}

\begin{lemma}
  The following are equivalent:
  \begin{enumerate}
    \item $\ker g = \im f$;
    \item $\ker g_x = \im f_x$
      for all $x \in X$.
  \end{enumerate}
\end{lemma}

\begin{proof}
  Similar to above.
\end{proof}

\section{Constructions with Sheaves}

\begin{definition}
  Let $\mathcal{F}_1, \mathcal{F}_2$
  be sheaves on $X$. Then
  $\mathcal{F}_1 \oplus \mathcal{F}_2$
  is a sheaf defined by
  \[
    U \longmapsto
    \mathcal{F}_1(U) \oplus \mathcal{F}_2(U).
  \]
  This is a \emph{biproduct} in
  the category of sheaves.
\end{definition}

\begin{example}
  Let $X$ be a variety with
  connected components $X_1, \dots, X_n$.
  Then
  \[
    \OO_X \cong \OO_{X_1} \oplus \cdots \oplus \OO_{X_n}
  \]
  as sheaves of abelian groups.
\end{example}

\begin{definition}
  Let $U \subseteq X$ be open.
  Then $\mathcal{F}_i|_U$
  is a sheaf on $U$ given
  by $V \mapsto \mathcal{F}_i(V)$
\end{definition}

\begin{definition}
  $\Hom(\mathcal{F}_1, \mathcal{F}_2)$
  is the sheaf
  $U \mapsto \Hom^{\mathrm{sheaves}}(\mathcal{F}_1|_U, \mathcal{F}_2|_U)$.
\end{definition}

\begin{definition}[Gluing]
  Let $\{U_i\}$ be an open cover of
  $X$ with a sheaf
  $\mathcal{F}_i$ on each
  $U_i$ and
  isomorphisms
  $\alpha_{i, j} : \mathcal{F}_j|_{U_{i, j}} \to \mathcal{F}_i|_{U_{i, j}}$
  such that $\alpha_{i, j} = \alpha_{j, i}^{-1}$,
  $\alpha_{i, j} \circ \alpha_{j, k} = \alpha_{i, k}$, and
  $\alpha_{i, i} = \id$. Then there
  exists a sheaf $\mathcal{F}$
  with isomorphisms
  $\beta_i : \mathcal{F}|_{U_i} \to \mathcal{F}_i$
  such that the following
  diagram commutes:
  \[
    \begin{tikzcd}
      \mathcal{F}|_{U_{i, j}}
      \ar[r, "\id"] \ar[d, swap, "\beta_j"] & \mathcal{F}|_{U_{i, j}} \ar[d, "\beta_i"] \\
      \mathcal{F}_i|_{U_{i, j}} \ar[r, "\alpha_{i, j}", swap]
      & \mathcal{F}_j|_{U_{i, j}}
    \end{tikzcd}
  \]
  One can define $\mathcal{F}$
  as follows and check that it
  satisfies the above properties:
  \[
    \mathcal{F}(U)
    = \{(s_i)_{i \in I} : s_i \in \mathcal{F}(U_i) \text{ and } \alpha_{i, j}(s_j|_{U_{i, j}}) = s_i\}.
  \]
\end{definition}
