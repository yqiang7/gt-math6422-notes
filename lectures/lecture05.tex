\chapter{Jan.~27 --- \texorpdfstring{$\OO_X$}{OX}-Modules}

\section{Sheaves and Continuous Maps}

\begin{remark}
  The category $\mathrm{Sh}_X$
  of sheaves of abelian
  groups on a topological space $X$
  is an abelian category, i.e.
  it has or satisfies the following:
  \begin{itemize}
    \item zero object (the $\underline{0}$
      sheaf);
    \item $\Hom(\mathcal{F}, \mathcal{G})$
      is an abelian group and
      composition is bilinear;
    \item finite biproducts exist;
    \item kernels and cokernels exist;
    \item the image
      coincides with the coimage.
  \end{itemize}
\end{remark}

\begin{remark}
  For the rest of this section, let
  $f : X \to Y$ be a continuous
  map of topological spaces, $\mathcal{F}$
  a sheaf on $X$, and $\mathcal{G}$ a
  sheaf on $Y$.
\end{remark}

\begin{definition}
  The \emph{pushforward} $f_* \mathcal{F}$
  is the sheaf on $Y$ defined by
  \[
    V \longmapsto \mathcal{F}(f^{-1}(V)).
  \]
\end{definition}

\begin{example}
  We have the following:
  \begin{itemize}
    \item If $i : \{p\} \hookrightarrow X$ and
    $A$ is an abelian group, then
    $i_* \underline{A}$ is the skyscraper sheaf
    on $X$ at $p$.
    \item If $X$ is a variety and
      $i : Z \hookrightarrow X$
      with $Z$ a closed subvariety, then
      \[
        i_* \OO_Z(U)
        = \{\varphi : U \cap Z \to k : \varphi \text{ is regular}\}.
      \]
  \end{itemize}
\end{example}

\begin{definition}
  Let $\widetilde{f^{-1}} \mathcal{G}(U) = \varinjlim_{V \supseteq f(U)} \mathcal{G}(V)$,
  which is a presheaf.
  The \emph{pullback} is
  $f^{-1} \mathcal{G} := (\widetilde{f^{-1}} \mathcal{G})^+$.
\end{definition}

\begin{example}
  Let $f : \{p\} \hookrightarrow X$.
  Then $\widetilde{f^{-1}} \mathcal{G} \cong \underline{\mathcal{G}_p}$,
  which is a sheaf, so
  $f^{-1} \mathcal{G} \cong \widetilde{f^{-1}} \mathcal{G} \cong \underline{\mathcal{G}_p}$.
\end{example}

\begin{example}
  If $i : U \hookrightarrow X$ is the
  inclusion of an open set, then
  $i^{-1} \mathcal{F} \cong \mathcal{F}|_U$.
\end{example}

\begin{remark}
  The pushforward $f_*$ and pullback $f^{-1}$ are functors:
  \[
    \begin{tikzcd}
      \mathrm{Sh}_X
      \ar[r, bend left, "f_*"]
      & \mathrm{Sh}_Y
      \ar[l, bend left, "f^{-1}"]
    \end{tikzcd}
  \]
  How are $f_*$ and $f^{-1}$ related?
  There are natural maps
  $f^{-1} f_* \mathcal{F} \to \mathcal{F}$ and
  $\mathcal{G} \to f_* f^{-1} \mathcal{G}$
  induced by:
  \begin{enumerate}
    \item For $U \subseteq X$, define
      \begin{align*}
        \widetilde{f^{-1}} f_* \mathcal{F}(U)
        = \varinjlim_{V \supseteq f(U) \text{ open}} \mathcal{F}(f^{-1}(V))
        &\longrightarrow \mathcal{F}(U) \\
        s &\longmapsto s|_U.
      \end{align*}
    \item For $V \subseteq Y$, define
      \[
        \mathcal{G}(V)
        \longrightarrow
        \varinjlim_{V \supseteq V' \supseteq f(f^{-1}(V)) \text{ open}} \mathcal{G}(V')
        = (f_* \widetilde{f^{-1}} \mathcal{G})(V).
      \]
      Note that $V \supseteq f(f^{-1}(V))$,
      so we can add the
      $V \supseteq V' \supseteq f(f^{-1}(V))$ condition.
  \end{enumerate}
  Another way to think about this is
  via adjoints.
\end{remark}

\begin{prop}
  For $\mathcal{F} \in \mathrm{Sh}_X$ and
  $\mathcal{G} \in \mathrm{Sh}_Y$,
  there exist functorial bijections
  \[
    \Hom_{\mathrm{Sh}_X}(f^{-1} \mathcal{G}, \mathcal{F})
    \longrightarrow
    \Hom_{\mathrm{Sh}_Y}(\mathcal{G}, f_* \mathcal{F}),
  \]
  i.e. $(f^{-1}, f_*)$ is an adjoint pair.
\end{prop}

\begin{proof}
  Given $\phi : f^{-1} \mathcal{G} \to \mathcal{F}$, using map (2) from above we get
  \[
    \begin{tikzcd}
      \mathcal{G}
      \ar[r, "(2)"]
      & f_* f^{-1} \mathcal{G}
      \ar[r, "f_* \phi"]
      & f_* \mathcal{F}.
    \end{tikzcd}
  \]
  Similarly, given
  $\psi : \mathcal{G} \to f_* \mathcal{F}$, using map (1) from above we get
  \[
    \begin{tikzcd}
      f^{-1} \mathcal{G}
      \ar[r, "f^{-1} \psi"]
      & f^{-1} f_* \mathcal{F}
      \ar[r, "(1)"]
      & \mathcal{F}.
    \end{tikzcd}
  \]
  One can (tediously) check that this
  gives a bijection.
\end{proof}

\begin{remark}
  The following are consequences of
  adjointness:
  \begin{itemize}
    \item $f_*$ is left exact, i.e.
      given an exact sequence
      \[
        \begin{tikzcd}
          0 \ar[r] & \mathcal{F}' \ar[r] & \mathcal{F} \ar[r] & \mathcal{F}'' \ar[r] & 0
        \end{tikzcd}
      \]
      in $\mathrm{Sh}_X$, we get an exact
      sequence
      \[
        \begin{tikzcd}
          0 \ar[r] & f_* \mathcal{F}' \ar[r] & f_* \mathcal{F} \ar[r] & f_* \mathcal{F}''.
        \end{tikzcd}
      \]
    \item $f^{-1}$ is right exact
      (defined similarly with the left
      $0$ missing).
  \end{itemize}
  One can also directly check these
  properties from the definitions.
\end{remark}

\section{Sheaves of \texorpdfstring{$\OO_X$}{OX}-Modules}

\begin{remark}
  To use tools from commutative algebra,
  we want to consider modules.
  For the
  rest of this section, let $X$ be a
  topological space with a sheaf of rings
  $\OO_X$ (e.g. $X$ a variety with
  $\OO_X$ the sheaf of regular functions,
  or $M$ a complex manifold with $\OO_M$
  the sheaf of holomorphic functions).
\end{remark}

\pagebreak

\begin{definition}
  A \emph{(pre)sheaf of $\OO_X$-modules}
  is a (pre)sheaf $\mathcal{F}$
  on $X$ such that for each $U \subseteq X$,
  $\mathcal{F}(U)$ has the structure of
  an $\OO_X(U)$-module compatible
  with restriction, i.e. such that
  \[
    (a \cdot s)|_V = a|_V \cdot s|_V
  \]
  for $V \subseteq U \subseteq X$ open,
  $a \in \OO_X(U)$, and $s \in \mathcal{F}(U)$.
\end{definition}

\begin{remark}
  We often say just ``$\OO_X$-module'' to mean a
  ``sheaf of $\OO_X$-modules.''
\end{remark}

\begin{definition}
  A \emph{morphism} of (pre)sheaves of $\OO_X$-modules
  $\phi : \mathcal{M} \to \mathcal{N}$
  is a morphism of presheaves such that
  $\mathcal{M}(U) \to \mathcal{N}(U)$ is a morphism of $\OO_X(U)$-modules
  for all $U \subseteq X$.
\end{definition}

\begin{example}
  We have the following:
  \begin{enumerate}
    \item $\OO_X$ has the structure of an
      $\OO_X$-module (similar to how a
      ring $A$ has the structure of an
      $A$-module).
    \item Let $X$ be a variety and
      $p : \EE \to X$ a vector bundle.
      Let $\mathcal{E}$ be the sheaf of
      sections of $p$, with
      \[
        [f \cdot s : U \to p^{-1}(U)]
        \in \mathcal{E}(U)
      \]
      as the product of
      $f \in \OO_X(U)$ and
      $[s : U \to p^{-1}(U)] \in \mathcal{E}(U)$.
      Then $\mathcal{E}$ is an $\OO_X$-module.
    \item[2'.] Let $\EE = \mathbbm{1}_X$,
      then $\mathcal{E}(U) \cong \OO_X(U)$
      as $\OO_X(U)$-modules. As the
      isomorphism is compatible
      with restriction, we have
      $\mathcal{E} \cong \OO_X$.
    \item Let $\mathcal{F}_1, \mathcal{F}_2$,
      be (pre)sheaves of $\OO_X$-modules.
      Then so
      is $\mathcal{F}_1 \oplus \mathcal{F}_2$.
    \item[2''.] If $\EE$ is a trivial
      vector bundle of rank $e$, then
      $\mathcal{E} \cong \OO_X^{\oplus e}$.
    \item If $\mathcal{F}$ is a presheaf
      of $\OO_X$-modules, then
      $\mathcal{F}^+$ is naturally
      a sheaf of $\OO_X$-modules (use the
      definition of $\mathcal{F}^+$ in the
      proof).
    \item If $\varphi : \mathcal{F} \to \mathcal{G}$
      is a morphism of sheaves of
      $\OO_X$-modules, then
      $\ker \varphi, \im \varphi, \coker \varphi$
      are sheaves of $\OO_X$-modules
      (use that $\ker \varphi, \widetilde{\im}\, \varphi, \widetilde{\coker}\, \varphi$ are presheaves of $\OO_X$-modules and then use (4)).

      Furthermore, the category
      $\mathrm{Mod}_{\OO_X}$ of sheaves of
      $\OO_X$-modules is an abelian.
    \item We can define the usual
      constructions on
      $\OO_X$-modules:
      $\otimes$, $\Sym^d$, $\wedge^d$, etc.
  \end{enumerate}
\end{example}

\begin{example}
  Let $\mathcal{F}, \mathcal{G}$ be
  $\OO_X$-modules. Then their
  tensor product $\mathcal{F} \otimes \mathcal{G}$
  is the sheafification of
  \[
    U \longmapsto \mathcal{F}(U) \otimes_{\OO_X(U)} \mathcal{G}(U).
  \]
\end{example}

\begin{definition}
  An $\OO_X$-module
  $\mathcal{F}$ is \emph{locally free}
  of rank $e$ if for any $p \in X$,
  there exists $p \in U \subseteq X$ open
  such that $\mathcal{F}|_U \cong \OO_U^{\oplus e}$.
  If $e = 1$, then we say that
  $\mathcal{F}$
\end{definition}

\begin{example}
  Let $X$ be a variety and
  $p : \EE \to X$ a vector bundle of rank $e$.
  For $p \in X$, there exists
  $p \in U \subseteq X$ open such that
  \[
    \begin{tikzcd}
      p^{-1}(U) \ar[r, "\cong"] \ar[d, "p", swap] & U \times \Affine^e \ar[dl, "\mathrm{pr}_1"]  \\
      U
    \end{tikzcd}
  \]
  Then $\mathcal{E}|_U \cong$ sheaf of
  sections of $U \times \Affine^e$
  $\cong \OO_U^{\oplus e}$.
\end{example}

\begin{remark}[Transition functions]
  Let $e = 1$ for simplicity, and
  $\mathcal{E}$ a locally free $\OO_X$-module
  of rank $e$. Then there exists an
  open cover $\{U_i\}$ of $X$ with
  isomorphisms $\alpha_i : \mathcal{E}|_{U_i} \to \OO_{U_i}^{\oplus e}$.
  So on $U_{i, j} = U_i \cap U_j$, we get
  isomorphisms
  $\alpha_{i, j} = \alpha_i \circ \alpha_j^{-1} : \OO_{U_{i, j}}^{\oplus e} \to \OO_{U_{i, j}}^{\oplus e}$.
\end{remark}
