\chapter{Jan.~29 --- \texorpdfstring{$\OO_X$}{OX}-Modules, Part 2}

\section{More on \texorpdfstring{$\OO_X$}{OX}-Modules}

\begin{remark}
  Recall that we have operations
  $\otimes$, $\oplus$,
  $\Sym^d$, $\wedge^d$,
  $\Homm(\cdot, \cdot)$ on
  $\OO_X$-modules.
\end{remark}

\begin{example}
  The sheaf $\Homm_{\OO_X}(\mathcal{F}, \mathcal{G})$
  is the sheafification of
  \[
    U \longmapsto
    \Hom_{\OO_X}(\mathcal{F}|_U, \mathcal{G}|_U).
  \]
  This is again an $\OO_X$-module.
\end{example}

\begin{example}
  We have $\mathcal{F}^\vee = \Homm_{\OO_X}(\mathcal{F}, \OO_X)$.
\end{example}

\begin{exercise}
  The invertible sheaves on $X$
  up to isomorphism forms a group
  with multiplication given by
  $\otimes$, identity $\OO_X$,
  and inverse
  $\mathcal{L}^{-1} = \mathcal{L}^\vee$.
\end{exercise}

\begin{remark}[Transition data]
  Let $\mathcal{L}$ be an invertible
  $\OO_X$-module. Then there exists an
  open cover $\{U_i\}$ of $X$ and
  isomorphisms
  $\alpha_i : \mathcal{L}|_{U_i} \to \OO_{U_i}$,
  so we get isomorphisms
  \[
    \alpha_{i, j}
    = \alpha_i \circ \alpha_j^{-1}
    : \OO_{U_{i, j}}
    \to \OO_{U_{i, j}}.
  \]
  For this to be an isomorphism,
  we must have
  $\alpha_{i, j}(U_{i, j})(1) = g_{i, j} \in \OO_X(U_{i, j})^\times$.
\end{remark}

\begin{prop}
  If $X$ is a variety, then there is a
  bijection
  \begin{align*}
    \{\text{line bundles on } X\} / {\cong}
    &\longrightarrow
    \{
      \text{invertible sheaves on } X
    \} / {\cong} \\
    \mathbb{L} &\longmapsto \mathcal{L}
  \end{align*}
\end{prop}

\begin{proof}
  To get the reverse map, fix
  an invertible $\OO_X$-module
  $\mathcal{L}$ with trivialization
  data $(U_i, g_{i, j})$. Send it
  to the line bundle with the same
  trivialization data.
  Check that this is well-defined
  as an exercise.

  To show that this gives an inverse,
  it suffices to show that if
  $\mathbb{L}$ is a line bundle
  with trivialization data
  $(U_i, g_{i, j})$, then the sheaf
  $\mathcal{L}$ of sections of
  $\mathbb{L}$ has the same
  trivialization data.
  The trivializations
  \begin{align*}
    \mathbb{L}|_{U_i}
    &\underset{\cong}{\overset{\phi_i}{\longrightarrow}}
    U_i \times \Affine^1
  \end{align*}
  give isomorphisms
  $U_{i, j} \times \Affine^1 \to U_{i, j} \times \Affine^1$
  by $(x, v) \mapsto (x, g_{i, j}(x) v)$.
  We get an isomorphism
  \begin{align*}
    \alpha_i :
    \mathbb{L}|_{U_i}
    &\underset{\cong}{\overset{\phi_i}{\longrightarrow}}
    \OO_{U_i}
  \end{align*}
  where $e_i = \alpha_i^{-1}(1) = [U_i \xrightarrow{x \mapsto (x, 1)} U_i \times \Affine^1 \xrightarrow{\phi_i^{-1}} \mathbb{L}|_{U_i}]$.
  Now we have
  \[
    \begin{tikzcd}[row sep=small]
      \OO_{U_{i, j}}
      \ar[r, "\alpha_j^{-1}"]
      & \mathcal{L}|_{U_{i, j}}
      \ar[r, "\alpha_i"]
      & \OO_{U_{i, j}} \\
      1 \ar[r, mapsto]
      & e_j = e_i g_{i, j}
      \ar[r, mapsto] & g_{i, j}.
    \end{tikzcd}
  \]
  So we get the same transition
  functions $(U_i, g_{i, j})$
  for $\mathcal{L}$.
\end{proof}

\begin{remark}
  Given a morphism of rings
  $\phi : A \to B$, we have functors
  \[
    \begin{tikzcd}
      \mathrm{Mod}_B
      \ar[r, bend left, "\Phi"]
      & \mathrm{Mod}_A
      \ar[l, bend left, "\Psi"]
    \end{tikzcd}
  \]
  given as follows:
  \begin{enumerate}
    \item \emph{Extension of scalars}:
      $\mathrm{Mod}_A \ni M \longmapsto M \otimes_A B \in \mathrm{Mod}_B$,
      where the multiplication by $B$
      is
      \[
        c (m \otimes b)
        = m \otimes (cb).
      \]
      For example, if
      $M = A^{\oplus I}$, then
      $M \otimes_A B = B^{\oplus I}$.
    \item \emph{Restriction of scalars}:
      $\mathrm{Mod}_B \ni N \longmapsto N_A \in \mathrm{Mod}_A$,
      where $N_A := N$ as
      abelian groups with
      \[
        a \cdot n = \phi(a) n
      \]
      as the multiplication by $A$.
  \end{enumerate}
\end{remark}

\begin{prop}
  There is a functorial bijection
  \begin{align*}
    \Hom_B(M \otimes_A B, N)
    &\longleftrightarrow
    \Hom_A(M, N_A) \\
    [f : M \otimes_A B \to N]
    &\longmapsto
    [m \mapsto f(m \otimes 1)] \\
    [m \otimes b \mapsto b \cdot g(m)]
    &\mathrel{\reflectbox{\ensuremath{\longmapsto}}} [g : M \to N_A]
  \end{align*}
  for $M \in \mathrm{Mod}_A$
  and $N \in \mathrm{Mod}_B$.
\end{prop}

\begin{remark}
  Given the result for rings, we
  want a similar statement for
  $\OO_X$-modules.
\end{remark}

\section{\texorpdfstring{$\OO_X$}{OX}-Modules and Continuous Maps}

\begin{definition}
  A \emph{morphism of ringed spaces}
  $(X, \OO_X) \to (Y, \OO_Y)$
  is the data of
  \begin{enumerate}
    \item a continuous map $f : X \to Y$,
    \item a morphism of sheaves
      of rings $f^\# : \OO_Y \to f_* \OO_X$.
  \end{enumerate}
\end{definition}

\begin{example}
  If $X \to Y$ is a morphism of
  varieties, then
  $\OO_Y \to f_* \OO_X$ is given
  for $U \subseteq Y$ open by
  \begin{align*}
    \OO_Y(U) &\longrightarrow
    \OO_X(f^{-1}(U)) \\
    \varphi &\longmapsto f^* \varphi.
  \end{align*}
\end{example}

\begin{remark}
  Our goal will be to define
  functors
  \[
    \begin{tikzcd}
      \mathrm{Mod}_{\OO_X}
      \ar[r, bend left, "f_*"]
      & \mathrm{Mod}_{\OO_Y}
      \ar[l, bend left, "f^{*}"]
    \end{tikzcd}
  \]
\end{remark}

\begin{remark}[Pushforward]
  Given an $\OO_X$-module $\mathcal{F}$,
  the sheaf pushforward
  $f_* \mathcal{F}$
  is naturally an $f_* \OO_X$-module.
  Via the map
  $f^\# : \OO_Y \to f_* \OO_X$,
  we get an $\OO_Y$-module structure
  on $f_* \mathcal{F}$.
  More concretely, for
  $U \subseteq Y$ open,
  $s \in f_* \mathcal{F}(U) = \mathcal{F}(f^{-1}(U))$,
  and $a \in \OO_Y(U)$, we can define
  \[
    a \cdot s
    = f^\#(U)(a) \cdot s.
  \]
\end{remark}

\begin{remark}[Pullback]
  Given an $\OO_Y$-module
  $\mathcal{G}$, we get an
  $f^{-1} \OO_Y$-module
  $f^{-1} \mathcal{G}$. By the
  adjoint property for
  $(f^{-1}, f_*)$, the morphism
  $\OO_Y \to f_* \OO_X$
  corresponds to a morphism
  $f^{-1} \OO_Y \to \OO_X$. So we get
  \[
    f^* \mathcal{G} := f^{-1} \mathcal{G} \otimes_{f^{-1} \OO_Y} \OO_X
  \]
  is an $\OO_X$-module.
  Thus we get a functor
  $f^* : \mathrm{Mod}_{\OO_Y} \to \mathrm{Mod}_{\OO_X}$.
\end{remark}

\begin{prop}
  The pair $(f^*, f_*)$ are
  adjoint functors.
\end{prop}

\begin{proof}
  Similar to before.
\end{proof}

\begin{example}\label{ex:pullbackofstructure}
  Recall that if $A \to B$ is a
  morphism of rings, then
  $A \otimes_A B \cong B$. In
  our setting, we get
  \[
    f^* \OO_Y
    = (f^{-1} \OO_Y \widetilde{\otimes}_{f^{-1} \OO_Y} \OO_X)^+
    \cong \OO_X^+ \cong \OO_X.
  \]
  Similarly, we have
  $f^*(\OO_Y^{\oplus I}) \cong (f^* \OO_Y)^{\oplus I} \cong \OO_X^{\oplus I}$
  (as left-adjoint functors
  commute with coproducts).
\end{example}

\begin{remark}
  If $\mathcal{E}$ is a locally
  free rank $m$ $\OO_X$-module,
  then $f^* \mathcal{E}$ is a
  locally free rank $m$ $\OO_Y$-module,
  (as $f^*$ can be computed locally
  on $Y$ using Example \ref{ex:pullbackofstructure}).
\end{remark}
