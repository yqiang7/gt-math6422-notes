\chapter{Feb.~3 --- Coherent Sheaves}

\section{Review of Localization}

\begin{remark}
  Let $A$ be a ring and $S \subseteq A$
  a multiplicative system (i.e.
  $1 \in S$, and $a, b \in S$ implies
  $ab \in S$).
  For example, we could take
  $S = \langle f \rangle = (1, f, f^2, \dots)$
  for $f \in A$ or $S = A \setminus \p$
  for a prime ideal $\p \le A$.
\end{remark}

\begin{definition}
  The \emph{localization} of $A$ at $S$
  is
  \[
    S^{-1} A = \{ a / s : a \in A, s \in S\},
  \]
  where $a / s = a' / s'$
  if and only if $t(a s' - a' s) = 0$
  for some $t \in S$.
\end{definition}

\begin{remark}
  The localization satisfies the
  following universal property:
  \[
    \begin{tikzcd}
      A \ar[r, "a \mapsto a / 1"] \ar[dr, "f", swap]
      & S^{-1} A \ar[d, dashed, "\exists !"] \\
      & T
    \end{tikzcd}
  \]
  whenever $f(S)$ lands in
  the units of $T$.
\end{remark}

\begin{definition}
  For an $A$-module $M$, the
  \emph{localization} of $M$ at $S$ is
  \[
    S^{-1} M = \{ m / s : m \in M, s \in S\},
  \]
  where $m / s = m' / s'$
  if and only if
  $t(s' m - s m') = 0$
  for some $t \in S$.
\end{definition}

\begin{remark}
  For $S = \langle f \rangle$, we
  will write $S^{-1} M = M_f$.
  For $S = A \setminus \p$,
  we will write
  $S^{-1} M = M_\p$.
\end{remark}

\begin{prop}\label{prop:localization-properties}
  We have the following properties
  for localization:
  \begin{enumerate}
    \item There is an isomorphism
      \begin{align*}
        M \otimes_A S^{-1} A
        &\overset{\cong}{\longrightarrow} S^{-1} M \\
        m \otimes (a / s)
        &\longmapsto (am) / s.
      \end{align*}
    \item Localization gives an
      exact functor
      \begin{align*}
        \mathrm{Mod}_A
        &\longrightarrow \mathrm{Mod}_{S^{-1} A} \\
        M &\longmapsto S^{-1} M.
      \end{align*}
    \item A sequence in
      $\mathrm{Mod}_A$
      \[
        \begin{tikzcd}
          0 \ar[r] & M' \ar[r] & M \ar[r] & M'' \ar[r] & 0
        \end{tikzcd}
      \]
      is exact if and only if the
      sequence
      \[
        \begin{tikzcd}
          0 \ar[r] & M'_\p \ar[r] & M_\p \ar[r] & M''_\p \ar[r] & 0
        \end{tikzcd}
      \]
      is exact for all
      maximal (equivalently, prime)
      ideals $\p \le A$.
  \end{enumerate}
\end{prop}

\begin{example}
  Recall that if $X$ is an affine
  variety, then
  \[
    \OO_X(D(f)) \cong A(X)_f
    \quad \text{and} \quad
    \OO_{X, x} \cong A(X)_{\m_x},
  \]
  where $\m_x = I(\{x\}) \le A(X)$.
\end{example}

\section{Coherent Sheaves on Affine Varieties}

\begin{remark}
  For this section, let
  $X$ be an affine variety and
  $A = \OO_X(X) = A(X)$. We want a
  functor
  \[
    \mathrm{Mod}_A
    \longrightarrow \mathrm{Mod}_{\OO_X}.
  \]
\end{remark}

\begin{theorem}\label{thm:unique-OX-module}
  For $M \in \mathrm{Mod}_A$,
  there exists a unique $\OO_X$-module
  $\widetilde{M}$ such that
  \begin{enumerate}
    \item $\widetilde{M}(D(f)) \cong M_f$;
    \item for $D(g) \subseteq D(f)$,
      we have
      \[
        \begin{tikzcd}
          \widetilde{M}(D(f))
          \ar[r] \ar[d, "=", swap]
          & \widetilde{M}(D(g)) \ar[d, "="] \\
          M_f
          \ar[r, "\mathrm{natural\, map}", swap]
          & M_g
        \end{tikzcd}
      \]
  \end{enumerate}
\end{theorem}

\begin{remark}
  How is the natural map
  $M_f \to M_g$ defined? If
  $D(g) \subseteq D(f)$, then we have
  $V(g) \supseteq V(f)$, so
  $\sqrt{(g)} \subseteq \sqrt{(f)}$.
  Thus $g \in \sqrt{(f)}$, so
  $g^d = fh$ for some $d > 0$ and
  $h \in A$. So we get a map
  \begin{align*}
    M_f &\longrightarrow M_g \\
    m / f^i
    &\longmapsto 
    mh^i / g^{d i}.
  \end{align*}
  Alternatively, if $D(g) \subseteq D(f)$,
  then $D(g) = D(gf)$, so we could
  instead consider
  \[
    \begin{tikzcd}
      \widetilde{M}(D(f))
      \ar[r] \ar[d, "=",swap]
      & \widetilde{M}(D(fg)) \ar[d] \\
      M_f
      \ar[r, "m / f^i \mapsto m g^i / (fg)^i", swap]
      & M_{fg}
    \end{tikzcd}
  \]
\end{remark}

\begin{remark}
  To construct $\widetilde{M}$,
  we need the notion of
  \emph{sheaves on a basis}. Let
  $(X, \OO_X)$ be a ringed space and
  $\mathcal{P}$ a collection of
  open sets in $X$ such that
  \begin{enumerate}
    \item $\mathcal{P}$ is a
      basis for $X$;
    \item  if $U, V \in \mathcal{P}$,
      then $U \cap V \in \mathcal{P}$.
  \end{enumerate}
\end{remark}

\begin{example}
  If $(X, \OO_X)$ is an affine
  variety, we can take $\mathcal{P} = \{D(f) : f \in A(X)\}$.
  Also, if $(X, \OO_X)$ is a
  general algebraic variety, then
  we can take $\mathcal{P}$
  to be the affine open subsets of $X$ 
  (as $X$ is separated, the intersection
  of two affine open subsets is
  again affine open).
\end{example}

\begin{definition}
  A \emph{$\mathcal{P}$-sheaf of $\OO_X$-modules}
  $\mathcal{F}$
  on $X$ is the data of:
  \begin{itemize}
    \item an $\OO_X(U)$-module
      $\mathcal{F}(U)$ for
      each $U \in \mathcal{P}$,
    \item homomorphisms of abelian
      groups $\mathcal{F}(U) \to \mathcal{F}(V)$
      for all $U, V \in \mathcal{P}$
      with $V \subseteq U$
  \end{itemize}
  satisfying the following properties:
  \begin{itemize}
    \item the multiplication is
      compatible with restriction,
    \item the sheaf axiom with respect
      to open sets in $\mathcal{P}$.
  \end{itemize}
\end{definition}

\begin{example}
  If $\mathcal{F}$ is a sheaf of
  $\OO_X$-modules, then we get
  $\mathcal{F}^{\mathcal{P}}$,
  a $\mathcal{P}$-sheaf of
  $\OO_X$-modules.
\end{example}

\begin{theorem}\label{thm:P-sheaf-equivalence}
  There is an equivalence of categories
  \begin{align*}
    \mathrm{Mod}_{\OO_X}
    &\longrightarrow
    \mathrm{Mod}_{\OO_X}^{\mathcal{P}} \\
    \mathcal{F}
    &\longmapsto
    \mathcal{F}^{\mathcal{P}}.
  \end{align*}
  In particular,
  $\Hom_{\OO_X}(\mathcal{F}, \mathcal{G}) \to \Hom(\mathcal{F}^{\mathcal{P}}, \mathcal{G}^{\mathcal{P}})$
  is a bijection, and for any
  $\mathcal{H} \in \mathrm{Mod}_{\OO_X}^{\mathcal{P}}$,
  there exists some $\mathcal{F} \in \mathrm{Mod}_{\OO_X}$
  such that $\mathcal{F}^{\mathcal{P}} \cong \mathcal{H}$.
\end{theorem}

\begin{proof}
  We construct the inverse functor.
  Take
  $\mathcal{H} \in \mathrm{Mod}_{\OO_X}^{\mathcal{P}}$, and
  define $\mathcal{F} \in \mathrm{Mod}_{\OO_X}$
  by setting
  \[
    \mathcal{F}(U)
    = \{(s_V)_{V \in \mathcal{P}, V \subseteq U} : s_V|_{V \cap V'} = s_V'|_{V \cap V'} \text{ for all } V, V' \in \mathcal{P} \text{ with } V, V' \subseteq U\}.
  \]
  One can check that this defines a
  functor and an equivalence of categories.
\end{proof}

\begin{remark}
  Returning to algebraic geometry,
  let $X$ be an affine variety,
  $A = A(X)$, and $M$ an $A$-module.
\end{remark}

\begin{prop}\label{prop:P-sheaf-from-module}
  For $\mathcal{P} = \{D(f) : f \in A\}$,
  the assignment $D(f) \mapsto M_f$
  with the natural restriction maps
  defines a $\mathcal{P}$-sheaf of $\OO_X$-modules.
\end{prop}

\begin{proof}
  See Mustata Lemma 8.3.2.
  The hard part is to check the
  sheaf axiom for $\mathcal{P}$, which
  is similar to the computation
  that $\OO_X(D(f)) \cong A_f$
  for an affine variety $X$.
\end{proof}

\begin{proof}[Proof of Theorem \ref{thm:unique-OX-module}]
  Combining
  Theorem \ref{thm:P-sheaf-equivalence}
  and Proposition
  \ref{thm:P-sheaf-equivalence},
  we get an $\OO_X$-module
  $\widetilde{M}$ such that
  $\widetilde{M}(D(f)) \cong M_f$,
  and it is unique up to isomorphism.
\end{proof}

\begin{example}
  We have
  $\widetilde{A} \cong \OO_X$, since
  \[
    \widetilde{A}(D(f))
    \cong A_f \cong \OO_X(D(f)).
  \]
  Similarly, one can check that
  $\widetilde{A^{\oplus I}} \cong \widetilde{\OO_X^{\oplus I}}$.
\end{example}

\begin{exercise}
  For $Z \subseteq X$ closed and
  $I(Z) \le A(X)$, we have
  $\widetilde{I(Z)} \cong \mathcal{I}_Z$ (the ideal sheaf of $Z$).
\end{exercise}

\begin{remark}
  We have the following:
  \begin{enumerate}
    \item For $x \in X$ with
      $\p := I(\{x\}) \le A$, we have
      $\widetilde{M}_x \cong \varinjlim_{D(f) \ni x} \widetilde{M}(D(f)) = \varinjlim_{f \in A \setminus \p} M_f \cong M_\p$.
    \item For an $A$-module
      homomorphism $\varphi : M \to N$,
      we get homomorphisms
      \[
        \begin{tikzcd}
          \widetilde{M}(D(f)) \ar[r]
          \ar[d, "\cong", swap]
          & \widetilde{N}(D(f)) \ar[d, "\cong"] \\
          M_f \ar[r, "\mathrm{natural\, map}", swap]
          & N_f
        \end{tikzcd}
      \]
      which are
      compatible with restriction.
      So we get a homomorphism
      of $\OO_X$-modules
      $\widetilde{M} \to \widetilde{N}$.
      One can check that this gives a
      functor
      $\Phi : \mathrm{Mod}_A \to \mathrm{Mod}_{\OO_X}$.
  \end{enumerate}
\end{remark}

\begin{prop}\label{prop:properties-of-tilde-functor}
  We have the following:
  \begin{enumerate}
    \item $\Phi$ is exact;
    \item $\Phi$ is fully
      faithful, i.e. the following map
      is a bijection:
      \begin{align*}
        \Hom_A(M, N)
        &\longrightarrow
        \Hom_{\OO_X}(\widetilde{M}, \widetilde{N}) \\
        \varphi &\longmapsto \widetilde{\varphi}.
      \end{align*}
  \end{enumerate}
\end{prop}

\begin{proof}
  (1)  If $0 \to M' \to M \to M'' \to 0$
  is exact, then
  $0 \to M'_\p \to M_\p \to M''_\p \to 0$
  is exact for every
  $\p \le A$ maximal by
  Proposition \ref{prop:localization-properties}(3).
  So the sequence
  \[
    \begin{tikzcd}
      0 \ar[r] & \widetilde{M}'_x
      \ar[r] & \widetilde{M}_x
      \ar[r] & \widetilde{M}''_x
      \ar[r] & 0
    \end{tikzcd}
  \]
  is exact for all
  $x \in X$. So
  $0 \to \widetilde{M}' \to \widetilde{M} \to \widetilde{M}'' \to 0$
  is exact.

  (2) We want a map
  $\Hom_{\OO_X}(\widetilde{M}, \widetilde{N}) \to \Hom_A(M, N)$.
  Given an $\OO_X$-module
  homomorphism
  $\varphi : \widetilde{M} \to \widetilde{N}$,
  we get
  $f = \varphi(X) : \widetilde{M}(X) = M \to \widetilde{N}(X) = N$.
  We want to show that
  $\widetilde{f} = \varphi$. We have
  \[
    \begin{tikzcd}
      \widetilde{M}(X)
      \ar[r, "\varphi(X)"] \ar[d]
      & \widetilde{N}(X) \ar[d] \\
      \widetilde{M}(D(g))
      \ar[r, "\varphi(D(f))", swap]
      & \widetilde{N}(D(g))
    \end{tikzcd}
  \]
  and this diagram commutes.
  So we get $\widetilde{f}(D(g)) = \varphi(D(g))$.
\end{proof}

\begin{remark}
  By
  Proposition \ref{prop:properties-of-tilde-functor}(1),
  given an $A$-module
  homomorphism $f : M \to N$, we have
  \[
    \ker(\widetilde{f})
    \cong \widetilde{\ker(f)},
    \quad \coker(\widetilde{f})
    \cong \widetilde{\coker(f)},
    \quad \im(\widetilde{f})
    \cong \widetilde{\im(f)}.
  \]
\end{remark}

\begin{prop}
  For $M, N \in \mathrm{Mod}_A$,
  we have
  \begin{enumerate}
    \item $\widetilde{M} \otimes_{\OO_X} \widetilde{N}
      \cong \widetilde{M \otimes_A N}$;
    \item $\widetilde{\Hom_A(M, N)}
      \cong \Homm_{\OO_X}(\widetilde{M}, \widetilde{N})$;
    \item $\widetilde{\bigoplus_{i \in I} M_i}
      \cong \bigoplus_{i \in I} \widetilde{M}_i$.
  \end{enumerate}
\end{prop}

\begin{proof}
  (1) We have a homomorphism
  \[
    \begin{tikzcd}
    M \otimes_A N
    \ar[r, "\cong"]
    &
    \Gamma(X, \widetilde{M}\, \widetilde{\otimes}_{\OO_X}\, \widetilde{N}) \ar[d] \\
    & \Gamma(X, \widetilde{M} \otimes_{\OO_X} \widetilde{N})
    \end{tikzcd}
  \]
  This gives a homomorphism
  of $\OO_X$-modules
  $\widetilde{M \otimes_A N} \to \widetilde{M} \otimes_{\OO_X} \widetilde{N}$.
  Now at the stalks for
  $x \in X$ with $\m = I(\{x\}) \le A(X)$,
  we can see that
  \[
    (\widetilde{M \otimes_A N})_x
    \cong (M \otimes_A N) \otimes_A A_\m
    \cong M_\m \otimes_{A_\m} N_\m.
  \]
  Similarly, we have
  \[
    (\widetilde{M} \otimes_{\OO_X} \widetilde{N})_x
    \cong \widetilde{M}_x \otimes_{\OO_{X, x}} \widetilde{N}_x
    \cong M_\m \otimes_{A_\m} N_\m.
  \]
  Thus we have isomorphisms
  at the stalks.
\end{proof}

\begin{remark}
  The functor $\Phi : \mathrm{Mod}_A \to \mathrm{Mod}_{\OO_X}$
  is left adjoint to
  $\mathrm{Mod}_{\OO_X} \to \mathrm{Mod}_A$
  given by $\mathcal{F} \mapsto \mathcal{F}(X)$.
\end{remark}
