\chapter{Feb.~5 --- Coherent Sheaves, Part 2}

\section{Coherent Sheaves on Affine Varieties, Continued}

\begin{prop}
  Let $\varphi : X \to Y$ be a morphism
  of affine varieties, and let
  \[
    A = \OO_Y(Y)
    \overset{\varphi^\#}{\longrightarrow}
    \OO_X(X) = B
  \]
  be the corresponding
  ring homomorphism on the coordinate
  rings. Then:
  \begin{enumerate}
    \item For $N \in \mathrm{Mod}_B$,
      we have $\varphi_* \widetilde{N} = \widetilde{N_A}$.
    \item For $M \in \mathrm{Mod}_A$,
      we have $\varphi^* \widetilde{M} = \widetilde{M \otimes_A B}$.
  \end{enumerate}
\end{prop}

\begin{proof}
  (1) For $f \in A = \OO_Y(Y)$,
  the left-hand side is given on $D(f)$
  by
  \[
    (\varphi_* \widetilde{N})(D(f))
    = \widetilde{N}(\varphi^{-1}(D(f)))
    = \widetilde{N}(D(\varphi^\# f))
    = N_{\varphi^\# f}
    = (N_A)_f.
  \]
  So we get that
  $\varphi_* \widetilde{N} = \widetilde{N_A}$.

  (2) (Hartshorne says this holds
  by definition lol.)
  First assume $M = A^{\oplus I}$. Then
  \[
    \varphi^* \widetilde{M}
    = \varphi^*(\OO_Y^{\oplus I})
    = (\varphi^* \OO_Y)^{\oplus I}
    = \OO_X^{\oplus I}
    = \widetilde{B^{\oplus I}}
    = \widetilde{M \otimes_A B}.
  \]
  For an arbitrary $A$-module $M$,
  we use the following:
  \begin{quote}
    \textbf{Claim:} There exists an
    exact sequence
    \[
      \begin{tikzcd}
        A^{\oplus J} \ar[r, "\alpha"]
        & A^{\oplus I} \ar[r, "\beta"]
        & M \ar[r]
        & 0
      \end{tikzcd}
    \]

    \begin{proof}[Proof of claim]
      Choose generators $(m_i)_{i \in I}$ for
      $M$, and set
      $\beta(e_i) = m_i$. Choose
      generators $(n_j)_{j \in J}$
      for $\ker \beta$. Then we can set
      $\alpha(f_j) = n_j$.
    \end{proof}
  \end{quote}
  Apply $\cdot \otimes_A B$ (which is
  right exact)
  to the
  exact sequence from the claim to
  get
  \[
    \begin{tikzcd}
      B^{\oplus J} \ar[r]
      & B^{\oplus I} \ar[r]
      & M \otimes_A B \ar[r]
      & 0.
    \end{tikzcd}
  \]
  Applying $\varphi^* (\, \widetilde{\cdot}\,)$
  (note that $\widetilde{\cdot}$
  is exact and $\varphi^*$ is right exact)
  to the original sequence to get
  \[
    \begin{tikzcd}[row sep=small]
      \varphi^*(\widetilde{A^{\oplus J}}) \ar[r]
      & \varphi^*(\widetilde{A^{\oplus I}}) \ar[r]
      & \varphi^* \widetilde{M} \ar[r]
      & 0 \\
      B^{\oplus J} \ar[u, equal]
      & B^{\oplus I} \ar[u, equal]
    \end{tikzcd}
  \]
  using the previous case.
  Thus $\varphi^* \widetilde{M} = \coker(\widetilde{B^{\oplus J}} \to \widetilde{B^{\oplus I}}) = (\coker(B^{\oplus J} \to B^{\oplus I}))^{\sim} = \widetilde{M \otimes_A B}$.
\end{proof}

\section{Quasicoherent and Coherent Sheaves}
\begin{remark}
  For the rest of this lecture,
  assume $(X, \OO_X)$ is a variety
  (not necessarily affine).
\end{remark}

\begin{definition}
  An $\OO_X$-module $\mathcal{F}$
  is \emph{quasicoherent}
  if there exists an affine cover
  $X = \bigcup_{i \in I} U_i$
  such that $\mathcal{F}|_{U_i} \cong \widetilde{M}_i$
  for some $\OO_X(U_i)$-module $M_i$.
  It is \emph{coherent} if the
  $M_i$ are finitely generated
  $\OO_X(U_i)$-modules.
\end{definition}

\begin{remark}
  We have $M_i \cong \widetilde{M}_i(U_i) \cong \mathcal{F}(U_i)$,
  so we may replace the
  $\mathcal{F}|_{U_i} \cong \widetilde{M}_i$
  condition with one of the
  following:
  $\mathcal{F}|_{U_i} \cong \widetilde{\mathcal{F}(U_i)}$,
  or $\mathcal{F}(U_i)_f \to \mathcal{F}(D_{U_i}(f))$
  is an isomorphism for
  all $f \in \OO_X(U_i)$.
\end{remark}

\begin{example}
  We have the following:
  \begin{enumerate}
    \item If $\mathcal{E}$ is a
      locally free sheaf of rank $r$ 
      on $X$, then there exists an
      open cover $X = \bigcup U_i$
      such that $\mathcal{E}|_{U_i} \cong \OO_X^{\oplus r}|_{U_i}$.
      Refining the cover, we may
      assume the $U_i$ are affine,
      so
      \[
        \mathcal{E}|_{U_i}
        = \widetilde{\OO_X(U_i)^{\oplus r}}.
      \]
      Thus we see that
      $\mathcal{E}$ is coherent.
    \item For $Z \hookrightarrow X$
      a closed embedding,
      $\mathcal{I}_Z \subseteq \OO_X$
      is coherent.
    \item For $\mathcal{F}$ a
      (quasi)coherent $\OO_X$-module
      and $U \subseteq X$ open,
      $\mathcal{F}|_U$ is
      (quasi)coherent.

      To see this, use that if
      $\mathcal{F}|_{U_i} \cong \widetilde{M}_i$
      for $M_i$ an $\OO_X(U_i)$-module
      and $f \in \OO_X(U_i)$, then
      \[\mathcal{F}|_{D_{U_i}(f)} \cong \widetilde{(M_{i})_f}.\]
      Furthermore, if $M_i$
      is finitely generated, then so
      is $(M_i)_f$. So by
      refining our open affine cover in
      with principal
      opens (which form a basis)
      we may assume $U = \bigcup_{i, U_i \subseteq U} U_i$.
      This gives the result.
  \end{enumerate}
\end{example}

\begin{prop}[Key proposition]\label{prop:qc-coh-equiv}
  Let $\mathcal{F} \in \mathrm{Mod}_{\OO_X}$.
  The following are equivalent:
  \begin{enumerate}
    \item $\mathcal{F}$ is
      quasicoherent (resp. coherent).
    \item For any affine open set
      $U \subseteq X$, we have
      $\mathcal{F}|_U \cong \widetilde{\mathcal{F}(U)}$
      (with $\mathcal{F}(U)$ finitely
      generated in the coherent case).
  \end{enumerate}
\end{prop}

\begin{lemma}[Affine communication]\label{lem:affine-communication}
  If $X$ is a variety,
  $U, V \subseteq X$ affine open
  subsets, and $p \in U \cap V$,
  then there exists an open set
  $p \in W \in U \cap V$ that is a
  principal open of both $U$ and $V$.
\end{lemma}

\begin{proof}
  Choose a principal open of $U$
  with $p \in W_1 = D_U(h) \subseteq U \cap V$
  for some $h \in \OO_X(U)$,
  and choose a principal open
  of $V$ with $p \in W = D_V(g) \subseteq W_1$
  for some $g \in \OO_X(V)$. Now
  \[
    g|_{W_1}
    = \OO_X(W_1)
    = \OO_X(U)_h,
  \]
  so $g|_{W_1} = f / h^i$ for some
  $f \in \OO_X(U)$ and $i \ge 0$. Now
  $D_V(g) = W = D_{W_1}(g|_{W_1}) = D_U(f h)$.
\end{proof}

\begin{proof}[Proof of Proposition \ref{prop:qc-coh-equiv}]
  $(2 \Rightarrow 1)$ This is clear.

  $(1 \Rightarrow 2)$ Assume
  $\mathcal{F} \in \mathrm{QCoh}_X$.
  So there exists an open affine
  cover $\{U_i\}$ such that
  $\mathcal{F}|_{U_i} = \widetilde{\mathcal{F}(U_i)}$.
  Fix $U \subseteq X$ affine open. By
  refining the cover, we may assume
  that $U = \bigcup_{U_i \subseteq U} U_i$.
  Now replacing $X$ with $U$, we may
  assume that $X = U$. Using
  Lemma \ref{lem:affine-communication},
  we may assume $U_i = D(f_i)$
  for some $f_i \in \OO_X(X)$.

  So now
  $X$ is affine, $X = \bigcup_{i = 1}^r D(f_i)$,
  and $\mathcal{F}|_{D(f_i)} \cong \widetilde{\mathcal{F}(D(f_i))}$.
  We want to show that
  for any $f \in A$, the natural
  map $\mathcal{F}(X)_f \to \mathcal{F}(D(f))$
  is an isomorphism (this would imply
  $\mathcal{F} \cong \widetilde{\mathcal{F}(X)}$). Now
  \[
    \begin{tikzcd}
      0 \ar[r]
      & \mathcal{F}(X)_f \ar[r] \ar[d, "\alpha"]
      & \bigoplus_i \mathcal{F}(D(f_i))_f \ar[r] \ar[d, "\beta"]
      & \bigoplus_{i, j} \mathcal{F}(D(f_i f_j))_f \ar[d, "\gamma"] \\
      0 \ar[r]
      & \mathcal{F}(D(f)) \ar[r]
      & \bigoplus_i \mathcal{F}(D(f_i f))
      \ar[r]
      & \bigoplus_{i, j} \mathcal{F}(D(f_i f_j f))
    \end{tikzcd}
  \]
  where the first row is exact by
  the sheaf property and using that
  localization is exact, and the
  second row is exact by the sheaf
  property. Note that
  $\beta$ and $\gamma$
  are isomorphisms as
  \[
    \mathcal{F}|_{D(f_i)}
    \cong \widetilde{\mathcal{F}(D(f_i))}
    \quad \text{and} \quad
    \mathcal{F}|_{D(f_i f_j)}
    \cong \widetilde{\mathcal{F}(D(f_i f_j))}.
  \]
  So by the five lemma,
  $\alpha$ is an isomorphism, and
  thus $\mathcal{F} \cong \widetilde{\mathcal{F}(X)}$.
\end{proof}

\begin{remark}
  For the coherent case, use the
  fact that if $M$ is an $A$-module,
  $A = (f_1, \dots, f_r)$,
  and $M_{f_i}$ is finitely
  generated for $i = 1, \dots, r$, then
  $M$ is finitely generated.
\end{remark}

\begin{prop}
  We have the following:
  \begin{enumerate}
    \item If $\varphi : \mathcal{F} \to \mathcal{G}$
      is a morphism of
      (quasi)coherent sheaves, then
      $\ker \varphi,
      \im \varphi,
      \coker \varphi$
      are (quasi)coherent.
    \item If $\mathcal{F}$ and
      $\mathcal{G}$ are
      (quasi)coherent, then so
      are $\mathcal{F} \otimes \mathcal{G}$
      and $\Homm(\mathcal{F}, \mathcal{G})$.
    \item If $\mathcal{F}_i$ is
      quasicoherent for
      $i \in I$, then
      $\bigoplus_{i \in I} \mathcal{F}_i$
      is quasicoherent.
      Furthermore, if $\mathcal{F}_i$
      is coherent and $|I| < \infty$,
      then $\bigoplus_{i \in I} \mathcal{F}_i$
      is coherent.
  \end{enumerate}
\end{prop}

\begin{proof}
  (1) Choose $U \subseteq X$ affine
  open. So we can write
  $\mathcal{F}|_U = \widetilde{M}$
  and $\mathcal{G}|_U = \widetilde{N}$.
  Furthermore, $\varphi|_U = \widetilde{\alpha}$
  for some $\OO_X(U)$-module
  homomorphism $\alpha : M \to N$.
  Now
  \[
    (\ker \varphi)|_U
    = \ker \varphi|_U
    = \ker \widetilde{\alpha}
    = \widetilde{\ker \alpha}.
  \]
  Furthermore, if $\mathcal{F}$ and
  $\mathcal{G}$ are coherent, then
  $M$ and $N$ are finitely generated,
  so $\ker \alpha$ is also
  finitely generated.
  One can show the same for
  $\im \varphi$ and
  $\coker \varphi$ similarly.
\end{proof}

\begin{remark}
  We have full subcategories
  $\mathrm{Coh}_X \subseteq \mathrm{QCoh}_X \subseteq \mathrm{Mod}_{\OO_X}$,
  which are abelian.
\end{remark}
