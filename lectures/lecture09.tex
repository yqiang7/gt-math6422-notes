\chapter{Feb.~10 --- Coherent Sheaves, Part 3}

\section{Quasicoherent Sheaves, Continued}

\begin{prop}\label{prop:pushforward-pullback}
  Let $f : X \to Y$ be a morphism of
  varieties. Then
  \begin{enumerate}
    \item If $\mathcal{G} \in \mathrm{QCoh}_Y$
      (resp. $\mathrm{Coh}_Y$), then $f^* \mathcal{G} \in \mathrm{QCoh}_X$
      (resp. $\mathrm{Coh}_X$).
    \item If $\mathcal{F} \in \mathrm{QCoh}_X$, then $f_* \mathcal{F} \in \mathrm{QCoh}_Y$.
  \end{enumerate}
\end{prop}

\begin{proof}
  (1) Fix $\mathcal{G} \in \mathrm{QCoh}_Y$
  and $x \in X$. There exist affine opens
  $x \in U \subseteq X$ and $f(x) \in V \subseteq Y$
  such that $f(U) \subseteq V$.
  Write $g = f|_U : U \to V$,
  then $(f^* \mathcal{G})|_U = g^* (\mathcal{G}|_V)$.
  Now letting $A = \OO_Y(V)$, $B = \OO_X(U)$,
  $M = \mathcal{G}(V)$,
  we see that
  \[
    f^* \mathcal{G}|_U
    = g^*(\mathcal{G}|_V)
    = g^* \widetilde{M}
    = \widetilde{M \otimes_A B}.
  \]
  So $f^* \mathcal{G}$ is quasicoherent.
  The coherent version follows from
  the fact that if $M$ is a finitely
  generated $A$-module, then
  $M \otimes_A B$ is a finitely generated
  $B$-module.

  (2) Fix $\mathcal{F} \in \mathrm{QCoh}_X$.
  We can check quasicoherence locally,
  so we can reduce to the case where
  $Y$ is affine (cover $Y$ by affine opens
  $U_i$ and replace $f$ with $f^{-1}(U_i) \to U_i$).
  Choose an affine cover $X = U_1 \cup \dots \cup U_r$,
  and note that $U_i \cap U_j$ is again
  affine for a variety.
  Write $\alpha_i : U_i \hookrightarrow X$
  and $\alpha_{i, j} : U_{i, j} \hookrightarrow X$.
  As $\mathcal{F}$ is a sheaf,
  we have an exact sequence
  \[
    \begin{tikzcd}
      0 \ar[r] & \mathcal{F}
      \ar[r] & \bigoplus_i (\alpha_i)_* \mathcal{F}|_{U_i}
      \ar[r] & \bigoplus_{i, j} (\alpha_{i, j})_* \mathcal{F}|_{U_{i, j}}.
    \end{tikzcd}
  \]
  Applying $f_*$, which is left exact,
  we get an exact sequence
  \[
    \begin{tikzcd}
      0 \ar[r] & f_* \mathcal{F}
      \ar[r] & \bigoplus_i(f \circ \alpha_i)_* \mathcal{F}|_{U_i}
      \ar[r] & \bigoplus_{i, j} (f \circ \alpha_{i, j})_* \mathcal{F}|_{U_{i, j}}.
    \end{tikzcd}
  \]
  Note that the last two terms
  are both quasicoherent (e.g.
  $\mathcal{F}|_{U_i}$ is
  quasicoherent and $f \circ \alpha_i$
  is a morphism of affine varieties, so
  $(f \circ \alpha_i)_* \mathcal{F}|_{U_i}$
  is quasicoherent).\footnote{If $f : X \to Y$ is a morphism of affine varieties, and $A := \OO_Y(Y)$, $B := \OO_X(X)$, then $f_* \widetilde{N} = \widetilde{N_A}$ for
  $N \in \mathrm{Mod}_B$ and $f^* \widetilde{M}$ for $M \in \mathrm{Mod}_A$. This shows that pushforwards and pullbacks of quasicoherent sheaves on \emph{affine} varieties are again quasicoherent.}
\end{proof}

\begin{remark}
  The coherent version of Proposition
  \ref{prop:pushforward-pullback}(2) fails:
  For
  \[i : \Affine^1 \setminus \{0\} \longhookrightarrow \Affine^1,\]
  the
  pushforward
  $i_* \OO_{\Affine^1 \setminus \{0\}}$
  is not coherent (but $\OO_{\Affine^1 \setminus \{0\}}$ is coherent).
\end{remark}

\begin{remark}
  If $X \to Y$ is \emph{projective},
  i.e. there exists a factorization
  \[
    \begin{tikzcd}
      X \ar[r, hook] \ar[dr] & Y \times \PP^n
      \ar[d, "\mathrm{pr}_1"] \\
                      & Y
    \end{tikzcd}
  \]
  then we do get a coherent version for
  the pushfoward $f_* \mathcal{F}$.
  One can prove this via
  sheaf cohomology.
\end{remark}

\section{Morphisms to Projective Space}

\begin{remark}
  Recall that there is a bijection
  \[
    \{\LL \to X \text{ with } s_0, \dots, s_n \in \Gamma(X, \LL) \text{ nowhere vanishing}\} / {\cong}
    \longleftrightarrow
    \{\text{morphisms } X \to \PP^n\}.
  \]
  We want to rephrase this using the
  bijection
  \[
    \{\mathcal{L} \text{ invertible sheaves of } \OO_X\text{-modules}\}
    \longleftrightarrow
    \{
      \text{line bundles } \LL \to X
    \}
  \]
  and determine when $X \to \PP^n$ is
  injective or a closed embedding.
\end{remark}

\begin{remark}
  Let $X$ be a variety and $\mathcal{L}$
  an invertible $\OO_X$-module.
  For $x \in X$, we have
  \[
    \mathcal{L}(x)
    := \mathcal{L}_x / \m_x \mathcal{L}_x
    \cong \OO_{X, x} / \m_x
    \cong k,
  \]
  where $\m_x \subseteq \OO_{X, x}$ is a
  maximal ideal.
\end{remark}

\begin{definition}
  We say $s_0, \dots, s_n \in \Gamma(X, \mathcal{L})$
  \emph{generate} $\mathcal{L}$ if
  \[
    s_0(x), \dots, s_n(x)
    \in \mathcal{L}(x) \cong k
  \]
  generate $\mathcal{L}(x)$
  as a $k$-vector space
  (i.e. at least one of $s_0(x), \dots, s_n(x)$ is nonzero)
  for all $x \in X$.
\end{definition}

\begin{prop}
  The following are equivalent:
  \begin{enumerate}
    \item $s_0, \dots, s_n \in \Gamma(X, \mathcal{L})$
      generate $\mathcal{L}$;
    \item the morphism
      $\varphi : \OO_X^{\oplus (n + 1)} \to \mathcal{L}$
      given by $e_i \mapsto s_i$ is
      surjective;
    \item for every $U \subseteq X$ open
      affine,
      $s_0|_U, \dots, s_n|_U$ generate
      $\mathcal{L}(U)$ as an
      $\OO_X(U)$-module.
  \end{enumerate}
\end{prop}

\begin{proof}
  By commutative algebra (in particular
  Nakayama's lemma),
  $s_0(x), \dots, s_n(x)$ span
  $\mathcal{L}(x)$ (1) if and only if
  $(s_0)_x, \dots, (s_n)_x$ generate $\mathcal{L}_x$ as an $\OO_{X, x}$-module.
  This happens if and only if
  $\varphi_x$ is surjective, which happens
  if and only if $\varphi$ is surjective (2).
  This happens if and only if
  $\varphi(U)$ is surjective for all
  $U \subseteq X$ open affine (3)
  (note that $\coker(\widetilde{M} \to \widetilde{N} = (\coker(M \to N))^{\sim}$).
\end{proof}
