\chapter{Feb.~12 --- Morphisms to Projective Space}

\section{Morphisms to Projective Space, Continued}

\begin{definition}
  When any of the conditions
  in Proposition \ref{prop:generate} hold, we say that
  $\mathcal{L}$ is \emph{generated}
  by $s_0, \dots, s_n$. Furthermore,
  we say $\mathcal{L}$
  is \emph{globally generated} if
  there exist $s_0, \dots, s_n$ generating
  $\mathcal{L}$.
\end{definition}

\begin{example}
  $\OO_X$ is globally generated,
  e.g. by $1 \in \Gamma(X, \OO_X)$.
\end{example}

\begin{example}
  For $X = \PP^n$, the invertible sheaf
  $\OO_{\PP^n}(1)$
  on $\PP^n$ defined by transition data
  $(U_i, x_j / x_i)$
  is globally generated. There are
  isomorphisms
  $\alpha_{i} : \OO_{\PP^n}(1)|_{U_i} \to \OO_{U_i}$
  such that $\alpha_i \circ \alpha_j^{-1} =$
  multiplication by $x_j / x_i$.
  Note that we have an isomorphism
  \begin{align*}
    k[x_0, \dots, x_n]_1
    &\longrightarrow \Gamma(\PP^n, \OO_{\PP^n}(1)) \\
    f &\longmapsto s_f,
  \end{align*}
  where $\alpha_i(s_f|_{U_i}) = f / x_i$
  (check that this is well-defined).

  Then $x_0, \dots, x_n \in \Gamma(\PP^n, \OO_{\PP^n}(1))$
  generate $\OO_{\PP^n}(1)$.
  To see this, note that on $U_i$, we have
  \[
    \alpha_i(x_i) = 1 \in \OO_{U_i},
  \]
  so $x_i|_{U_i}$ generates
  $\OO_{\PP^n}(1)(U_i)$ as
  an $\OO_{\PP^n}(U_i)$-module
  by Proposition \ref{prop:generate}(4).
  So the $x_0, \dots, x_n$
  generate $\OO_{\PP^n}(1)$. Alternatively,
  one can note that for
  $a = [a_0 : \cdots : a_n] \in \PP^n$,
  we have $[x_0(a) : \cdots : x_n(a)] = [a_0 : \cdots : a_n]$,
  so the $x_i$ generate $\OO_{\PP^n}(1)$.
\end{example}

\begin{example}
  Let $f : X \to \PP^n$ be a morphism.
  Then $\mathcal{L} = f^* \OO_{\PP^n}(1)$
  is generated by
  \[f^* x_0, \dots, f^* x_n \in \Gamma(X, \mathcal{L}).\]
\end{example}

\begin{remark}
  If $\mathcal{L}$ is generated by
  $s_0, \dots, s_n$, then we get a map
  \begin{align*}
    X &\longrightarrow \PP^n \\
    x &\longmapsto [s_0(x) : \cdots : s_n(x)].
  \end{align*}
  This is a morphism: If we fix $a \in X$ 
  and pick $a \in U \subseteq X$ open such
  that
  \[
    \alpha : \mathcal{L}|_U \overset{\cong}{\longrightarrow} \OO_U,
  \]
  then $f|_U(x) = [\alpha(s_0|_U)(x) : \cdots : \alpha(s_n|_U)(x)]$.
  Since each
  $\alpha(s_i)$ is a regular function,
  $f|_U$ is a morphism.
  Thus we see that $f$ is a morphism.
\end{remark}

\begin{remark}
  Similar to before, we have a bijection
  \[
    \{\text{morphisms } X \to \PP^n\}
    \longleftrightarrow
    \{\text{invertible sheaves $\mathcal{L}$ on $X$ with } s_0, \dots, s_n \in \Gamma(X, \mathcal{L} \text{ generators}\} / {\cong}
  \]
  The maps are given by
  $f \mapsto \mathcal{L} = f^* \OO_{\PP^n}(1)$
  with $s_i = f^* x_i$ with
  inverse $(\mathcal{L}, s_i) \mapsto [x \mapsto [s(x)]]$.
\end{remark}

\begin{example}
  Recall the \emph{Veronese embedding}
  given by
  \begin{align*}
    \PP^n &\longrightarrow \PP^{Nd} \\
    a &\longmapsto [x^I(a) : x^I \text{ monomial of degree $d$ in $x_0, \dots, x_n$}].
  \end{align*}
  This map corresponds to
  $\OO_{\PP^n}(d)$ with sections
  $x^I \in \Gamma(\PP^n, \OO_{\PP^n}(d))$.
\end{example}

\section{Injectivity and Closed Embeddings}

\begin{remark}
  When is $f : X \to \PP^n$ injective or a
  closed embedding?
\end{remark}

\begin{definition}
  Let $V = \Span\{s_0, \dots, s_n\} \subseteq \Gamma(X, \mathcal{L})$
  such that $s_0, \dots, s_n$ generate
  $\mathcal{L}$. We say that
  \begin{enumerate}
    \item $s_0, \dots, s_n$
      \emph{separate points} of $X$ if
      for all $x \ne y$, there exists
      $s \in V$ such that $s(x) = 0 \ne s(y)$.\footnote{Note that $s(x) = 0$ if and only if $s \in \m_p \mathcal{L}_p$.}
    \item $s_0, \dots, s_n$
      \emph{separate tangent vectors}
      if for each $p \in X$,
      \[
        \{s_p : s \in V \text{ and } s(p) = 0\}
      \]
      generates $\m_p \mathcal{L}_p / \m_p^2 \mathcal{L}_p$ as a $k$-vector space.\footnote{Here $\m_p \le \OO_{X, p}$ is its unique maximal ideal. Recall that $(T_p X)^* \cong \m_p / \m_p^2 \cong \m_p \mathcal{L}_p / \m_p^2 \mathcal{L}_p$.}
  \end{enumerate}
\end{definition}

\begin{prop}
  Assume that $X$ is complete and
  $\mathcal{L}$ an invertible sheaf on
  $X$ generated by $s_0, \dots, s_n \in \Gamma(X, Y)$.
  Write
  \begin{align*}
    f : X &\longrightarrow \PP^n \\
    x &\longmapsto [s_0(x) : \cdots : s_n(x)].
  \end{align*}
  Then we have the following:
  \begin{enumerate}
    \item $f$ is injective if and only if
      $s_0, \dots, s_n$ separate points;
    \item $f$ is a closed embedding
      if and only if
      $s_0, \dots, s_n$ separate points and
      tangent vectors.
  \end{enumerate}
\end{prop}

\begin{proof}
  (1) For $s = \sum_{i = 0}^n a_i s_i$
  with $a_i \in k$, we have
  \[
    \{x \in X : s(x) = 0\}
    = f^{-1}(H_a)
  \]
  where $H_a = V(\sum a_i x_i)$. Now
  $f$ is injective if and only if
  for all $x \ne y \in X$, there exists
  a hyperplane $H \subseteq \PP^n$ such that
  $f(x) \in H$ and $f(y) \not\in H$. This
  happens if and only if
  $s_0, \dots, s_n$ separate points.

  (2) Assume that $f$ is injective 
  (equivalently, $s_0, \dots, s_n$ separate points by (1)).
  For $p \in X$, we get a local ring
  homomorphism
  $\OO_{\PP^n, f(p)} \to \OO_{X, p}$,
  which induces a map on cotangent spaces
  \[
    \begin{tikzcd}
      \m_{f(p)} / \m_{f(p)}^2
      \arrow[r] \arrow[d, "\cong", swap] &
      \m_p / \m_p^2 \arrow[d, "\cong"] \\
      \m_{f(p)} \OO_{\PP^n}(1)_{f(p)} / \m^2_{f(p)} \OO_{\PP^n}(1)_{f(p)} &
      \m_p \mathcal{L}_p / \m_p^2 \mathcal{L}_p
    \end{tikzcd}
  \]
  (as $\mathcal{L}$ is invertible, it is
  locally trivial), where the bottom map
  is given by
  \[
    \left[\sum {a_i x_i}\right]
    \longmapsto \left[\sum a_i s_i\right].
  \]
  So we get that $f$ separates tangent
  vectors at $p$ if and only if
  $\m_{f(p)} / \m_{f(p)}^2 \to \m_p / \m_p^2$
  is surjective. Since
  $f \hookrightarrow \PP^n$ is injective
  and a closed map (as $X$ is complete),
  we get $X \hookrightarrow f(X) \subseteq \PP^n$ is a
  homeomorphism (we know it is a bijection,
  preimages of open sets are open, and
  images of closed sets are closed), and
  $f(X) \subseteq \PP^n$ is
  closed. Using this, $f$ is a
  closed embedding if and only if
  $\OO_{\PP^n} \to f_* \OO_X$ is
  surjective (HW) which happens if and only
  if $\OO_{\PP^n, f(p)} \to \OO_{X, p}$ is
  surjective for all $p \in X$. This
  happens if and only if
  \[
    \m_{f(p)} / \m_{f(p)}^2
    \longrightarrow \m_p / \m_p^2
  \]
  is surjective for all $p \in X$
  (see Mustata's notes, uses that
  $f_* \OO_X$ is coherent).
\end{proof}

\begin{definition}
  Let $\mathcal{L}$ be an invertible sheaf
  on a complete variety $X$.
  \begin{enumerate}
    \item $\mathcal{L}$ is
      \emph{very ample} if there exists
      a closed embedding
      $f : X \hookrightarrow \PP^n$
      and $\mathcal{L} \cong \OO_X(1) := f^* \OO_{\PP^n}(1)$.
    \item $\mathcal{L}$ is
      \emph{ample} if there exists $m > 0$ 
      such that $\mathcal{L}^{\otimes m}$
      is very ample.
  \end{enumerate}
\end{definition}

\begin{example}
  Let $X = \PP^n$ (for $n \ge 1$), then $\mathcal{L} = \OO_{\PP^n}(d)$ if and
  only if $d > 0$, if and only if
  $\OO_{\PP^n}(d)$ is ample (though this is
  not true in general).
\end{example}

\begin{exercise}
  If $\mathcal{F}$ is a coherent sheaf on
  a complete variety $X$ and $\OO_X(1)$ is
  then show that there exists an exact
  sequence:
  \[
    \begin{tikzcd}
    \bigoplus_{i = 1}^s \OO_X(n_i)
    \ar[r] & \bigoplus_{i = 1}^r \OO_X(m_i)
    \ar[r] & \mathcal{F} \ar[r] & 0
    \end{tikzcd}
  \]
\end{exercise}

\section{Divisors}

\begin{remark}
  For the rest of the lecture,
  let $X$ be an irreducible variety.
\end{remark}

\begin{definition}
  A \emph{prime divisor} on $X$ is a
  closed irreducible subvariety
  $D \subseteq X$ of codimension $1$.
\end{definition}

\begin{example}
  We have the following:
  \begin{enumerate}
    \item Prime divisors on $\PP^n$
      correspond to irreducible hypersurfaces.
    \item Prime divisors
      on a curve $C$ correspond to points.
  \end{enumerate}
\end{example}

\begin{definition}
  A \emph{(Weil) divisor} on $X$ is a
  formal sum
  \[
    D = \sum_{i = 1}^r a_i D_i,
  \]
  such $a_i \in \Z$ and the $D_i$ are
  distinct prime divisors. Write
  \[
    \Div X
    = \{\text{divisors on $X$}\}
    = \bigoplus_{\substack{E \subseteq X \\ \text{prime divisor}}} \Z[E].
  \]
\end{definition}

\begin{remark}
  We will see that when $X$ is smooth,
  there is a surjective group homomorphism
  \begin{align*}
    \Phi : \Div X &\longrightarrow \Pic X
    = \{\text{invertible sheaves on } X\} / {\cong} \\
    D &\longmapsto \OO_X(D).
  \end{align*}
  Furthermore, $\ker \Phi = \{\divv \varphi : \varphi \in K(X)^\times\}$,
  where $\divv \varphi$ is the divisor
  of zeros and poles of $\varphi$.
\end{remark}

\pagebreak

\begin{remark}[Local rings]
  For a closed irreducible subset
  $Z \subseteq X$, we can define a local
  ring
  \begin{align*}
    \OO_{X, Z}
    &= \{
      (\varphi, U) : \varphi \in \OO_X(U),\,
      U \subseteq X \text{ open},\,
      Z \cap U \ne \varnothing
    \} / {\sim} \\
    &= \varinjlim_{\substack{U \subseteq X \text{ open} \\ Z \cap U \ne \varnothing}} \OO_X(U),
  \end{align*}
  where $(\varphi, U) \sim (\psi, V)$
  if there exists $W \subseteq U \cap V$
  open with $Z \cap W \ne \varnothing$
  such that $\varphi|_W = \psi|_W$.
\end{remark}
