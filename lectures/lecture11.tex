\chapter{Feb.~17 --- Divisors}

\section{Divisors, Continued}
\begin{prop}
   We have the following:
   \begin{enumerate}
     \item If $U \subseteq X$ is affine open,
       then
       we have an isomorphism
       \[
         \OO_X(U)_{\mathcal{I}_Z(U)}
         \overset{\cong}{\longrightarrow}
         \OO_{X, Z}
       \]
       when $Z \cap U \ne \varnothing$.
     \item $\dim \OO_{X, Z} = \codim_X Z$.
     \item $\OO_{X, Z}$ is a local ring.
     \item Let $\m_Z$ be the maximal
       ideal of $\OO_{X, Z}$.
       If $X$ is smooth, then
       $\OO_{X, Z}$ is regular, i.e.
       \[
         \dim \OO_{X, Z}
         = \dim \m_Z / \m_Z^2,
       \]
       where the right-hand side is the
       dimension as a vector space over
       $\OO_{X, Z} / \m_Z$.
   \end{enumerate}
\end{prop}

\begin{proof}
  (1) The proof is the same as when
  $Z = \{x\}$.

  (2) We can reduce to the case where
  $X$ is affine. Then we have a bijection
  \[
    \begin{tikzcd}
      \{\text{prime ideals $\p \le A$ such that $\p \cap (A \setminus I_Z) = \varnothing$}\} \ar[d]
    \ar[r, "\q \mapsto \q \OO_{X, Z}"]
    &
    \{\text{prime ideals of $\OO_{X, Z}$}\} \ar[l] \\
    \{\text{closed irreducible subvarieties $Z \subseteq Y \subseteq X$}\} \ar[u]
    \end{tikzcd}
  \]
  where $A = \OO_X(X)$.
  This proves the claim.

  (3) We use (1). Since
  $\mathcal{I}_Z(U) \subseteq \OO_X(U)$
  is prime, $\OO_{X, Z}$ is a local
  ring with maximal ideal
  \[
    \m_Z := \mathcal{I}_Z(U) \OO_{X, Z}
    = \{(\varphi, U) \in \OO_{X, Z} : \varphi|_{Z \cap U} = 0\} \subseteq \OO_{X, Z}.
  \]

  (4) In Algebraic Geometry I, we have
  seen that this holds when $Z$ is a point.
  Choose $x \in Z$, then
  \[
    \OO_{X, Z} \cong (\OO_{X, x})_{\{\varphi \in \OO_{X, x}\, :\, \varphi|_Z = 0\}}.
  \]
  As $\OO_{X, x}$ is regular, so is
  $\OO_{X, Z}$ by commutative algebra.
\end{proof}

\begin{remark}
  For the rest of this
  lecture, assume that $X$ is a smooth
  irreducible variety (so that the local
  rings are regular and we can talk about
  the function field of $X$).
\end{remark}

\begin{remark}
  Let $E \subseteq X$ be a prime divisor.
  Then $\OO_{X, E}$ is a regular local ring
  of dimension $1$. By results from
  commutative algebra,
  $\OO_{X, E}$ is a
  \emph{discrete valuation ring} (DVR).
  In particular, this implies that
  $\OO_{X, E}$ is an integral domain
  which is a UFD with a unique irreducible
  element up to multiplication by units.

  So there exist an irreducible
  element $\pi \in \OO_{X, E}$ such that
  for any $0 \ne f \in \Frac(\OO_{X, E}) = K(X)$,
  \[
    f = u \pi^d
  \]
  for some $u \in \OO_{X, E}^\times$ and
  $d \in \Z$ (with $d \ge 0$ if and only if
  $f \in \OO_{X, E}$).
  Such a $\pi$ is called a \emph{uniformizer}.

  We will write $\ord_E(f) = d$ when
  $f = u \pi^d$, which we view as the
  multiplicity of vanishing of $f$ along $E$.
\end{remark}

\begin{remark}
  Note the following:
  \begin{itemize}
    \item $\ord_E(fg) = \ord_E(f) + \ord_E(g)$.
    \item $\ord_E(f + g) \ge \min\{\ord_E(f), \ord_E(g)\}$.
  \end{itemize}
  Along with some other properties,
  the above implies that $\ord_E$ is a
  \emph{valuation} on $K(X)$.
\end{remark}

\begin{example}
  Consider $0 \in \Affine^1$. Then we have
  \[
    \OO_{\Affine^1, 0}
    \cong \left\{\frac{f}{g} : f, g \in k[x],\, g(0) \ne 0\right\}.
  \]
  For $\varphi \in \OO_{\Affine^1, 0}$,
  we can write
  \[
    \varphi = u x^m
  \]
  such that $u \in \OO_{\Affine^1, 0}^\times = \{f / g : f, g \in k[x],\, f(0), g(0)  \ne 0\}$.
  So $\ord_0 \varphi = m$.
\end{example}

\begin{remark}
  We can also think of
  $\OO_{X, E}
    = \{\varphi \in K(X) : \ord_E(\varphi) \ge 0 \}$.
\end{remark}

\begin{definition}
  For $0 \ne \varphi \in K(X)$, its
  \emph{divisor of zeros and poles} is
  \[
    \divv(\varphi) = \divv_X(\varphi)
    := \sum_{\substack{E \subseteq X \\ \text{prime}}} \ord_E(\varphi) E.
  \]
\end{definition}

\begin{prop}
  The divisor of zeros and poles
  $\divv(\varphi)$ is a divisor, i.e. we have
  $\ord_E(\varphi) = 0$ for all but finitely
  many prime divisors $E \subseteq X$.
\end{prop}

\begin{proof}
  For an open affine $\varnothing \ne U \subseteq X$.
  Since $X \setminus U$ contains at most
  finitely many prime divisors, it suffices
  to show that $\divv_U(\varphi)$
  is a divisor. Write
  $\varphi = f / g$ with
  $f, g \in \OO_X(U) = A$. Since
  \[
    \divv_U(\varphi) = \divv_U(f) - \divv_U(g),
  \]
  it suffices to consider the case when
  $\varphi \in \OO_X(U)$. Now for
  $E \subseteq X$ prime with
  $E \cap U \ne \varnothing$, we have
  \begin{align*}
    \ord_E(\varphi) > 0
    &\iff \varphi \in \m_E \subseteq \OO_{X, E} \\
    &\iff \varphi \text{ vanishes on } E \cap U \\
    &\iff E \subseteq V(\varphi).
  \end{align*}
  As $V(\varphi)$ contains finitely many
  prime divisors,
  we get that $\divv_U(\varphi)$
  is a divisor.
\end{proof}

\begin{example}
  Let $f \in k[x_1, \dots, x_n] \in \OO_{\Affine^n}(\Affine^n)$.
  We can write
  \[
    f = cf_1^{a_1} \cdots f_r^{a_r}
  \]
  with $f_i$ irreducible and  $a_i \ge 0$.
  Then we have
  \[
    \divv_{\Affine^n}(f) = \sum_{i = 1}^r a_i V(f_i).
  \]
  This follows since $\ord_{V(f_i)} f_i = 1$,
  which in turns follows from
  $\OO_{\Affine^n, V(f_i)} \cong k[x_1, \dots, x_n]_{(f_i)}$,
  which has maximal ideal
  $(f_i)$ and hence uniformizer $f_i$.
\end{example}

\begin{prop}[Algebraic Hartog's lemma]
  For $0 \ne \varphi \in K(X)$, we have
  $\divv \varphi \ge 0$ (i.e.
  $\ord_E \varphi \ge 0$ for all prime
  divisors $E \subseteq X$)
  if and only if $\varphi \in \OO_X(X)$.
\end{prop}

\begin{proof}
  It suffices to show the statement on
  an affine cover. So we may assume that
  $X$ is affine. Write $A = \OO_X(X)$.
  Hartog's lemma in commutative algebra
  implies that
  \[
    A = \bigcap_{\substack{\p \le A \text{ prime} \\ \text{height $1$}}} A_\p
    \subseteq \Frac(A).
  \]
  Thus for $\varphi \in \Frac(A) \cong K(X)$,
  we have
  $\varphi \in A$ if and only if
  $\varphi \in A_\p$ for all height $1$
  prime ideals $\p \le A$, which happens
  if and only if $\ord_E(\varphi) \ge 0$
  for all prime divisors $E \subseteq X$, i.e.
  $\divv_X(\varphi) \ge 0$.
\end{proof}

\section{Class Groups}
\begin{definition}
  A divisor $D \in \Div X$ is \emph{principal}
  if $D = \divv_X(\varphi)$ for some
  $0 \ne \varphi \in K(X)$.
\end{definition}

\begin{example}
  Any divisor on $\Affine^n$ is principal
  by the previous examples (this is
  also true when $\Affine^n$ is replaced
  by an affine variety $X$ such that
  $\OO_X(X)$ is a UFD).
\end{example}

\begin{remark}
  Note that $\PDiv X = \{\divv \varphi : 0 \ne \varphi \in K(X)\} \subseteq \Div X$
  is a subgroup.
\end{remark}

\begin{definition}
  We say that $D_1, D_2 \in \Div X$ are
  \emph{linearly equivalent} if
  $D_1 - D_2$ is principal. The
  \emph{class group} of $X$ is
  \[
    \Cl(X) = \Div(X) / \PDiv(X)
    = \text{group of divisors up to $\sim$}.
  \]
\end{definition}

\begin{example}
  We have the following:
  \begin{enumerate}
    \item $\Cl(\Affine^n) = 0$.
    \item $\Cl(\PP^n) = \Z[H]$, where
      $H \subseteq \PP^n$ is a hyperplane.

      To see this, recall that we have
      a bijection
      \[
        \{\text{prime divisors in $\PP^n$}\}
        \longleftrightarrow
        \{\text{irreducible hypersurfaces}\}.
      \]
      Consider $\deg : \Div(\PP^n) \to \Z$
      which sends
      $\sum a_i E_i = \sum a_i \deg E_i$,
      which is clearly surjective. For
      $\varphi \in K(\PP^n)$, write
      $\varphi = f / g$ where
      $f, g \in k[x_0, \dots, x_n]$
      are homogeneous of the same degree.
      So
      \[
        f = f_1^{a_1} \cdots f_r^{a_r}
        \quad\text{and}\quad
        g = g_1^{b_1} \cdots g_s^{b_s}
      \]
      with $f_i, g_i$ homogeneous and
      $a_i, b_i \in \Z_{\ge 0}$, and
      $\sum a_i \deg f_i = \sum b_i \deg g_i$.
      Now
      \[
        \divv \varphi
        = \sum_{i = 1}^r a_i V(f_i)
        - \sum_{i = 1}^s b_i V(g_i).
      \]
      So $\deg \varphi = 0$.
      This shows that
      $\PDiv(\PP^n) \subseteq \ker \deg$.
      One can show the reverse inclusion as well,
      so $\PDiv(\PP^n) = \ker \deg$.
      Thus we have
      $\Cl(\PP^n) = \Div(\PP^n) / \PDiv(\PP^n) \cong \Z$.
  \end{enumerate}
\end{example}

\section{Cartier Divisors}

\begin{definition}
  A divisor $D$ on $X$ is \emph{Cartier}
  if it is locally principal, i.e. at
  each $x \in X$, there exists an open
  neighborhood $x \in U_x \subseteq X$
  and $0 \ne \varphi_x \in K(X)$ such that
  $\divv_{U_x}(\varphi_x) = D|_{U_x}$.
\end{definition}

\begin{example}
  Principal divisors are Cartier
  (in fact, they are globally principal).
\end{example}

\begin{example}
  Let $H = V(x_0) \subseteq \PP^n$.
  Note $H \nsim 0$, so
  $H$ is not principal. But on
  $U_i = \{x_i \ne 0\} \subseteq \PP^n$,
  \[
    H|_{U_i} = \divv_{U_i}(x_0 / x_i),
  \]
  so $H$ is Cartier (alternatively,
  $\PP^n$ has an open cover by the $U_i$,
  and each $\Cl(U_i) = 0$).
\end{example}

\begin{prop}\label{prop:cartier}
  Every divisor $D$ on $X$ is Cartier.
\end{prop}

\begin{proof}
  If $D_1, D_2$ are Cartier, then so is
  $D_1 + D_2$. So it suffices to consider
  the case when $D = E$ for some prime
  divisor $E$ on $X$. Now fix $x \in X$.
  If $x \notin E$, then set
  $U_x = X \setminus E$ and $\varphi_x = 1$.
  Now assume $x \in E$. Since
  $\OO_{X, x}$ is a regular local ring,
  the Auslander-Buchsbaum theorem
  implies that $\OO_{X, x}$ is a UFD.
  Now we have a bijection
  \[
    \{\text{prime ideals of $\OO_{X, x}$}\}
    \longleftrightarrow
    \{\text{irreducible closed subvarieties } x \in Y \subseteq X\}.
  \]
  Write $J \subseteq \OO_{X, x}$
  for the prime ideal corresponding to $E$.
  As $J$ is a height $1$ prime ideal and
  $\OO_{X, x}$ is a UFD, there exists
  $f \in \OO_{X, x}$ such that $J = (f)$.
  So there exists an isomorphism
  \[
    \varphi : (\OO_{X, x})_{(f)}
    \overset{\cong}{\longrightarrow}
    \OO_{X, E}
  \]
  such that $\varphi(f) \OO_{X, E} = \m_E$.
  So $\ord_E(f) = 1$. One then checks that
  $\divv_{U_x}(f) = E|_{U_x}$ for some
  $U_x$.
\end{proof}

\begin{remark}
  Proposition \ref{prop:cartier}
  crucially uses our assumption
  that $X$ is smooth. It may fail in
  general.
\end{remark}
