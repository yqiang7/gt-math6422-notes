\chapter{Feb.~19 --- Divisors, Part 2}

\section{Cartier Divisors, Continued}

\begin{example}
  Consider $X = V(xy - z^2)$.
  Note that this is singular. Also, all
  of the local rings for this particular
  variety are still DVRs, so we can
  still talk about divisors. The prime
  divisors are
  \[
    D_1 = V(x, z)
    \quad\text{and}\quad
    D_2 = V(y, z).
  \]
  One can compute that
  $\divv_X(x) = 2D_1$,
  $\divv_X(y) = 2D_2$, and
  $\divv_X(z) = D_1 + D_2$.
  So $D_1$ is not Cartier.
\end{example}

\begin{example}
  Consider $X = V(xy - zw)$, which is
  also singular. Then
  $D = V(x, z)$ is a prime divisor, and
  one can check that $mD$ is not
  Cartier for all $m \ne \Z \setminus \{0\}$.
\end{example}

\section{Divisors and Invertible Sheaves}

\begin{definition}
  For $D \in \Div X$, define
  $\OO_X(D) \in \mathrm{Mod}_{\OO_X}$
  such that for $U \subseteq X$ open,
  \[
    \OO_X(D)(U)
    = \{
      \varphi \in K(X) :
      \divv_U(\varphi) + D|_U \ge 0
    \}
  \]
\end{definition}

\begin{remark}
  If $D \ge 0$, then one can think of this
  as saying that ``the poles of $\varphi$
  are $\le D$.''
\end{remark}

\begin{remark}
  Note that $\OO_X(D) \subseteq
  \underline{K(X)}$ is a subsheaf.
\end{remark}

\begin{example}
  Let $H = V(x_1) \subseteq \Affine^n$.
  Then
  \begin{align*}
    \OO_{\Affine^n}(H)(\Affine^n)
    &= \{
      \varphi \in k[x_1, \dots, x_n]
      : \ord_{H}(\varphi) \ge -1
      \text{ and }
      \ord_E(\varphi) \ge 0
      \text{ for all prime $H \ne E
      \subseteq X$}
    \} \\
    &= \frac{1}{x_1} k[x_1, \dots, x_n]
    \subseteq k(x_1, \dots, x_n).
  \end{align*}
\end{example}

\begin{prop}
  If $D$ is principal and
  $D = \divv_X(g)$, then
  \[
    \OO_X(D) = \frac{1}{g} \OO_X
    \subseteq \underline{K(X)}.
  \]
\end{prop}

\begin{proof}
  Fix $U \subseteq X$ open. For
  $\varphi \in K(X)$, we have
  $\varphi \in \OO_X(D)(U)$ if and only if
  \[
    \divv_U(\varphi)
    + D|_U
    = \divv_U(\varphi) + \divv_U(g)
    \ge 0,
  \]
  if and only if $\divv_U(\varphi g) \ge 0$,
  if and only if $\varphi g \in \OO_X(U)$,
  if and only if $\varphi = \psi / g$ for
  some $\varphi \in \OO_X(U)$.
  So we have
  $\OO_X(D)(U) = (1 / g) \OO_X(U)$.
\end{proof}

\begin{corollary}
  For any $D \in \Div X$, $\OO_X(D)$ is an
  invertible $\OO_X$-module.
\end{corollary}

\begin{proof}
  Since $D$ is necessarily Cartier (recall
  the smoothness assumption),
  there exists an open cover
  $X = U_1 \cup \cdots \cup U_r$
  and $g_i \in K(X)^\times$ such that
  $D|_{U_i} = \divv_{U_i}(g_i)$. Now
  \[
    \OO_X(D)|_{U_i}
    = \OO_{U_i}(D|_{U_i})
    = \frac{1}{g_i} \OO_{U_i}.
  \]
  Now we have an isomorphism
  \begin{align*}
    \alpha : \OO_X(D)|_{U_i}
    = \frac{1}{g_i} \OO_{U_i}
    &\longrightarrow \OO_{U_i} \\
    f &\longmapsto f g_i.
  \end{align*}
  So $\OO_X(D)$ is an invertible
  $\OO_X$-module with trivialization
  data $(U_i, g_{i, j} = g_i / g_j)$.
\end{proof}

\begin{example}
  If $H = V(x_0) \subseteq \PP^n$, then we
  have
  \[
    H|_{U_i} = \divv_{U_i}(x_0 / x_i),
  \]
  so $\OO_{\PP^n}(H)$ is the invertible
  sheaf with transition data
  $(U_i, x_j / x_i)$.
  So $\OO_{\PP^n}(H) \cong \OO_{\PP^n}(1)$.
\end{example}

\begin{prop}\label{prop:divisor-invertible-sheaf}
  The map
  \begin{align*}
    \Div X &\longrightarrow \Pic X \\
    D &\longmapsto \OO_X(D)
  \end{align*}
  is a surjective homomorphism
  with kernel $\PDiv X$.
  In particular, $\Cl(X) \cong \Pic X$, and
  $D_1 \sim D_2$ if and only if
  $\OO_X(D_1) \cong \OO_X(D_2)$.
\end{prop}

\begin{proof}
  We first check that this map is a
  group homomorphism. Fix
  $D_1, D_2 \in \Div X$. Choose an open cover
  $\{U_i\}$ of $X$ such that there exist
  $g_i^1, g_i^2 \in K(X)^\times$ with
  \[
    D_1|_{U_i} = \divv_{U_i}(g_i^1)
    \quad \text{and} \quad
    D_2|_{U_i} = \divv_{U_i}(g_i^2).
  \]
  So we have $(D_1 + D_2)|_{U_i}
  = \divv_{U_i}(g_i^1 g_i^2)$.
  Then we have:
  \begin{center}
    \begin{tabular}{c|c}
      invertible sheaf & transition data \\
      \hline
      $\OO_X(D_1)$ & $g_i^1 / g_j^1$ \\
      $\OO_X(D_2)$ & $g_i^2 / g_j^2$ \\
      $\OO_X(D_1 + D_2)$ & $g_i^1 g_i^2 / g_j^1 g_j^2$
    \end{tabular}
  \end{center}
  So we see that $\OO_X(D_1 + D_2) \cong
  \OO_X(D_1) \otimes \OO_X(D_2)$.

  Now we check the kernel.
  If $D \sim 0$, then $D = \divv_X g$
  for some $g \in K(X)^\times$. So
  \[
    \OO_X(D) = \frac{1}{g} \OO_X \cong \OO_X.
  \]
  Conversely, assume that there exists an
  isomorphism
  $\beta : \OO_X \to \OO_X(D)$.
  Set $f = \beta(1)$. Since
  $1 \OO_X = \OO_X$,
  \[
    f \OO_X = \OO_X(D).
  \]
  So $f \OO_{U_i} = (1 / g_i) \OO_{U_i}$,
  so $f g_i \in \OO_X(U_i)^\times$.
  Then
  \[\divv_{U_i} f =
  \divv_{U_i}(1 / g_i) = -D|_{U_i}.\]
  So $\divv_X f = -D$, and thus
  $D$ is principal.

  Finally, we check surjectivity.
  Fix an invertible sheaf
  $\mathcal{L} \in \Pic X$, say with
  trivialization data
  $(U_i, g_{i, j})_{i \in I}$. Fix an element
  $0 \in I$ and set $g_0 = 1$,
  $g_i = g_{i, 0}$ for $i \in I \setminus \{0\}$.
  Then
  \[
    g_i / g_j
    = g_{i, 0} / g_{j, 0}
    = g_{i, j} \in \OO_X(U_{i, j})^\times.
  \]
  Let $D^i = \divv_{U_i}(g_i) \in \Div(U_i)$.
  Note that
  \[
    D^i_{U_{i, j}}
    - D^j_{U_{i, j}}
    = \divv_{U_{i, j}}(g_i / g_j)
    = \divv_{U_{i, j}}(g_{i, j}) = 0,
  \]
  so the $D^i$ glue to a divisor
  $D \in \Div X$ such that
  $D|_{U_i} = \divv_{U_i}(g_i)$.
  So $\OO_X(D) \cong \mathcal{L}$.
\end{proof}

\begin{remark}
  Proposition \ref{prop:divisor-invertible-sheaf}
  implies the following:
  \begin{itemize}
    \item $\Pic(\Affine^n) \cong \Cl(\Affine^n) = 0$.
    \item $\Pic(\PP^n) \cong \Cl(\PP^n)
      \cong \Z$,
      and $\Pic(\PP^n) = \langle \OO_{\PP^n}(1) \rangle$.
  \end{itemize}
\end{remark}

\section{Effective Cartier Divisors}

\begin{remark}
  Fix $D = \sum a_i D_i \in \Div X$
  such that the $D_i$ are distinct
  prime divisors.
\end{remark}

\begin{definition}
  A divisor $D$ is called \emph{effective}
  if $a_i \ge 0$ for all $i$, and
  we write $D \ge 0$.
\end{definition}

\begin{remark}
  If $D \ge 0$, then we have
  \[
    \OO_X(-D)
    \subseteq \OO_X
    \subseteq \underline{K(X)},
  \]
  and $\OO_X(-D)$ is a coherent ideal
  sheaf. We write
  $\OO_D := \OO_X / \OO_X(-D)$.
\end{remark}

\begin{example}
  Let $H = V(x_1) \subseteq \Affine^n$.
  Then
  $\OO_{\Affine^n}(-H) = x_1 \OO_{\Affine^n}$,
  and
  \[
    \OO_H = (k[x_1, \dots, x_n] / (x_1))^\sim
    = i_* \OO_H,
  \]
  where $i : H \hookrightarrow \Affine^n$
  is the inclusion. In general, we have
  \begin{enumerate}
    \item $\Supp(\OO_D) = \bigcup_{i = 1}^r E_i$,
      where $D = \sum_{i = 1}^r b_i E_i$
      with $b_i > 0$ and
      $E_i$ distinct primes.

      To see this, we can write
      $(\OO_D)_x = \OO_{X, x} / \OO_X(-D)_x$,
      which is zero if and only if
      $x \notin \bigcup_{i = 1}^r E_i$.
      We will call this locus the
      \emph{support} of $D$.
    \item If $b_1 = \cdots = b_r = 1$, then
      \[
        \OO_X(-D)
        = \mathcal{I}_{E_1 \cup \cdots \cup E_r}.
      \]
      Thus $\OO_D = i_* \OO_{E_1 \cup \cdots \cup E_r}$, where,
      where $i : E_1 \cup \cdots \cup E_r \hookrightarrow X$ is the inclusion.
  \end{enumerate}
\end{example}

\section{Effective Divisors and Zero Loci of Sections}

\begin{remark}
  Fix an invertible sheaf
  $\mathcal{L}$ on $X$. Fix trivialization
  data
  \[
    (U_i, \alpha_i : \mathcal{L}|_{U_i}
    \overset{\cong}{\longrightarrow} \OO_{U_i}).
  \]
\end{remark}

\pagebreak

\begin{prop}
  For $0 \ne s \in \Gamma(X, \mathcal{L})$,
  there exists a divisor $D \in \Div X$
  such that
  \begin{enumerate}
    \item $D|_{U_i} = \divv_{U_i}(\alpha_i(s))$,
    \item $D \ge 0$,
    \item $\OO_X(D) \cong \mathcal{L}$.
  \end{enumerate}
  Write $V(s) := D$ for this
  divisor.
\end{prop}

\begin{proof}
  (1) It suffices to show
  $\divv_{U_{i, j}}(\alpha_i(s))
  = \divv_{U_{i, j}}(\alpha_j(s))$, and
  this holds as
  $\alpha_i(s) / \alpha_j(s) \in \OO_X(U_{i, j})^\times$.

  (2) Use that $\alpha_i(s) \in \OO_X(U_i)$.

  (3) Left as an exercise. One
  works with transition functions.
\end{proof}

\begin{remark}
  We get a bijection
  \begin{align*}
    \{
      \text{$1$-dimensional subspaces in
      $\Gamma(X, \mathcal{L})$}
    \}
    &\longleftrightarrow
    \{
      \text{effective Cartier divisors
      $D$ such that
    $\OO_X(D) \cong \mathcal{L}$}
    \} \\
    ks &\longmapsto V(s) \\
    \mathcal{L} \cong \OO_X(D) \ni 1
    &\mathrel{\reflectbox{\ensuremath{\longmapsto}}}
    D.
  \end{align*}
  If $\mathcal{L} \cong \OO_X(D_0)$,
  then there is another bijection with the
  effective Cartier divisors $D \sim D_0$.
\end{remark}

\begin{example}
  Let $X = \PP^n$ and
  $\mathcal{L} = \OO(m)$ for $m \ge 1$.
  Then $\mathcal{L} \cong \OO(mH)$
  for a hyperplane $H$, and
  \begin{align*}
    \{
      \text{effective Cartier divisors
      $D \sim mH$}
    \}
    &=
    \left\{
      \sum a_i H_i :
      \substack{\displaystyle a_i \ge 0,\, H_i
      \text{ irreducible hypersurfaces} \\
  \displaystyle \text{with }\sum a_i \deg H_i = m}
    \right\} \\
    &= \{\text{hypersurfaces of degree
    $m$ in $\PP^n$ counting multiplicities}\}.
  \end{align*}
\end{example}

\begin{remark}
  If $s_0, \dots, s_n \in
  \Gamma(X, \mathcal{L})$
  globally generate $\mathcal{L}$,
  \begin{align*}
    f : X &\longrightarrow \PP^n \\
    x &\longmapsto [s_0(x) : \cdots : s_n(x)],
  \end{align*}
  and $H_a = V(\sum a_i x_i) \subseteq \PP^n$ is a hyperplane,
  then $f^{-1}(H_a) = V(\sum a_i s_i)$.
\end{remark}
