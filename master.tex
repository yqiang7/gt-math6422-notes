\documentclass[12pt, letterpaper, oneside]{book}
\usepackage[margin={0.6in, 0.75in}]{geometry}
\usepackage{microtype}
% \usepackage{kpfonts}
\usepackage{amsmath, amssymb, amsthm}
\usepackage{bbm}
\usepackage{parskip}
\usepackage[many]{tcolorbox}
\usepackage{footnote}
\usepackage{cancel}
\usepackage{titlesec}
\usepackage{pgffor}
\usepackage[shortlabels, inline]{enumitem}
\usepackage{hyperref}
\usepackage{tikz-cd}
\usepackage[new]{old-arrows}

\usepackage[overload]{textcase}
\usepackage{graphicx}

\renewcommand{\chaptername}{Lecture}
\newtheorem{axiom}{Axiom}[chapter]
\newtheorem{theorem}{Theorem}[chapter]
\newtheorem{prop}{Proposition}[chapter]
\newtheorem{corollary}{Corollary}[theorem]
\newtheorem{lemma}{Lemma}[chapter]
\newtheorem{conjecture}{Conjecture}[theorem]
\theoremstyle{definition}
\newtheorem{definition}{Definition}[chapter]
\newtheorem{exercise}{Exercise}[chapter]
\newtheorem{example}{Example}[definition]
\newtheorem*{remark}{Remark}

\tcbset{sharp corners, breakable, enhanced, parbox=false}
\newtcolorbox{mybox}[3][]
{
  colframe = #2!150,
  colback  = #2!5,
  coltitle = #2!0!white,  
  title    = {#3},
  #1,
}

\titleformat{\chapter}[display]
    {\normalfont\huge\bfseries}{\chaptertitlename\ \thechapter}{20pt}{\Huge}
\titlespacing*{\chapter}{0pt}{0pt}{40pt}

\newcommand{\R}{\mathbb{R}}
\newcommand{\N}{\mathbb{N}}
\newcommand{\Z}{\mathbb{Z}}
\newcommand{\C}{\mathbb{C}}
\newcommand{\Q}{\mathbb{Q}}
\newcommand{\F}{\mathbb{F}}
\newcommand{\PP}{\mathbb{P}}
\newcommand{\EE}{\mathbb{E}}
\newcommand{\LL}{\mathbb{L}}
\newcommand{\Affine}{\mathbb{A}}
\newcommand{\OO}{\mathcal{O}}
\newcommand{\p}{\mathfrak{p}}
\newcommand{\q}{\mathfrak{q}}
\newcommand{\m}{\mathfrak{m}}
\newcommand{\Mod}[1]{\ {\mathrm{mod}\ #1}}
\newcommand{\Pmod}[1]{\ (\mathrm{mod}\ #1)}

\newcommand{\T}{\mathcal{T}}
\newcommand{\B}{\mathcal{B}}
\newcommand{\Homm}{\mathcal{H}om}

\DeclareMathOperator{\vol}{vol}
\DeclareMathOperator{\Int}{int}
\DeclareMathOperator{\area}{area}
\DeclareMathOperator{\id}{id}
\DeclareMathOperator{\ord}{ord}
\DeclareMathOperator{\Cl}{Cl}
\DeclareMathOperator{\lcm}{lcm}
\DeclareMathOperator{\Gal}{Gal}
\DeclareMathOperator{\im}{im}
\DeclareMathOperator{\coker}{coker}
\DeclareMathOperator{\GL}{GL}
\DeclareMathOperator{\Spec}{Spec}
\DeclareMathOperator{\Tr}{Tr}
\DeclareMathOperator{\codim}{codim}
\DeclareMathOperator{\height}{ht}
\DeclareMathOperator{\trdeg}{tr{.}deg}
\DeclareMathOperator{\Frac}{Frac}
\DeclareMathOperator{\Open}{Open}
\DeclareMathOperator{\Rings}{Rings}
\DeclareMathOperator{\Hom}{Hom}
\DeclareMathOperator{\Char}{char}
\DeclareMathOperator{\Aut}{Aut}
\DeclareMathOperator{\PGL}{PGL}
\DeclareMathOperator{\pr}{pr}
\DeclareMathOperator{\Span}{span}
\DeclareMathOperator{\rank}{rk}
\DeclareMathOperator{\Sing}{Sing}
\DeclareMathOperator{\Smooth}{Smooth}
\DeclareMathOperator{\Jac}{Jac}
\DeclareMathOperator{\mult}{mult}
\DeclareMathOperator{\Div}{Div}
\DeclareMathOperator{\PDiv}{PDiv}
\DeclareMathOperator{\divv}{div}
\DeclareMathOperator{\Pic}{Pic}
\DeclareMathOperator{\genus}{genus}
\DeclareMathOperator{\Sym}{Sym}

\title{MATH 6422: Algebraic Geometry II}
\author{Frank Qiang\\Instructor: Harold Blum}
\date{Georgia Institute of Technology\\Spring 2026}

\begin{document}
  \maketitle

  \begingroup
  \let\cleardoublepage\clearpage
  \tableofcontents
  \endgroup

  % \foreach \i in {00, 01, 02, 03, 04, ..., 50} {%
  %   \edef\FileName{lectures/lecture\i.tex}%     The % here are necessary to eliminate any
  %   \IfFileExists{\FileName}{%  spurious spaces that may get inserted
  %      \input{\FileName}%       at these points
  %   }
  % }
  \chapter{Jan.~13 --- Overview and Review}

\section{Course Overview}

\begin{remark}
  This course will cover the following
  topics:
  \begin{enumerate}[(i)]
    \item vector bundles and line
      bundles in algebraic geometry;
    \item coherent sheaves;
    \item differentials;
    \item sheaf cohomology: in
      particular, we will see that
      $H^k_{\mathrm{dR}}(X^{\mathrm{an}}, \C) = \bigoplus_{i + j = k} H^i(X, \wedge^j T_X^*)$);
    \item the Riemann-Roch theorem:
      if $\omega = f\, dz$ is a rational
      $1$-form on a smooth
      projective curve $C$, then
      \[
        (\# \text{ zeroes of } \omega)
        - (\# \text{ poles of } \omega)
        = 2 \genus(C) - 2;
      \]
    \item surfaces and toric varieties;
    \item schemes: for example,
      $\Spec \Z$ has points
      corresponding to the primes $p$
      and $0$.
  \end{enumerate}
\end{remark}

\section{Review of Algebraic Geometry I}

\begin{remark}
  Let $k = \overline{k}$ be an
  algebraically closed field.
\end{remark}

\begin{remark}[Hilbert's Nullstellensatz]
  There is a correspondence
  \begin{align*}
    \text{closed subvarieties of $\Affine^n$}
    &\longleftrightarrow
    \text{radical ideals in $k[x_1, \dots, x_n]$} \\
    Z &\longmapsto I(Z) \\
    V(J) &\mathrel{\reflectbox{\ensuremath{\longmapsto}}} J.
  \end{align*}
  Under this correspondence,
  $Z$ being
  irreducible (resp. a point)
  corresponds to $I(Z)$ being
  prime (resp. maximal).
\end{remark}

\begin{remark}[Zariski topology on $\Affine^n$]
  The closed sets in $\Affine^n$ are
  of the form $V(J)$, and this induces
  a Zarisiki topology on any subset
  of $\Affine^n$.
\end{remark}

\begin{remark}[Embedded affine varieties]
  Let $J \le k[x_1, \dots, x_n]$.
  Then we can associate to $J$ a
  ringed space $(X, \OO_X)$
  by setting $X := V(J) \subseteq \Affine^n$
  with the Zariski topology, and
  $\OO_X$ to be the sheaf of regular functions on $X$, i.e.
  for $U \subseteq X$ open, we have
  \[
    \OO_X(U)
    := \{
      \varphi : U \to k \mid
      \varphi \text{ is regular}
    \}.
  \]
  Here, $\varphi : U \to k$ is
  \emph{regular} if for each $p \in U$,
  there exists an open set
  $p \in U_p \subseteq U$ and
  $f, g \in k[x_1, \dots, x_n]$
  such that $\varphi(x) = f(x) / g(x)$
  for all $x \in U_p$.
\end{remark}

\begin{remark}[Coordinate ring]
  The \emph{coordinate ring} of $X$ is
  \[
    A(X) :=
    \OO_X(X)
    \cong k[x_1, \dots, x_n] / I(X).
  \]
  We also get a version of
  Hilbert's Nullstellensatz for
  $A(X)$:
  \[
    \text{closed subsets of $X$}
    \longleftrightarrow
    \text{radical ideals of $A(X)$}.
  \]
\end{remark}

\begin{remark}[Distinguished open sets]
  The \emph{distinguished open sets} of $X$
  are
  \[
    D(f)
    := \{x \in X : f(x) \ne 0\}
    = X \setminus V(f).
  \]
  These for a basis for $X$ as we vary
  $f \in A(X)$.
\end{remark}

\begin{definition}
  An \emph{affine variety} is a
  ringed space $(X, \OO_X)$ (here
  $\OO_X$ is a sheaf of $k$-valued
  functions) which is
  isomorphic to an embedded affine
  variety.
\end{definition}

\begin{example}
  If $(X, \OO_X)$ is an affine variety
  and $f \in \OO_X(X)$, then
  \[
    (D(f), \OO_X|_{D(f)})
  \]
  is again an affine variety. To
  see this, we may assume that
  $X = V(J) \subseteq \Affine^n$
  with $J \le k[x_1, \dots, x_n]$
  a radical ideal. Now we can define
  a map
  \begin{align*}
    D(f) &\longrightarrow
    V(J, fy - 1) \subseteq \Affine^n_{x_i} \times \Affine^1_y \\
    x &\longmapsto (x, 1 / f(x)),
  \end{align*}
  which one can check is an isomorphism.
  Now that this also shows
  \[
    \OO_X(D(f))
    = A(D(f))
    \cong \frac{k[x_1, \dots, x_n, y]}{(J, fy - 1)}
    \cong \frac{(k[x_1, \dots, x_n] / J)[y]}{(\overline{f} y - 1)}
    \cong A(X)_f.
  \]
\end{example}

\begin{theorem}
  There is an equivalence of
  categories
  \begin{align*}
    \Phi :
    \mathrm{Aff\text{-}var}
    &\longrightarrow
  \mathrm{Red\text{-}f.g.\text{-}} k \mathrm{\text{-}alg}^{\mathrm{op}} \\
  (X, \OO_X)
    &\longmapsto A(X).
  \end{align*}
  This implies the following:
  \begin{enumerate}
    \item There is a bijection
      \begin{align*}
        \Hom_{\mathrm{aff}\text{-}\mathrm{var}}(X, Y)
        &\longrightarrow
        \Hom_{k\text{-}\mathrm{alg}}(A(Y), A(X)) \\
        f &\longmapsto f^*.
      \end{align*}
    \item For any reduced finitely
      generated $k$-algebra $A$,
      there exists an affine variety
      with $A \cong A(X)$.
  \end{enumerate}
\end{theorem}

\begin{remark}
  How can we explicitly
  define the inverse
  functor
  $
  \mathrm{Red\text{-}f.g.\text{-}} k \mathrm{\text{-}alg}^{\mathrm{op}}
    \rightarrow
  \mathrm{Aff\text{-}var}
  $?
  We can define this as
  $A \mapsto (X, \OO_X)$, where
  $X$ is the set of maximal ideals
  of $A$. Think about what
  $\OO_X$ should be.
\end{remark}

\begin{remark}[Varieties]
  A \emph{variety} $(X, \OO_X)$
  is a ringed space such that
  \begin{itemize}
    \item there exists a finite open
      cover of $X$ by affine varieties,
    \item the diagonal $\Delta_X$ is
      closed in $X \times X$.
  \end{itemize}
\end{remark}

\begin{example}
  The following are examples of
  varieties:
  \begin{itemize}
    \item affine varieties,
    \item open or closed subsets of
      varieties,
    \item $\PP^n = (\Affine^{n + 1} \setminus \{0\}) / k^\times$.
  \end{itemize}
\end{example}

\begin{remark}[Projective spaces]
  Recall that $\PP^n$ has an
  open cover by
  \[
    U_i
    = \{[x] \in \PP^n : x_i \ne 0\}
    \cong \Affine^n.
  \]
  A basis for $\PP^n$ by distinguished
  open sets is given by
  \[
    D(f)
    = \{[x] \in \PP^n : f(x) \ne 0\}
  \]
  with $f \in k[x_0, \dots, x_n]$
  homogeneous.
\end{remark}

\section{Vector and Line Bundles}

\begin{definition}
  Let $X$ be a variety. A
  \emph{vector bundle (of rank $m$)}
  on $X$ is a variety
  $\EE$ with a morphism
  $p : \EE \to X$ such that
  \begin{enumerate}
    \item $\EE_x = p^{-1}(X)$ has the
      structure of a rank $m$
      vector space for every $x \in X$ 
      (i.e. $k^m$),
    \item for every $x \in X$,
      there exists an open neighborhood
      $x \in U \subseteq X$ 
      and an isomorphism
      $p^{-1}(U) \to U \times \Affine^m$
      such that for any $y \in U$,
      the map
      $\EE_y \to \{y\} \times \Affine^m$
      is an isomorphism of vector
      spaces, i.e.
      \[
        \begin{tikzcd}
          p^{-1}(U)
          \ar[rr, "\phi_U"] \ar[dr ] & &
          U \times \Affine^m \ar[dl] \\
          & U
        \end{tikzcd}
      \]
      commutes.
      We will call the map $\phi_U$ a
      \emph{trivialization.}
  \end{enumerate}
\end{definition}

\begin{definition}
  A \emph{line bundle} on $X$ is a
  rank $1$ vector bundle.
\end{definition}

\begin{remark}
  A different way to think about
  this is the following:
  \begin{enumerate}
    \item Given two trivializations
      $\phi_U : p^{-1}(U) \to U \times \Affine^m$
      and $\phi_V : p^{-1}(V) \to V \times \Affine^m$,
      we get a morphism
      \[
        \begin{tikzcd}
        (U \cap V) \times \Affine^m
        \ar[rr, "\phi_{U, V}"]
        \ar[dr, "\phi_V^{-1}", swap]
        & & (U \cap V) \times \Affine^m \\
        & p^{-1}(U \cap V) \ar[ur, "\phi_U", swap]
        \end{tikzcd}
      \]
      with $\phi_{U, V} = \phi_U \circ \phi_V^{-1}$.
      Observe
      $\phi_{U, V}(x, v) = (x, g_{U, V}(x) v)$
      for some
      $g_{U, V}(x) \in \GL(m, k)$.
      Furthermore,
      $g_{U, V} : U \cap V \to \GL(m, k)$
      is a morphism.
      We will call the
      $g_{U, V}$ \emph{transition functions}.

      In the special case where
      $m = 1$ (so
      $\EE$ is a line bundle
      and $\GL(1, k) = k^\times$),
      the map
      $g_{U, V} : U \cap V\to \GL(m, k)$
      is equivalent to the data of a
      non-vanishing function
      $g_{U, V} : U \cap V \to k$.
    \item The data of a vector bundle
      of rank $m$
      is equivalent to the data of
      \begin{itemize}
        \item an open cover
          $X = \bigcup_{i \in I} U_i$,
        \item and morphisms
          $g_{i, j} : U_i \cap U_j \to \GL(m, k)$
      \end{itemize}
      such that
      $g_{i, k} = g_{i, j} g_{j, k}$,
      $g_{i, j} = g_{j, i}^{-1}$,
      and $g_{i, i} = \id$.

      To recover the vector bundle,
      we can glue
      $\EE_i = U_i \times \Affine^m$
      for $i \in I$
      via
      \begin{align*}
        \EE_{i, j}
        = (U_i \cap U_j) \times \Affine^m
        &\longrightarrow
        \EE_{j, i}
        = (U_j \cap U_i) \times \Affine^m \\
        (x, v) &\longmapsto
        (x, g_{i, j}(x) v).
      \end{align*}
      One can check that this defines
      a vector bundle $\EE$ on $X$.
  \end{enumerate}
\end{remark}

\begin{example}[Trivial vector bundle]
  Define the vector bundle
  $\EE : X \times \Affine^m \to X$
  by $(x, v) \to x$.
  Given a cover $X = \bigcup_{i \in I} U_i$,
  we get $g_{i, j} : U_i \cap U_j \to \GL(m, k)$
  as $x \mapsto I_m$.
\end{example}

\begin{example}[Trivial line bundle]
  We will denote the trivial
  line bundle by
  $\mathbbm{1}_X : X \times \Affine^1 \to X$.
\end{example}

\begin{example}
  Let $X = \PP^n$ and
  $\mathbb{L} = \{(\ell, x) \in \PP^n \times \Affine^{n + 1} : x \in \ell\}$. Consider
  \[
    \begin{tikzcd}
      & \mathbb{L} \ar[dl, "p"] \ar[dr, "q"] \\
      \PP^n & & \Affine^{n + 1}
    \end{tikzcd}
  \] 
  The map $q : \mathbb{L} \to \Affine^{n + 1}$ is the blowup.
  We claim that $p : \mathbb{L} \to \PP^n$
  is a line bundle. We have:
  \begin{itemize}
    \item $p^{-1}([x]) = kx$,
      a $1$-dimensional vector space;
    \item let $U_i = \{[x] \in \PP^n : x_i \ne 0\}$,
      then we can define
      \begin{align*}
        p^{-1}(U_i)
        &\longrightarrow U_i \times \Affine^1 \\
        ([x], y)
        &\longmapsto ([x], y_i),
      \end{align*}
      which one can check is
      a trivialization.
  \end{itemize}
\end{example}

  \chapter{Jan.~15 --- Vector and Line Bundles}

\section{Vector and Line Bundles, Continued}

\begin{example}[Tautological bundle]
  Let $X = \PP^n$ and
  $\mathbb{L} = \{(\ell, x) \in \PP^n \times \Affine^{n + 1} : x \in \ell\}$. Consider
  \[
    \begin{tikzcd}
      & \mathbb{L} \ar[dl, "p", swap] \ar[dr, "q"] \\
      \PP^n & & \Affine^{n + 1}
    \end{tikzcd}
  \] 
  The map $q : \mathbb{L} \to \Affine^{n + 1}$ is the blowup.
  We claim that $p : \mathbb{L} \to \PP^n$
  is a line bundle. We have:
  \begin{itemize}
    \item $p^{-1}([x]) = \{([x], cx) : c \in k\} \cong kx$,
      a $1$-dimensional vector space;
    \item let $U_i = \{[x] \in \PP^n : x_i \ne 0\}$,
      then we can define
      \begin{align*}
        p^{-1}(U_i)
        &\longrightarrow U_i \times \Affine^1 \\
        ([x], y)
        &\longmapsto ([x], y_i),
      \end{align*}
      which we claim is
      a trivialization. To see this,
      observe that for fixed
      $[x] \in \PP^n$, we have
      \begin{align*}
        \mathbb{L}_{[x]}
        = \{([x], cx) : c \in k\}
        &\longrightarrow
        \{[x]\} \times \Affine^1 \\
        ([x], cx)
        &\longmapsto
        ([x], cx_i),
      \end{align*}
      which is a vector space
      isomorphism.
  \end{itemize}
  We can also compute the
  transitions functions. Let
  $U_{i, j} = U_i \cap U_j$. We have
  \[
    \begin{tikzcd}[row sep=small]
      U_{i, j} \times \Affine^1
      \ar[r, "{\phi_j^{-1}}", swap]
      \ar[rr, bend left=15, "{\phi_{i, j}}"]
      & p^{-1}(U_{i, j})
      \ar[r, "{\phi_i}", swap]
      & U_{i, j} \times \Affine^1 \\
      ([x], t)
      \ar[r, mapsto] &
      ([x], (t x_0 / x_j, \dots, t x_n / x_j))
      \ar[r, mapsto] &
      ([x], tx_i / x_j).
    \end{tikzcd}
  \]
  Thus we see that
  $g_{i, j} = x_i / x_j$.
  This is called the \emph{tautological bundle},
  or $\OO_{\PP^n}(-1)$.
\end{example}

\begin{example}[Hyperplane bundle, or $\OO_{\PP^n}(1)$]
  Consider
  \begin{align*}
    \mathbb{L} := \PP^{n + 1} \setminus \{[0 : \cdots : 0 : 1]\}
    &\longrightarrow \PP^n \\
    {[x_0 : \cdots : x_n : x_{n + 1}]}
    &\longmapsto {[x_0 : \cdots : x_n]}.
  \end{align*}
  Then $\mathbb{L}$ is a line
  bundle with transition functions
  with respect to $\{U_i\}$ given
  by $g_{i, j} = x_j / x_i$ (HW).
\end{example}

\section{Operations on Vector Bundles}

\begin{remark}
  The philosophy is:
  Every natural operation of
  vector spaces gives one for
  vector bundles.
\end{remark}

\begin{example}[Direct sum]
  Let
  $p : \EE \to X$ and $q : \mathbb{F} \to X$
  be vector bundles of rank
  $e$ and $f$ on $X$, respectively.
  There exists trivializations
  with respect to a common open cover
  $\{U_i\}$ (just take intersections)
  with transition functions
  $g_{i, j}$ and $h_{i, j}$ for
  $\EE$ and $\mathbb{F}$, respectively.

  Then we define the vector bundle
  $\EE \oplus \mathbb{F}
    \longrightarrow X$
  as follows:
  \begin{itemize}
    \item As a set, it is
      $r : \EE \oplus \mathbb{F} = \{(x, u, v) : (x, u) \in \EE, (x, v) \in \mathbb{F}\} \to X$.
    \item We give $\EE \oplus \mathbb{F}$
      the structure of a variety by
      requiring that
      \begin{align*}
        r^{-1}(U_i)
        &\longrightarrow U_i \times \Affine^{e + f} \\
        (x, u, v)
        &\longmapsto
        (x, \pr_2(\phi^E_i(x, u)), \pr_2(\phi^F_i(x, v)))
      \end{align*}
      be an isomorphism,
      where $\phi^E_i$ and $\phi^F_i$
      are the trivializations of
      $\EE$ and $\mathbb{F}$,
      and $\pr_2$ is the
      second projection.
      This gives a variety structure on
      $r^{-1}(U_i)$, and one can show that
      these are consistent
      on $U_{i, j}$, so that this
      gives a variety structure on
      all of $\EE \oplus \mathbb{F}$.
  \end{itemize}
  Note that the transition functions
  for $\EE \oplus \mathbb{F}$
  with respect to $\{U_i\}$ are given
  by the block matrix
  \[
    \begin{bmatrix}
      g_{i, j} & 0 \\
      0 & h_{i, j}
    \end{bmatrix}
    : U_{i, j} \longrightarrow \GL(e + f, k).
  \]
\end{example}

\begin{example}
  Let $\EE$ and
  $\mathbb{F}$ be vector bundles on $X$
  of ranks $e$ and $f$, respectively.
  Then the following are
  also vector bundles on $X$:
  \begin{enumerate}
    \item $\Hom(\EE, \mathbb{F})$,
      of rank $ef$;
    \item $\EE^\vee = \Hom(\EE, \mathbbm{1}_X)$,
      of rank $e$;
    \item $\EE \otimes \mathbb{F}$,
      of rank $ef$;
    \item $\wedge^k \EE$
      and $\mathrm{Sym}^d \EE$.
  \end{enumerate}
\end{example}

\begin{remark}
  Let $\mathbb{L}, \mathbb{M}$ be line bundles
  on $X$ with trivializations
  on $\{U_i\}$ and transition functions
  $g_{i, j}, h_{i, j} \in \OO_X(U_{i, j})^\times$.
  In this case, we can describe
  operations on
  $\mathbb{L}, \mathbb{M}$
  more explicitly:
  \begin{enumerate}
    \item $\mathbb{L} \otimes \mathbb{M}$
      has transition functions
      $g_{i, j} h_{i, j}$;
    \item $\Hom(\mathbb{L}, \mathbb{M})$
      has transition functions
      $h_{i, j} / g_{i, j}$;
    \item $\mathbb{L}^\vee = \Hom(\mathbb{L}, \mathbbm{1}_X)$
      has transition functions
      $1 / g_{i, j}$;
    \item $\mathbb{L}^{\otimes m} =
      \begin{cases}
        \mathbb{L}^{\otimes m}, & \text{if } m > 0, \\
        \mathbbm{1}_X, & \text{if } m = 0, \\
        (\mathbb{L}^\vee)^{\otimes -m}, & \text{if } m < 0
      \end{cases}$
      has transition functions
      $g_{i, j}^m$.
  \end{enumerate}
\end{remark}

\begin{example}
  Define $\OO_{\PP^n}(m) := \OO_{\PP^n}(1)^{\otimes m}$
  with transition functions
  $(x_j / x_i)^m$ with respect
  to the standard open cover
  for $\PP^n$.
\end{example}

\section{Morphisms of Vector Bundles}

\begin{remark}
  Let $p : \EE \to X$ and
  $q : \mathbb{F} \to X$ be
  vector bundles on $X$, as before.
\end{remark}

\begin{definition}
  A \emph{morphism of vector
  bundles} $\EE \to \mathbb{F}$
  is a morphism of
  varieties
  \[
    \begin{tikzcd}
      \EE \ar[rr, "a"] \ar[dr, "p", swap] & & \mathbb{F} \ar[dl, "q"] \\
      & X
    \end{tikzcd}
  \]
  such that the diagram commutes
  and $a$ is linear on each fiber.
\end{definition}

\begin{remark}
  More concretely, given an
  open cover
  $\{U_i\}$ which trivializes
  both vector bundles, we have
  \[
    \begin{tikzcd}[row sep=small]
      p^{-1}(U_i) \ar[r, "a"]
      \ar[dd, "\phi_i", swap]
      \ar[dd, "\cong"]
      & q^{-1}(U_i)
      \ar[dd, "\psi_j"]
      \ar[dd, "\cong", swap]
      \\ \\
      U_i \times \Affine^e
      \ar[r] &U_i \times \Affine^f \\
      (x, v)
      \ar[r, mapsto]
             & (x, a_i(x) v)
    \end{tikzcd}
  \]
  such that $a_i : U_i \to \Hom(k^e, k^f)$
  is regular.
  On $U_{i, j}$, we have
  \[
    \begin{tikzcd}
      U_{i, j} \times \Affine^e
      \ar[r, "a_j"]
      \ar[d, "g_{i, j}", swap]
      & U_{i, j} \times \Affine^f
      \ar[d, "h_{i, j}"] \\
      U_{i, j} \times \Affine^e
      \ar[r, "a_i", swap]
      & U_{i, j} \times \Affine^f
    \end{tikzcd}
  \]
  So $h_{i, j} a_j = a_i g_{i, j}$,
  or equivalently,
  $a_i = h_{i, j} a_j g_{i, j}^{-1}$.

  As a special case when
  $e = f$,
  $a : \EE \to \mathbb{F}$
  is an isomorphism if and only if
  the $a_i$ are isomorphisms.
\end{remark}

\begin{remark}
  When is a line bundle
  $\mathbb{L}$ given by
  the trivialization data
  $\{U_i, g_{i, j}\}$ isomorphic
  to $\mathbbm{1}_X$?
  We have
  \begin{align*}
    \mathbb{L} \cong \mathbbm{1}_X
    &\iff
    \text{if and only if there exists
    an isomorphism }
    a : \mathbbm{1}_X \to \mathbb{L} \\
    &\iff
    \text{there exist }
    a_i \in \OO_X(U_i)^\times
    \text{ such that }
    (a_j / a_i)|_{U_{i, j}} = g_{i, j}.
  \end{align*}
\end{remark}

\begin{definition}
  Define the \emph{Picard group}
  of $X$ to be
  \[
    \Pic X
    := \{\text{line bundles on $X$}\} / {\cong}.
  \]
  This is a group with respect
  to $\otimes$ with
  $\mathbbm{1}_X$
  as the identity and
  $\mathbb{L}^\vee \otimes \mathbb{L} \cong \mathbbm{1}_X$.
\end{definition}

\section{Global Sections}

\begin{definition}
  A \emph{(global) section}
  of a vector bundle $p : \EE \to X$
  is a morphism
  $s : X \to \EE$
  such that $p \circ s = \id_X$.
  Note that for $x \in X$, we have
  $s(x) \in \EE_x$.
\end{definition}

\begin{example}[Zero section]
  Let $s : X \to \EE$ where
  $s(x)$ is the zero element
  in $\EE_x$.
\end{example}

\begin{example}
  Let $\EE = \mathbbm{1}_X$.
  Then sections $s : X \to X \times \Affine^1$
  of $\EE$ correspond to
  morphisms $X \to \Affine^1$,
  which correspond to
  regular functions $X \to k$.
\end{example}

\begin{remark}[Local description of sections]
  Let $\{U_i, g_{i, j}\}$ be the
  trivialization data for
  $\EE \to X$, and let
  $s : X \to \EE$ be a section.
  On $U_i$, we have:
  \[
    \begin{tikzcd}
      & U_i \times \Affine^e \\
      U_i \ar[r, "{s|_{U_i}}", swap] \ar[ur, "{x \mapsto (x, s_i(x))}"] & p^{-1}(U_i) \ar[u, "\phi_i", swap] \ar[u, "\cong"]
    \end{tikzcd}
  \]
  Note that $s_i : U_i \to k^e$
  is a regular function (i.e.
  regular on each coordinate).
  These maps must satisfy
  the compatibility condition
  $s_i = g_{i, j} s_j$, since
  we have the diagram:
  \[
    \begin{tikzcd}
      & & U_{i, j} \times \Affine^e \ar[dd, "{(x, v) \mapsto (x, g_{i, j}(v))}"] \\
      U_{i, j} \ar[urr, bend left=30, "{(x, s_j(x))}"] \ar[drr, bend right=30, swap, "{(x, s_i(x))}"] \ar[r, "s|_{U_{i, j}}"]& p^{-1}(U_{i, j}) \ar[ur, "\phi_j"] \ar[dr, swap, "\phi_i"] \\
               & & U_{i, j} \times \Affine^e
    \end{tikzcd}
  \]
\end{remark}

\begin{example}
  We can use the above compatibility
  condition to compute the
  global sections of $\OO_{\PP^1}(1)$.
  Write $\PP^1_{x_0 : x_1} = U_0 \cup U_1$.
  Given a section $s : \PP^1 \to \OO(1)$,
  we get regular functions
  \begin{align*}
    s_0 : U_0 &\longrightarrow k \\
    s_1 : U_1 &\longrightarrow k
  \end{align*}
  satisfying
  $(x_1 / x_0) s_1 = s_0$ $(*)$.
  We can write
  \[
    s_0 = \sum_{m \ge 0} a_m (x_1 / x_0)^m
    \quad \text{and}
    \quad
    s_1 = \sum_{m \ge 0} b_m (x_0 / x_1)^m
  \]
  with $a_m, b_m \in k$ (finitely
  many nonzero).
  Then $(*)$ implies that
  \[
    a_0 + a_1 (x_1 / x_0) + \cdots
    = (x_1 / x_0)(b_0 + b_1 (x_0 / x_1) + \cdots),
  \]
  so $a_0 = b_1$, $a_1 = b_0$, and
  all other terms are $0$. So
  we can relate $s$ to a linear form
  \[
    f = a_0 x_0 + a_1 x_1,
  \]
  where $s_0 = (1 / x_0) f$
  and $s_1 = (1 / x_1) f$.
\end{example}

  \chapter{Jan.~20 --- Sections}

\section{Global Sections, Continued}

\begin{definition}
  Let $\Gamma(X, \EE) := \{\text{sections of } \EE \to X\}$,
  which
  has the structure of a
  $k$-vector space by
  \[
    (s + t)(x) = s(x) + t(x)
    \quad \text{and} \quad
    (cs)(x) = c s(x)
  \]
  for $s, t \in \Gamma(X, \EE)$ and
  $c \in k$.
\end{definition}

\begin{example}
  One can check that
  $\Gamma(\PP^n, \OO_{\PP^n}(d)) \cong k[x_0, \dots, x_n]_d$ (HW).
  For example, for $d < 0$, we have
  $\Gamma(\PP^n, \OO(d)) = \{0\}$, and
  for $d = 0$, we have
  \[
    \Gamma(\PP^n, \OO(0))
    = \Gamma(\PP^n, \mathbbm{1}_{\PP^n})
    = \OO_{\PP^n}(\PP^n)
    = k.
  \]
  For $d = 1$,
  we can define an isomorphism
  \begin{align*}
    k[x_0, \dots, x_n]_1
    &\overset{\cong}{\longrightarrow}
    \Gamma(\PP^n, \OO(1)) \\
    f &\longmapsto
    \text{section } s : \PP^n \to \OO(1)
    \text{ given by }
    s_i = f / x_i.
  \end{align*}
  Note that $(x_j / x_i) s_j = s_i$
  holds, so this is a section.
  An alternative perspective
  is that $s$ corresponds to
  \begin{align*}
    \PP^n &\longrightarrow \PP^{n + 1} \setminus \{[0 : \cdots : 0 : 1]\} \\
    x &\longmapsto [x_0 : \cdots : x_n : f(x)].
  \end{align*}
\end{example}

\section{Morphisms and Sections}

\begin{definition}
  Given a section $s : X \to \EE$,
  its \emph{vanishing locus} is
  \[
    V(s) := \{s = 0\}
    = \{x \in X : s(x) = 0\}.
  \]
  Using a trivializing cover, one
  can check that $V(s)$ is closed in
  $X$.
\end{definition}

\begin{example}
  For a section $s : \PP^n \to \OO(1)$
  corresponding to
  $f \in k[x_0, \dots, x_n]_1$,
  we have $V(s) = V_{\PP^n}(f)$.
\end{example}

\begin{remark}
  Recall that there is a bijection
  \begin{align*}
    \{\text{morphisms } X \to \Affine^n\}
    &\longleftrightarrow
    \{f_1, \dots, f_n \in \OO_X(X)\} \\
    [f : X \to \Affine^n]
    &\longmapsto
    [f_1 = f^* x_1, \dots, f_n = f^* x_i \in \OO_X(X)] \\
    [x \mapsto (f_1(x), \dots, f_n(x))] &\mathrel{\reflectbox{\ensuremath{\longmapsto}}} [f_1, \dots, f_n \in \OO_X(X)].
  \end{align*}
  We want a similar statement
  for $\PP^n$.
\end{remark}

\begin{definition}
  Given a line bundle
  $\LL \to X$ and
  $s_0, \dots, s_n \in \Gamma(X, \LL)$,
  they
  are \emph{nowhere vanishing}
  if
  \[V(s_0) \cap \cdots \cap V(s_n) = \varnothing.\]
\end{definition}

\begin{example}
  For $\OO(1)$, the sections
  $x_0, \dots, x_n \in \Gamma(\PP^n, \OO(1))$
  are nowhere vanishing.
\end{example}

\begin{remark}
  If $s_0, \dots, s_n \in \Gamma(X, \LL)$
  are nowhere vanishing, then we
  get a morphism
  \begin{align*}
    X &\longrightarrow \PP^n \\
    x &\longmapsto
    [s_0(x) : \cdots : s_n(x)].
  \end{align*}
  Note that $(s_0(x), \dots, s_n(x))$
  is a well-defined point in
  $\Affine^{n + 1}$ up to scaling.
  One can check that this map is a morphism
  by working locally.
\end{remark}

\begin{example}[Linear maps]
  Let $X = \PP^n$ and $\LL = \OO(1)$.
  \begin{enumerate}[(i)]
    \item $x_0, \dots, x_n \in \Gamma(\PP^n, \OO(1))$
      gives
      $\id : \PP^n \to \PP^n$.
    \item For $A \in \GL_{n + 1}(k)$,
      we get a map
      \begin{align*}
         \PP^n &\longrightarrow \PP^n \\
         [x] &\longmapsto
         [A x]
      \end{align*}
      given by
      $A x_0, \dots, A x_n \in \Gamma(\PP^n, \OO(1))$.
  \end{enumerate}
\end{example}

\begin{remark}
  Now given a morphism
  $X \to \PP^n$, we want to get
  a line bundle with sections.
\end{remark}

\begin{definition}[Pullback]
  Let $p : \EE \to X$ be a vector
  bundle and
  $f : Y \to X$ a morphism.
  Define
  \[
    f^* \EE
    = \{
      (e, y) : e \in \EE, y \in Y
      \text{ with } p(e) = f(y)
    \}
    \longrightarrow Y.
  \]
  One can show that this has
  the structure of a vector bundle
  in a natural way.
\end{definition}

\begin{remark}
  An alternative way to define the
  pullback is to choose
  trivialization data
  $(U_i, g_{i, j})$ for $\EE \to X$.
  Then we can define
  $f^* : \EE \to Y$ to be the
  vector bundle with
  trivialization data
  $(f^{-1}(U_i), f^* g_{i, j})$.
\end{remark}

\begin{remark}
  Now to go in reverse,
  given a morphism
  $X \to \PP^n$ and nowhere
  vanishing sections $x_0, \dots, x_n \in \Gamma(\PP^n, \OO(1))$, we get
  nowhere vanishing sections
  \[
    f^* x_0, \dots, f^* x_n \in \Gamma(X, f^* \OO(1)).
  \]
  We can define the pullback of a
  section in one of two ways:
  by
  $f^*(x_i)(a) = (x_i(f(a)), a) \in f^* \OO(1))$
  for $a \in X$ or
  by using trivializing covers.
\end{remark}

\begin{remark}
  Using the above, we get a bijection
  \begin{align*}
    \{{\text{morphisms } X \to \PP^n}\}
    &\longleftrightarrow
    \{
      \text{line bundles } \LL \to X
      \text{ with }
      s_0, \dots, s_n \in \Gamma(X, \LL)
      \text{ nowhere vanishing}
    \}.
  \end{align*}
  Note that we should consider the 
  right-hand side up to isomorphism
  of the line bundle.
  When do $\LL \to X$ and
  $s_0, \dots, s_n \in \Gamma(X, \LL)$
  give an injective morphism
  (or an embedding)?
\end{remark}

\begin{definition}
  Given a vector bundle $\EE \to X$,
  we get a sheaf of abelian groups
  $\mathcal{E}$ on $X$ by
  \[
    \mathcal{E}(U)
    := \{\text{sections of } p^{-1}(U) \to U\}
  \]
  for $U \subseteq X$ open.
  For $V \subseteq U \subseteq X$ open,
  the restriction map is given by
  \begin{align*}
    \mathcal{E}(U) &\longrightarrow \mathcal{E}(V) \\
    s &\longmapsto s|_V.
  \end{align*}
  \pagebreak
  We call $\mathcal{E}$ the
  \emph{sheaf of sections} of $\EE$.
  Also note that $\mathcal{E}(U)$ has the
  structure of an $\OO_X(U)$-module.
  We will see that this gives rise
  to the structure of an
  $\OO_X$-module.
\end{definition}

\section{Review of Sheaves}

\begin{definition}
  A \emph{presheaf} of abelian
  groups $\mathcal{F}$ on a topological
  space $X$ is the data of:
  \begin{itemize}
    \item for $U \subseteq X$ open,
      an abelian group $\mathcal{F}(U)$ (with $\mathcal{F}(\varnothing) = 0$),
    \item for $V \subseteq U \subseteq X$
      open, a group homomorphism
      $p_{V, U} : \mathcal{F}(U) \to \mathcal{F}(V)$.
  \end{itemize}
\end{definition}

\begin{remark}
  Note the following:
  \begin{enumerate}
    \item We may replace abelian groups in
      the above definition by
      rings, sets,
      $R$-modules, etc.
    \item We denote
      $\Gamma(U, \mathcal{F}) = \mathcal{F}(U)$,
      whose elements are called
      \emph{sections}.
    \item $s|_V := p_{V, U}(s)$
      is called the \emph{restriction}
      for $s \in \mathcal{F}(U)$ and
      $V \subseteq U \subseteq X$
      open.
    \item We may view $\mathcal{F}$
      as a functor
      $\mathrm{Open}_X \to \mathrm{Ab}\text{-}\mathrm{gps}$
      given by $U \mapsto \mathcal{F}(U)$.
  \end{enumerate}
\end{remark}

\begin{definition}
  For $\mathcal{F}$ a presheaf on
  $X$ and $x \in X$, the
  \emph{stalk} of $\mathcal{F}$ at $x$ is
  \[
    \mathcal{F}_x
    = \lim_{\substack{\longrightarrow \\ U \ni x \text{ open}}}
    \mathcal{F}(U)
    = \{(s, U) : s \in \mathcal{F}(U)\} / {\sim}.
  \]
\end{definition}

\begin{example}
  The following are examples of
  presheaves:
  \begin{enumerate}
    \item Let $M$ be a smooth manifold.
      Then
      \begin{itemize}
        \item $\OO_M =$ sheaf of
          smooth $\R$-valued functions on $M$,
        \item $\mathcal{E} =$
          sheaf of sections of a
          vector bundle $\EE \to M$.
      \end{itemize}
    \item Let $X$ be an algebraic
      variety, $\EE \to X$ a
      vector bundle, and
      $Z \subseteq X$ closed. Then
      \begin{itemize}
        \item $\OO_X$ and
          $\mathcal{E}$ are sheaves,
        \item $\mathcal{I}_Z =$
          ideal sheaf of $Z$, given by
          $\mathcal{I}_Z(U) = \{\varphi \in \OO_X(U) : \varphi|_Z = 0\}$.
      \end{itemize}
    \item Let $X$ be a topological
      space and $A$ an abelian group.
      \begin{itemize}
        \item $\underline{A}^{\mathrm{pre}}$
          given by
          $U \mapsto \{\text{constant functions } U \to A\}$, i.e.
          $\underline{A}^{\mathrm{pre}}(U) \cong A$ for
          $U \ne \varnothing$,
        \item $\underline{A}$
          given by
          $U \mapsto \{\text{locally constant functions $U \to A$}\}$,
        \item $i_p A =$ skyscraper
          sheaf, given by
          $U \mapsto
          \begin{cases}
            A & \text{if } p \in U, \\
            0 & \text{otherwise}.
          \end{cases}$
      \end{itemize}
  \end{enumerate}
\end{example}

\begin{definition}
  A presheaf $\mathcal{F}$ is a
  \emph{sheaf} if for any
  \begin{itemize}
    \item open set $U \subseteq X$,
    \item open cover
      $U = \bigcup_{i \in I} U_i$,
    \item and $s_i \in \mathcal{F}(U_i)$
      such that $s_i|_{U_{i, j}} = s_j|_{U_{i, j}}$
      for all $i, j \in I$,
  \end{itemize}
  then there exists a unique
  $s \in \mathcal{F}(U)$ such that
  $s|_{U_i} = s_i$ for every $i \in I$.
\end{definition}

\begin{remark}
  The presheaf
  $\underline{A}^{\mathrm{pre}}$
  is not a sheaf in general.
  All other examples above are sheaves.
\end{remark}

\pagebreak
\begin{definition}
  A \emph{morphism} of (pre)sheaves
  $\varphi : \mathcal{F} \to \mathcal{G}$
  on a topological space $X$ is
  the data of group homomorphisms
  $\varphi(U) : \mathcal{F}(U) \to \mathcal{G}(U)$
  for each $U \subseteq X$ open
  such that for all $V \subseteq U \subseteq X$, the following
  diagram commutes:
  \[
    \begin{tikzcd}
      \mathcal{F}(U) \ar[d, swap, "\mathrm{res}"] \ar[r, "{\varphi(U)}"]
      & \mathcal{G}(U) \ar[d, "\mathrm{res}"] \\
      \mathcal{F}(V) \ar[r, swap, "{\varphi(V)}"]
      & \mathcal{G}(V)
    \end{tikzcd}
  \]
\end{definition}

\begin{example}
  Let $X$ be a variety.
  \begin{enumerate}
    \item If
      $a : \EE \to \mathbb{F}$ is a morphism
      of vector bundles on $X$, then
      we get a morphism of sheaves
      $\mathcal{E} \to \mathcal{F}$
      by $s \mapsto a \circ s \in \mathcal{F}(U)$
      for $s \in \mathcal{E}(U)$.
    \item A closed subvariety
      $Z \subseteq X$ induces a
      morphism $\ell_Z \to \OO_X$
      given by inclusion.
  \end{enumerate}
\end{example}

\begin{remark}
  Given a morphism of (pre)sheaves
  and $\varphi : \mathcal{F} \to \mathcal{G}$
  and $p \in X$, we get an
  induced morphism
  \begin{align*}
    \mathcal{F}_p &\longrightarrow \mathcal{G}_p \\
    (s, U) &\longmapsto
    (\varphi(s), U).
  \end{align*}
\end{remark}

  \chapter{Jan.~22 --- Sheaves}

\section{Sheafification}

\begin{theorem}[Sheafification]
  For a presheaf $\mathcal{F}$ on
  a topological space $X$, there
  exists a morphism
  to a sheaf $i : \mathcal{F} \to \mathcal{F}^+$
  such that for any morphism to a sheaf
  $g : \mathcal{F} \to \mathcal{G}$,
  there exists a unique morphism
  $g^+ : \mathcal{F}^+ \to \mathcal{G}$
  such that
  $g = g^+ \circ i$, i.e. the following
  diagram commutes:
  \[
    \begin{tikzcd}
      \mathcal{F} \ar[d, "i", swap] \ar[r, "g"] & \mathcal{G} \\
      \mathcal{F}^+ \ar[ur, swap, "g^+", dashed]
    \end{tikzcd}
  \]
  In the above, $\mathcal{F}^+$
  is called the \emph{sheafification}
  of $\mathcal{F}$, and the pair
  $(i, \mathcal{F}^+)$ is unique
  up to isomorphism (as a consequence
  of the universal property).
\end{theorem}

\begin{proof}
  We first define
  $\mathcal{F}^+(U) = \{t : U \to \bigsqcup_{p \in X} \mathcal{F}_p : \text{(1) and (2) hold}\}$, where
  \begin{enumerate}
    \item $t(p) \in \mathcal{F}_p$;
    \item for any $x \in X$, there
      is an open set
      $x \in V_x \subseteq U$
      with $s \in \mathcal{F}(V_x)$
      such that $t(p) = s_p$ for
      all $p \in V_x$.
  \end{enumerate}
  It is straightforward to see that
  $\mathcal{F}^+$ is a sheaf and
  that
  \begin{align*}
    \mathcal{F}(U) &\longrightarrow \mathcal{F}^+(U) \\
    s
    &\longmapsto (X \ni p \mapsto s_p \in \mathcal{F}_p).
  \end{align*}
  gives a morphism
  $i : \mathcal{F} \to \mathcal{F}^+$.
  Now we check the universal property.
  Given a morphism
  $g : \mathcal{F} \to \mathcal{G}$
  with $\mathcal{G}$ a sheaf, we
  need to define $g^+ : \mathcal{F}^+ \to \mathcal{G}$.
  Fix $t \in \mathcal{F}^+(U)$.
  By definition, there exists an open
  cover  $\{U_i\}$ of $U$ and
  $s_i \in \mathcal{F}(U_i)$ such that
  $t(p) = (s_i)_p \in \mathcal{F}_p$
  for all $p \in U_i$.
  Set $t_i' := g(t_i)$. Note that
  \[
    (t_i')_p = g_p(t_p)
    = (t_j')_p
    \in \mathcal{G}_p
  \]
  for every $p \in U_i \cap U_j$.
  Since $\mathcal{G}$ is a sheaf, we
  get $t_i'|_{U_i \cap U_j} = t_j'|_{U_i \cap U_j}$.
  Thus there exists a unique
  $t' \in \mathcal{G}(U)$
  such that $t'|_{U_i} = t_i$
  for every $i \in I$. Then
  we can set $g^+(t) = t'$.
  One can check as an exercise
  that this gives a morphism
  $\mathcal{F}^+ \to \mathcal{G}$
  satisfying the universal property.
\end{proof}

\begin{example}
  We have the following:
  \begin{enumerate}
    \item If $\mathcal{F}$ is a
      sheaf, then $\mathcal{F} \to \mathcal{F}^+$
      is an isomorphism.
    \item For an abelian group
      $A$ and topological space
      $X$, we have
      $(\underline{A}^{\mathrm{pre}})^+ \cong \underline{A}$.
  \end{enumerate}
\end{example}

\begin{remark}\label{rem:properties-of-sheafification}
  We have the following:
  \begin{enumerate}
    \item If $p \in X$, the induced
      morphism on stalks
      $i_p : \mathcal{F}_p \to \mathcal{F}_p^+$ is
      an isomorphism for all
      $p \in X$.

      To construct the inverse map,
      consider $\mathcal{F}^+_p \to \mathcal{F}_p$
      defined by $(t, U) \mapsto t_p$
      for $t \in \mathcal{F}^+(U)$
      Check as an exercise that this
      is well-defined and is inverse
      to $i_p$.
    \item If $\mathcal{F} \subseteq \mathcal{G}$
      is a subpresheaf (i.e.
      $\mathcal{F}(U) \subseteq \mathcal{G}(U)$
      and $\rho_{V, U}^{\mathcal{F}} = \rho_{V, U}^{\mathcal{G}}|_{\mathcal{F}(U)}$
      for all $V \subseteq U \subseteq X$)
      and $\mathcal{G}$ is a sheaf,
      then we could alternatively
      define $\mathcal{F}^+$ as
      \[
        \mathcal{F}^+(U)
        = \{
          s \in \mathcal{G}(U)
          : \text{for all } x \in X,\,
          \text{there exists }
          x \in U_x \subseteq U
          \text{ open such that }
          s|_{U_x} \in \mathcal{F}(U_x)
        \}.
      \]
  \end{enumerate}
\end{remark}

\section{Kernel, Image, Cokernel for Sheaves}

\begin{remark}
  We want the following notions for
  sheaves:
  \begin{itemize}
    \item kernel, image, cokernel;
    \item short exact sequences;
    \item injectivity and surjectivity.
  \end{itemize}
\end{remark}

\begin{example}
  We want the following to be
  short exact sequences of sheaves:
  \begin{itemize}
    \item for $X$ a variety and
      $Z \hookrightarrow X$ a
      closed subvariety,
      \[
      \begin{tikzcd}
        0 \ar[r] & \mathcal{I}_Z \ar[r] & \OO_X \ar[r] & i_x \OO_Z \ar[r] & 0
      \end{tikzcd}
      \]
    \item for $M$ a complex manifold
      (e.g. $\C^n$) with
      $\OO_M$ the sheaf of
      $\C$-valued holomorphic functions,
      \[
        \begin{tikzcd}
          0 \ar[r] & \underline{\Z} \ar[r, "2\pi i \times"] & \OO_M \ar[r, "\varphi \mapsto e^\varphi"] & \OO_M^\times \ar[r] & 0
        \end{tikzcd}
      \]
  \end{itemize}
\end{example}

\begin{remark}
  Let $\varphi : \mathcal{F} \to \mathcal{G}$
  be a morphism of sheaves on
  a topological space $X$.
\end{remark}

\begin{definition}
  The \emph{kernel} of $\varphi$
  is $(\ker \varphi)(U) = \ker(\varphi(U) : \mathcal{F}(U) \to \mathcal{G}(U))$.
\end{definition}

\begin{remark}[Properties of the kernel]
  It is straightforward to check that
  $\ker \varphi$ is a sheaf. Moreover:
  \begin{enumerate}
    \item $\ker \varphi$
      satisfies the following
      universal property: For any
      morphism to a sheaf $\alpha$
      such that $\varphi \circ \alpha = 0$,
      there exists a unique morphism
      $\alpha'$
      such that the following diagram commutes:
      \[
        \begin{tikzcd}
          \mathcal{F}' \ar[r, "\alpha"] \ar[rr, bend left=30, "0"] \ar[dr, "\alpha'", dashed, swap]& \mathcal{F} \ar[r, "\varphi"] & \mathcal{G} \\
                                                & \ker \varphi \ar[u]
        \end{tikzcd}
      \]
      To see this, use the universal
      property of the kernel in the
      category of abelian groups.
    \item Since filtered limits are
      exact, we have
      $(\ker \varphi_p) = (\ker \varphi)_p$
      for all $p \in X$.
  \end{enumerate}
\end{remark}

\begin{lemma}[Injectivity for sheaves]
  The following are equivalent:
  \begin{enumerate}
    \item $\varphi(U) : \mathcal{F}(U) \to \mathcal{G}(U)$
      is injective for all
      $U \subseteq X$ open;
    \item $\varphi_x : \mathcal{F}_x \to \mathcal{G}_x$
      is injective for all $x \in X$.
  \end{enumerate}
  We say that $\varphi$ is
  \emph{injective} if either of these
  equivalent conditions hold.
\end{lemma}

\begin{proof}
  $(1 \Rightarrow 2)$ This is clear.

  $(2 \Rightarrow 1)$ Fix $s \in \mathcal{F}(U)$
  with $\varphi(U)(s) = 0$. Then
  \[
    \varphi_p(s_p)
    = (\varphi(U)(S))_p
    = 0
  \]
  for all $p \in X$, so
  $s_p = 0$ for all $p \in U$, so
  $s = 0$ by homework from Algebraic
  Geometry I.
\end{proof}

\begin{example}[Subtleties for the image]
  Consider the following:
  \begin{enumerate}
    \item Let $\varphi : \OO_{\C^n} \xrightarrow{\exp} \OO_{\C^n}^\times$.
      Then $U \mapsto \im(\OO_{\C^n}(U) \to \OO_{\C^n}^\times(U))$
      is a presheaf but not a sheaf.
      This is because logarithms
      only exist locally.
    \item Define
      $\varphi : \OO_{\PP^n} \to i_{p_1} \underline{k} \oplus i_{p_2} \underline{k}$
      by $f \mapsto (f(p_1), f(p_2))$.
      Again
      $U \mapsto \im(\varphi(U))$
      is not a sheaf.
  \end{enumerate}
\end{example}

\begin{definition}
  Let $\widetilde{\im}\, \varphi = (U \mapsto \im(\varphi(U))$.
  This is a presheaf and
  $\widetilde{\im}\, \varphi \subseteq \mathcal{G}$.
  Then the \emph{image} of $\varphi$
  is $\im \varphi = (\widetilde{\im}\, \varphi)^+$.
  By Remark \ref{rem:properties-of-sheafification},
  we can equivalently define
  \[
    (\im \varphi)(U)
    = \{
      s \in \mathcal{G}(U)
      : \text{there exists cover }
      \{U_i\} \text{ of } U
      \text{ such that }
      s|_{U_i} \in \im(\mathcal{F}(U_i) \to \mathcal{G}(U_i))
    \}.
  \]
\end{definition}

\begin{remark}
  We have $\im(\varphi_x) \cong (\widetilde{\im}\, \varphi)_x \cong (\im \varphi)_x$,
  where the first isomorphism
  is because filtered direct limits
  are exact and the second isomorphism
  is because sheafification
  preserves stalks.
\end{remark}

\begin{definition}
  Let $\widetilde{\coker}(\varphi) = (U \mapsto \coker(\varphi(U)))$.
  Then the \emph{cokernel} of
  $\varphi$ is $\coker(\varphi) = (\widetilde{\coker}(\varphi))^+$.
\end{definition}

\begin{remark}
  We have the following:
  \begin{enumerate}
    \item $\coker(\varphi)_x \cong \coker(\varphi_x)$
      (similar to above).
    \item $\coker(\varphi)$
      satisfies the universal
      property of the cokernel:
      \[
        \begin{tikzcd}
          \mathcal{F} \ar[r, "\varphi"] \ar[rr, "0", bend left=30]
          & \mathcal{G} \ar[r, "p_0"] \ar[d]
          & \mathcal{G}' \\
          & \widetilde{\coker}(\varphi)
          \ar[r] \ar[ur, dashed] & \coker(\varphi) \ar[u, dashed]
        \end{tikzcd}
      \]
    \item For a subsheaf
      $\mathcal{F}' \subseteq \mathcal{F}$,
      we can define the \emph{quotient sheaf}
      $\mathcal{F}/\mathcal{F}' = \coker(\mathcal{F'} \hookrightarrow \mathcal{F})$.
    \item By the universal property
      of the cokernel, we get natural
      maps
      \[
        \begin{tikzcd}
          \ker \varphi \ar[r] & \mathcal{F} \ar[r] \ar[d] & \im \mathcal{F} \\
          & \mathcal{F} / {\ker \varphi} \ar[ur, dashed, "\alpha", swap]
        \end{tikzcd}
      \]
      As the following diagram commutes,
      \[
        \begin{tikzcd}
          (\mathcal{F} / {\ker \varphi})_p \ar[r, "\alpha_p"] \ar[d, "\cong"] & (\im \varphi)_p \ar[d, "\cong", swap] \\
          \mathcal{F}_p / {(\ker \varphi)_p} \ar[r, "\cong", swap] & \im(\varphi_p)
        \end{tikzcd}
      \]
      $\alpha_p$
      is an isomorphism for all
      $p \in X$. So by HW,
      $\alpha$ is an isomorphism.
      So $\mathcal{F} / {\ker \varphi} \cong \im \varphi$.
  \end{enumerate}
\end{remark}

\begin{lemma}[Surjectivity for sheaves]
  The following are equivalent:
  \begin{enumerate}
    \item $\coker \varphi = 0$;
    \item $\im \varphi = \mathcal{G}$;
    \item $\varphi_x : \mathcal{F}_x \to \mathcal{G}_x$
      is surjective for all $x \in X$.
  \end{enumerate}
  We say that $\varphi$ is
  \emph{surjective} if any of these
  equivalent conditions hold.
\end{lemma}

\begin{proof}
  $(3 \Leftrightarrow 1)$
  We have (3) if and only if
  $\coker(\varphi_x) = 0$
  for all $x \in X$, if and only if
  $(\coker \varphi)_x = 0$ for all
  $x \in X$, if and only if (1).

  $(3 \Leftrightarrow 2)$
  We have (3) if and only if
  $\coker(\varphi_x) = 0$ for all
  $x \in X$, if and only if
  $\im \varphi_x = \mathcal{G}_x$
  for all $x \in X$, if and only if
  $(\im \varphi)_x \rightarrow \mathcal{G}_x$
  is an isomorphism for all
  $x \in X$, if and only if (2) by HW.
\end{proof}

\begin{remark}
  Note that if $\varphi(U) : \mathcal{F}(U) \to \mathcal{G}(U)$
  is surjective for all $U \subseteq X$, then
  $\varphi$ is surjective.
  However, the converse is false
  in general.
\end{remark}

\begin{definition}
  A sequence of morphisms of
  sheaves
  \[
    \begin{tikzcd}
      \mathcal{F} \ar[r, "f"]
      & \mathcal{G} \ar[r, "g"]
      & \mathcal{H}
    \end{tikzcd}
  \]
  is \emph{exact} at $\mathcal{G}$
  if $\ker g = \im f$.
\end{definition}

\begin{lemma}
  The following are equivalent:
  \begin{enumerate}
    \item $\ker g = \im f$;
    \item $\ker g_x = \im f_x$
      for all $x \in X$.
  \end{enumerate}
\end{lemma}

\begin{proof}
  Similar to above.
\end{proof}

\section{Constructions with Sheaves}

\begin{definition}
  Let $\mathcal{F}_1, \mathcal{F}_2$
  be sheaves on $X$. Then
  $\mathcal{F}_1 \oplus \mathcal{F}_2$
  is a sheaf defined by
  \[
    U \longmapsto
    \mathcal{F}_1(U) \oplus \mathcal{F}_2(U).
  \]
  This is a \emph{biproduct} in
  the category of sheaves.
\end{definition}

\begin{example}
  Let $X$ be a variety with
  connected components $X_1, \dots, X_n$.
  Then
  \[
    \OO_X \cong \OO_{X_1} \oplus \cdots \oplus \OO_{X_n}
  \]
  as sheaves of abelian groups.
\end{example}

\begin{definition}
  Let $U \subseteq X$ be open.
  Then $\mathcal{F}_i|_U$
  is a sheaf on $U$ given
  by $V \mapsto \mathcal{F}_i(V)$
\end{definition}

\begin{definition}
  $\Hom(\mathcal{F}_1, \mathcal{F}_2)$
  is the sheaf
  $U \mapsto \Hom^{\mathrm{sheaves}}(\mathcal{F}_1|_U, \mathcal{F}_2|_U)$.
\end{definition}

\begin{definition}[Gluing]
  Let $\{U_i\}$ be an open cover of
  $X$ with a sheaf
  $\mathcal{F}_i$ on each
  $U_i$ and
  isomorphisms
  $\alpha_{i, j} : \mathcal{F}_j|_{U_{i, j}} \to \mathcal{F}_i|_{U_{i, j}}$
  such that $\alpha_{i, j} = \alpha_{j, i}^{-1}$,
  $\alpha_{i, j} \circ \alpha_{j, k} = \alpha_{i, k}$, and
  $\alpha_{i, i} = \id$. Then there
  exists a sheaf $\mathcal{F}$
  with isomorphisms
  $\beta_i : \mathcal{F}|_{U_i} \to \mathcal{F}_i$
  such that the following
  diagram commutes:
  \[
    \begin{tikzcd}
      \mathcal{F}|_{U_{i, j}}
      \ar[r, "\id"] \ar[d, swap, "\beta_j"] & \mathcal{F}|_{U_{i, j}} \ar[d, "\beta_i"] \\
      \mathcal{F}_i|_{U_{i, j}} \ar[r, "\alpha_{i, j}", swap]
      & \mathcal{F}_j|_{U_{i, j}}
    \end{tikzcd}
  \]
  One can define $\mathcal{F}$
  as follows and check that it
  satisfies the above properties:
  \[
    \mathcal{F}(U)
    = \{(s_i)_{i \in I} : s_i \in \mathcal{F}(U_i) \text{ and } \alpha_{i, j}(s_j|_{U_{i, j}}) = s_i\}.
  \]
\end{definition}

  \chapter{Jan.~27 --- \texorpdfstring{$\OO_X$}{OX}-Modules}

\section{Sheaves and Continuous Maps}

\begin{remark}
  The category $\mathrm{Sh}_X$
  of sheaves of abelian
  groups on a topological space $X$
  is an abelian category, i.e.
  it has or satisfies the following:
  \begin{itemize}
    \item zero object (the $\underline{0}$
      sheaf);
    \item $\Hom(\mathcal{F}, \mathcal{G})$
      is an abelian group and
      composition is bilinear;
    \item finite biproducts exist;
    \item kernels and cokernels exist;
    \item the image
      coincides with the coimage.
  \end{itemize}
\end{remark}

\begin{remark}
  For the rest of this section, let
  $f : X \to Y$ be a continuous
  map of topological spaces, $\mathcal{F}$
  a sheaf on $X$, and $\mathcal{G}$ a
  sheaf on $Y$.
\end{remark}

\begin{definition}
  The \emph{pushforward} $f_* \mathcal{F}$
  is the sheaf on $Y$ defined by
  \[
    V \longmapsto \mathcal{F}(f^{-1}(V)).
  \]
\end{definition}

\begin{example}
  We have the following:
  \begin{itemize}
    \item If $i : \{p\} \hookrightarrow X$ and
    $A$ is an abelian group, then
    $i_* \underline{A}$ is the skyscraper sheaf
    on $X$ at $p$.
    \item If $X$ is a variety and
      $i : Z \hookrightarrow X$
      with $Z$ a closed subvariety, then
      \[
        i_* \OO_Z(U)
        = \{\varphi : U \cap Z \to k : \varphi \text{ is regular}\}.
      \]
  \end{itemize}
\end{example}

\begin{definition}
  Let $\widetilde{f^{-1}} \mathcal{G}(U) = \varinjlim_{V \supseteq f(U)} \mathcal{G}(V)$,
  which is a presheaf.
  The \emph{pullback} is
  $f^{-1} \mathcal{G} := (\widetilde{f^{-1}} \mathcal{G})^+$.
\end{definition}

\begin{example}
  Let $f : \{p\} \hookrightarrow X$.
  Then $\widetilde{f^{-1}} \mathcal{G} \cong \underline{\mathcal{G}_p}$,
  which is a sheaf, so
  $f^{-1} \mathcal{G} \cong \widetilde{f^{-1}} \mathcal{G} \cong \underline{\mathcal{G}_p}$.
\end{example}

\begin{example}
  If $i : U \hookrightarrow X$ is the
  inclusion of an open set, then
  $i^{-1} \mathcal{F} \cong \mathcal{F}|_U$.
\end{example}

\begin{remark}
  The pushforward $f_*$ and pullback $f^{-1}$ are functors:
  \[
    \begin{tikzcd}
      \mathrm{Sh}_X
      \ar[r, bend left, "f_*"]
      & \mathrm{Sh}_Y
      \ar[l, bend left, "f^{-1}"]
    \end{tikzcd}
  \]
  How are $f_*$ and $f^{-1}$ related?
  There are natural maps
  $f^{-1} f_* \mathcal{F} \to \mathcal{F}$ and
  $\mathcal{G} \to f_* f^{-1} \mathcal{G}$
  induced by:
  \begin{enumerate}
    \item For $U \subseteq X$, define
      \begin{align*}
        \widetilde{f^{-1}} f_* \mathcal{F}(U)
        = \varinjlim_{V \supseteq f(U) \text{ open}} \mathcal{F}(f^{-1}(V))
        &\longrightarrow \mathcal{F}(U) \\
        s &\longmapsto s|_U.
      \end{align*}
    \item For $V \subseteq Y$, define
      \[
        \mathcal{G}(V)
        \longrightarrow
        \varinjlim_{V \supseteq V' \supseteq f(f^{-1}(V)) \text{ open}} \mathcal{G}(V')
        = (f_* \widetilde{f^{-1}} \mathcal{G})(V).
      \]
      Note that $V \supseteq f(f^{-1}(V))$,
      so we can add the
      $V \supseteq V' \supseteq f(f^{-1}(V))$ condition.
  \end{enumerate}
  Another way to think about this is
  via adjoints.
\end{remark}

\begin{prop}
  For $\mathcal{F} \in \mathrm{Sh}_X$ and
  $\mathcal{G} \in \mathrm{Sh}_Y$,
  there exist functorial bijections
  \[
    \Hom_{\mathrm{Sh}_X}(f^{-1} \mathcal{G}, \mathcal{F})
    \longrightarrow
    \Hom_{\mathrm{Sh}_Y}(\mathcal{G}, f_* \mathcal{F}),
  \]
  i.e. $(f^{-1}, f_*)$ is an adjoint pair.
\end{prop}

\begin{proof}
  Given $\phi : f^{-1} \mathcal{G} \to \mathcal{F}$, using map (2) from above we get
  \[
    \begin{tikzcd}
      \mathcal{G}
      \ar[r, "(2)"]
      & f_* f^{-1} \mathcal{G}
      \ar[r, "f_* \phi"]
      & f_* \mathcal{F}.
    \end{tikzcd}
  \]
  Similarly, given
  $\psi : \mathcal{G} \to f_* \mathcal{F}$, using map (1) from above we get
  \[
    \begin{tikzcd}
      f^{-1} \mathcal{G}
      \ar[r, "f^{-1} \psi"]
      & f^{-1} f_* \mathcal{F}
      \ar[r, "(1)"]
      & \mathcal{F}.
    \end{tikzcd}
  \]
  One can (tediously) check that this
  gives a bijection.
\end{proof}

\begin{remark}
  The following are consequences of
  adjointness:
  \begin{itemize}
    \item $f_*$ is left exact, i.e.
      given an exact sequence
      \[
        \begin{tikzcd}
          0 \ar[r] & \mathcal{F}' \ar[r] & \mathcal{F} \ar[r] & \mathcal{F}'' \ar[r] & 0
        \end{tikzcd}
      \]
      in $\mathrm{Sh}_X$, we get an exact
      sequence
      \[
        \begin{tikzcd}
          0 \ar[r] & f_* \mathcal{F}' \ar[r] & f_* \mathcal{F} \ar[r] & f_* \mathcal{F}''.
        \end{tikzcd}
      \]
    \item $f^{-1}$ is right exact
      (defined similarly with the left
      $0$ missing).
  \end{itemize}
  One can also directly check these
  properties from the definitions.
\end{remark}

\section{Sheaves of \texorpdfstring{$\OO_X$}{OX}-Modules}

\begin{remark}
  To use tools from commutative algebra,
  we want to consider modules.
  For the
  rest of this section, let $X$ be a
  topological space with a sheaf of rings
  $\OO_X$ (e.g. $X$ a variety with
  $\OO_X$ the sheaf of regular functions,
  or $M$ a complex manifold with $\OO_M$
  the sheaf of holomorphic functions).
\end{remark}

\pagebreak

\begin{definition}
  A \emph{(pre)sheaf of $\OO_X$-modules}
  is a (pre)sheaf $\mathcal{F}$
  on $X$ such that for each $U \subseteq X$,
  $\mathcal{F}(U)$ has the structure of
  an $\OO_X(U)$-module compatible
  with restriction, i.e. such that
  \[
    (a \cdot s)|_V = a|_V \cdot s|_V
  \]
  for $V \subseteq U \subseteq X$ open,
  $a \in \OO_X(U)$, and $s \in \mathcal{F}(U)$.
\end{definition}

\begin{remark}
  We often say just ``$\OO_X$-module'' to mean a
  ``sheaf of $\OO_X$-modules.''
\end{remark}

\begin{definition}
  A \emph{morphism} of (pre)sheaves of $\OO_X$-modules
  $\phi : \mathcal{M} \to \mathcal{N}$
  is a morphism of presheaves such that
  $\mathcal{M}(U) \to \mathcal{N}(U)$ is a morphism of $\OO_X(U)$-modules
  for all $U \subseteq X$.
\end{definition}

\begin{example}
  We have the following:
  \begin{enumerate}
    \item $\OO_X$ has the structure of an
      $\OO_X$-module (similar to how a
      ring $A$ has the structure of an
      $A$-module).
    \item Let $X$ be a variety and
      $p : \EE \to X$ a vector bundle.
      Let $\mathcal{E}$ be the sheaf of
      sections of $p$, with
      \[
        [f \cdot s : U \to p^{-1}(U)]
        \in \mathcal{E}(U)
      \]
      as the product of
      $f \in \OO_X(U)$ and
      $[s : U \to p^{-1}(U)] \in \mathcal{E}(U)$.
      Then $\mathcal{E}$ is an $\OO_X$-module.
    \item[2'.] Let $\EE = \mathbbm{1}_X$,
      then $\mathcal{E}(U) \cong \OO_X(U)$
      as $\OO_X(U)$-modules. As the
      isomorphism is compatible
      with restriction, we have
      $\mathcal{E} \cong \OO_X$.
    \item Let $\mathcal{F}_1, \mathcal{F}_2$,
      be (pre)sheaves of $\OO_X$-modules.
      Then so
      is $\mathcal{F}_1 \oplus \mathcal{F}_2$.
    \item[2''.] If $\EE$ is a trivial
      vector bundle of rank $e$, then
      $\mathcal{E} \cong \OO_X^{\oplus e}$.
    \item If $\mathcal{F}$ is a presheaf
      of $\OO_X$-modules, then
      $\mathcal{F}^+$ is naturally
      a sheaf of $\OO_X$-modules (use the
      definition of $\mathcal{F}^+$ in the
      proof).
    \item If $\varphi : \mathcal{F} \to \mathcal{G}$
      is a morphism of sheaves of
      $\OO_X$-modules, then
      $\ker \varphi, \im \varphi, \coker \varphi$
      are sheaves of $\OO_X$-modules
      (use that $\ker \varphi, \widetilde{\im}\, \varphi, \widetilde{\coker}\, \varphi$ are presheaves of $\OO_X$-modules and then use (4)).

      Furthermore, the category
      $\mathrm{Mod}_{\OO_X}$ of sheaves of
      $\OO_X$-modules is an abelian.
    \item We can define the usual
      constructions on
      $\OO_X$-modules:
      $\otimes$, $\Sym^d$, $\wedge^d$, etc.
  \end{enumerate}
\end{example}

\begin{example}
  Let $\mathcal{F}, \mathcal{G}$ be
  $\OO_X$-modules. Then their
  tensor product $\mathcal{F} \otimes \mathcal{G}$
  is the sheafification of
  \[
    U \longmapsto \mathcal{F}(U) \otimes_{\OO_X(U)} \mathcal{G}(U).
  \]
\end{example}

\begin{definition}
  An $\OO_X$-module
  $\mathcal{F}$ is \emph{locally free}
  of rank $e$ if for any $p \in X$,
  there exists $p \in U \subseteq X$ open
  such that $\mathcal{F}|_U \cong \OO_U^{\oplus e}$.
  If $e = 1$, then we say that
  $\mathcal{F}$
\end{definition}

\begin{example}
  Let $X$ be a variety and
  $p : \EE \to X$ a vector bundle of rank $e$.
  For $p \in X$, there exists
  $p \in U \subseteq X$ open such that
  \[
    \begin{tikzcd}
      p^{-1}(U) \ar[r, "\cong"] \ar[d, "p", swap] & U \times \Affine^e \ar[dl, "\mathrm{pr}_1"]  \\
      U
    \end{tikzcd}
  \]
  Then $\mathcal{E}|_U \cong$ sheaf of
  sections of $U \times \Affine^e$
  $\cong \OO_U^{\oplus e}$.
\end{example}

\begin{remark}[Transition functions]
  Let $e = 1$ for simplicity, and
  $\mathcal{E}$ a locally free $\OO_X$-module
  of rank $e$. Then there exists an
  open cover $\{U_i\}$ of $X$ with
  isomorphisms $\alpha_i : \mathcal{E}|_{U_i} \to \OO_{U_i}^{\oplus e}$.
  So on $U_{i, j} = U_i \cap U_j$, we get
  isomorphisms
  $\alpha_{i, j} = \alpha_i \circ \alpha_j^{-1} : \OO_{U_{i, j}}^{\oplus e} \to \OO_{U_{i, j}}^{\oplus e}$.
\end{remark}

  \chapter{Jan.~29 --- \texorpdfstring{$\OO_X$}{OX}-Modules, Part 2}

\section{More on \texorpdfstring{$\OO_X$}{OX}-Modules}

\begin{remark}
  Recall that we have operations
  $\otimes$, $\oplus$,
  $\Sym^d$, $\wedge^d$,
  $\Homm(\cdot, \cdot)$ on
  $\OO_X$-modules.
\end{remark}

\begin{example}
  The sheaf $\Homm_{\OO_X}(\mathcal{F}, \mathcal{G})$
  is the sheafification of
  \[
    U \longmapsto
    \Hom_{\OO_X}(\mathcal{F}|_U, \mathcal{G}|_U).
  \]
  This is again an $\OO_X$-module.
\end{example}

\begin{example}
  We have $\mathcal{F}^\vee = \Homm_{\OO_X}(\mathcal{F}, \OO_X)$.
\end{example}

\begin{exercise}
  The invertible sheaves on $X$
  up to isomorphism forms a group
  with multiplication given by
  $\otimes$, identity $\OO_X$,
  and inverse
  $\mathcal{L}^{-1} = \mathcal{L}^\vee$.
\end{exercise}

\begin{remark}[Transition data]
  Let $\mathcal{L}$ be an invertible
  $\OO_X$-module. Then there exists an
  open cover $\{U_i\}$ of $X$ and
  isomorphisms
  $\alpha_i : \mathcal{L}|_{U_i} \to \OO_{U_i}$,
  so we get isomorphisms
  \[
    \alpha_{i, j}
    = \alpha_i \circ \alpha_j^{-1}
    : \OO_{U_{i, j}}
    \to \OO_{U_{i, j}}.
  \]
  For this to be an isomorphism,
  we must have
  $\alpha_{i, j}(U_{i, j})(1) = g_{i, j} \in \OO_X(U_{i, j})^\times$.
\end{remark}

\begin{prop}
  If $X$ is a variety, then there is a
  bijection
  \begin{align*}
    \{\text{line bundles on } X\} / {\cong}
    &\longrightarrow
    \{
      \text{invertible sheaves on } X
    \} / {\cong} \\
    \mathbb{L} &\longmapsto \mathcal{L}
  \end{align*}
\end{prop}

\begin{proof}
  To get the reverse map, fix
  an invertible $\OO_X$-module
  $\mathcal{L}$ with trivialization
  data $(U_i, g_{i, j})$. Send it
  to the line bundle with the same
  trivialization data.
  Check that this is well-defined
  as an exercise.

  To show that this gives an inverse,
  it suffices to show that if
  $\mathbb{L}$ is a line bundle
  with trivialization data
  $(U_i, g_{i, j})$, then the sheaf
  $\mathcal{L}$ of sections of
  $\mathbb{L}$ has the same
  trivialization data.
  The trivializations
  \begin{align*}
    \mathbb{L}|_{U_i}
    &\underset{\cong}{\overset{\phi_i}{\longrightarrow}}
    U_i \times \Affine^1
  \end{align*}
  give isomorphisms
  $U_{i, j} \times \Affine^1 \to U_{i, j} \times \Affine^1$
  by $(x, v) \mapsto (x, g_{i, j}(x) v)$.
  We get an isomorphism
  \begin{align*}
    \alpha_i :
    \mathbb{L}|_{U_i}
    &\underset{\cong}{\overset{\phi_i}{\longrightarrow}}
    \OO_{U_i}
  \end{align*}
  where $e_i = \alpha_i^{-1}(1) = [U_i \xrightarrow{x \mapsto (x, 1)} U_i \times \Affine^1 \xrightarrow{\phi_i^{-1}} \mathbb{L}|_{U_i}]$.
  Now we have
  \[
    \begin{tikzcd}[row sep=small]
      \OO_{U_{i, j}}
      \ar[r, "\alpha_j^{-1}"]
      & \mathcal{L}|_{U_{i, j}}
      \ar[r, "\alpha_i"]
      & \OO_{U_{i, j}} \\
      1 \ar[r, mapsto]
      & e_j = e_i g_{i, j}
      \ar[r, mapsto] & g_{i, j}.
    \end{tikzcd}
  \]
  So we get the same transition
  functions $(U_i, g_{i, j})$
  for $\mathcal{L}$.
\end{proof}

\begin{remark}
  Given a morphism of rings
  $\phi : A \to B$, we have functors
  \[
    \begin{tikzcd}
      \mathrm{Mod}_B
      \ar[r, bend left, "\Phi"]
      & \mathrm{Mod}_A
      \ar[l, bend left, "\Psi"]
    \end{tikzcd}
  \]
  given as follows:
  \begin{enumerate}
    \item \emph{Extension of scalars}:
      $\mathrm{Mod}_A \ni M \longmapsto M \otimes_A B \in \mathrm{Mod}_B$,
      where the multiplication by $B$
      is
      \[
        c (m \otimes b)
        = m \otimes (cb).
      \]
      For example, if
      $M = A^{\oplus I}$, then
      $M \otimes_A B = B^{\oplus I}$.
    \item \emph{Restriction of scalars}:
      $\mathrm{Mod}_B \ni N \longmapsto N_A \in \mathrm{Mod}_A$,
      where $N_A := N$ as
      abelian groups with
      \[
        a \cdot n = \phi(a) n
      \]
      as the multiplication by $A$.
  \end{enumerate}
\end{remark}

\begin{prop}
  There is a functorial bijection
  \begin{align*}
    \Hom_B(M \otimes_A B, N)
    &\longleftrightarrow
    \Hom_A(M, N_A) \\
    [f : M \otimes_A B \to N]
    &\longmapsto
    [m \mapsto f(m \otimes 1)] \\
    [m \otimes b \mapsto b \cdot g(m)]
    &\mathrel{\reflectbox{\ensuremath{\longmapsto}}} [g : M \to N_A]
  \end{align*}
  for $M \in \mathrm{Mod}_A$
  and $N \in \mathrm{Mod}_B$.
\end{prop}

\begin{remark}
  Given the result for rings, we
  want a similar statement for
  $\OO_X$-modules.
\end{remark}

\section{\texorpdfstring{$\OO_X$}{OX}-Modules and Continuous Maps}

\begin{definition}
  A \emph{morphism of ringed spaces}
  $(X, \OO_X) \to (Y, \OO_Y)$
  is the data of
  \begin{enumerate}
    \item a continuous map $f : X \to Y$,
    \item a morphism of sheaves
      of rings $f^\# : \OO_Y \to f_* \OO_X$.
  \end{enumerate}
\end{definition}

\begin{example}
  If $X \to Y$ is a morphism of
  varieties, then
  $\OO_Y \to f_* \OO_X$ is given
  for $U \subseteq Y$ open by
  \begin{align*}
    \OO_Y(U) &\longrightarrow
    \OO_X(f^{-1}(U)) \\
    \varphi &\longmapsto f^* \varphi.
  \end{align*}
\end{example}

\begin{remark}
  Our goal will be to define
  functors
  \[
    \begin{tikzcd}
      \mathrm{Mod}_{\OO_X}
      \ar[r, bend left, "f_*"]
      & \mathrm{Mod}_{\OO_Y}
      \ar[l, bend left, "f^{*}"]
    \end{tikzcd}
  \]
\end{remark}

\begin{remark}[Pushforward]
  Given an $\OO_X$-module $\mathcal{F}$,
  the sheaf pushforward
  $f_* \mathcal{F}$
  is naturally an $f_* \OO_X$-module.
  Via the map
  $f^\# : \OO_Y \to f_* \OO_X$,
  we get an $\OO_Y$-module structure
  on $f_* \mathcal{F}$.
  More concretely, for
  $U \subseteq Y$ open,
  $s \in f_* \mathcal{F}(U) = \mathcal{F}(f^{-1}(U))$,
  and $a \in \OO_Y(U)$, we can define
  \[
    a \cdot s
    = f^\#(U)(a) \cdot s.
  \]
\end{remark}

\begin{remark}[Pullback]
  Given an $\OO_Y$-module
  $\mathcal{G}$, we get an
  $f^{-1} \OO_Y$-module
  $f^{-1} \mathcal{G}$. By the
  adjoint property for
  $(f^{-1}, f_*)$, the morphism
  $\OO_Y \to f_* \OO_X$
  corresponds to a morphism
  $f^{-1} \OO_Y \to \OO_X$. So we get
  \[
    f^* \mathcal{G} := f^{-1} \mathcal{G} \otimes_{f^{-1} \OO_Y} \OO_X
  \]
  is an $\OO_X$-module.
  Thus we get a functor
  $f^* : \mathrm{Mod}_{\OO_Y} \to \mathrm{Mod}_{\OO_X}$.
\end{remark}

\begin{prop}
  The pair $(f^*, f_*)$ are
  adjoint functors.
\end{prop}

\begin{proof}
  Similar to before.
\end{proof}

\begin{example}\label{ex:pullbackofstructure}
  Recall that if $A \to B$ is a
  morphism of rings, then
  $A \otimes_A B \cong B$. In
  our setting, we get
  \[
    f^* \OO_Y
    = (f^{-1} \OO_Y \widetilde{\otimes}_{f^{-1} \OO_Y} \OO_X)^+
    \cong \OO_X^+ \cong \OO_X.
  \]
  Similarly, we have
  $f^*(\OO_Y^{\oplus I}) \cong (f^* \OO_Y)^{\oplus I} \cong \OO_X^{\oplus I}$
  (as left-adjoint functors
  commute with coproducts).
\end{example}

\begin{remark}
  If $\mathcal{E}$ is a locally
  free rank $m$ $\OO_X$-module,
  then $f^* \mathcal{E}$ is a
  locally free rank $m$ $\OO_Y$-module,
  (as $f^*$ can be computed locally
  on $Y$ using Example \ref{ex:pullbackofstructure}).
\end{remark}

  \chapter{Feb.~3 --- Coherent Sheaves}

\section{Review of Localization}

\begin{remark}
  Let $A$ be a ring and $S \subseteq A$
  a multiplicative system (i.e.
  $1 \in S$, and $a, b \in S$ implies
  $ab \in S$).
  For example, we could take
  $S = \langle f \rangle = (1, f, f^2, \dots)$
  for $f \in A$ or $S = A \setminus \p$
  for a prime ideal $\p \le A$.
\end{remark}

\begin{definition}
  The \emph{localization} of $A$ at $S$
  is
  \[
    S^{-1} A = \{ a / s : a \in A, s \in S\},
  \]
  where $a / s = a' / s'$
  if and only if $t(a s' - a' s) = 0$
  for some $t \in S$.
\end{definition}

\begin{remark}
  The localization satisfies the
  following universal property:
  \[
    \begin{tikzcd}
      A \ar[r, "a \mapsto a / 1"] \ar[dr, "f", swap]
      & S^{-1} A \ar[d, dashed, "\exists !"] \\
      & T
    \end{tikzcd}
  \]
  whenever $f(S)$ lands in
  the units of $T$.
\end{remark}

\begin{definition}
  For an $A$-module $M$, the
  \emph{localization} of $M$ at $S$ is
  \[
    S^{-1} M = \{ m / s : m \in M, s \in S\},
  \]
  where $m / s = m' / s'$
  if and only if
  $t(s' m - s m') = 0$
  for some $t \in S$.
\end{definition}

\begin{remark}
  For $S = \langle f \rangle$, we
  will write $S^{-1} M = M_f$.
  For $S = A \setminus \p$,
  we will write
  $S^{-1} M = M_\p$.
\end{remark}

\begin{prop}\label{prop:localization-properties}
  We have the following properties
  for localization:
  \begin{enumerate}
    \item There is an isomorphism
      \begin{align*}
        M \otimes_A S^{-1} A
        &\overset{\cong}{\longrightarrow} S^{-1} M \\
        m \otimes (a / s)
        &\longmapsto (am) / s.
      \end{align*}
    \item Localization gives an
      exact functor
      \begin{align*}
        \mathrm{Mod}_A
        &\longrightarrow \mathrm{Mod}_{S^{-1} A} \\
        M &\longmapsto S^{-1} M.
      \end{align*}
    \item A sequence in
      $\mathrm{Mod}_A$
      \[
        \begin{tikzcd}
          0 \ar[r] & M' \ar[r] & M \ar[r] & M'' \ar[r] & 0
        \end{tikzcd}
      \]
      is exact if and only if the
      sequence
      \[
        \begin{tikzcd}
          0 \ar[r] & M'_\p \ar[r] & M_\p \ar[r] & M''_\p \ar[r] & 0
        \end{tikzcd}
      \]
      is exact for all
      maximal (equivalently, prime)
      ideals $\p \le A$.
  \end{enumerate}
\end{prop}

\begin{example}
  Recall that if $X$ is an affine
  variety, then
  \[
    \OO_X(D(f)) \cong A(X)_f
    \quad \text{and} \quad
    \OO_{X, x} \cong A(X)_{\m_x},
  \]
  where $\m_x = I(\{x\}) \le A(X)$.
\end{example}

\section{Coherent Sheaves on Affine Varieties}

\begin{remark}
  For this section, let
  $X$ be an affine variety and
  $A = \OO_X(X) = A(X)$. We want a
  functor
  \[
    \mathrm{Mod}_A
    \longrightarrow \mathrm{Mod}_{\OO_X}.
  \]
\end{remark}

\begin{theorem}\label{thm:unique-OX-module}
  For $M \in \mathrm{Mod}_A$,
  there exists a unique $\OO_X$-module
  $\widetilde{M}$ such that
  \begin{enumerate}
    \item $\widetilde{M}(D(f)) \cong M_f$;
    \item for $D(g) \subseteq D(f)$,
      we have
      \[
        \begin{tikzcd}
          \widetilde{M}(D(f))
          \ar[r] \ar[d, "=", swap]
          & \widetilde{M}(D(g)) \ar[d, "="] \\
          M_f
          \ar[r, "\mathrm{natural\, map}", swap]
          & M_g
        \end{tikzcd}
      \]
  \end{enumerate}
\end{theorem}

\begin{remark}
  How is the natural map
  $M_f \to M_g$ defined? If
  $D(g) \subseteq D(f)$, then we have
  $V(g) \supseteq V(f)$, so
  $\sqrt{(g)} \subseteq \sqrt{(f)}$.
  Thus $g \in \sqrt{(f)}$, so
  $g^d = fh$ for some $d > 0$ and
  $h \in A$. So we get a map
  \begin{align*}
    M_f &\longrightarrow M_g \\
    m / f^i
    &\longmapsto 
    mh^i / g^{d i}.
  \end{align*}
  Alternatively, if $D(g) \subseteq D(f)$,
  then $D(g) = D(gf)$, so we could
  instead consider
  \[
    \begin{tikzcd}
      \widetilde{M}(D(f))
      \ar[r] \ar[d, "=",swap]
      & \widetilde{M}(D(fg)) \ar[d] \\
      M_f
      \ar[r, "m / f^i \mapsto m g^i / (fg)^i", swap]
      & M_{fg}
    \end{tikzcd}
  \]
\end{remark}

\begin{remark}
  To construct $\widetilde{M}$,
  we need the notion of
  \emph{sheaves on a basis}. Let
  $(X, \OO_X)$ be a ringed space and
  $\mathcal{P}$ a collection of
  open sets in $X$ such that
  \begin{enumerate}
    \item $\mathcal{P}$ is a
      basis for $X$;
    \item  if $U, V \in \mathcal{P}$,
      then $U \cap V \in \mathcal{P}$.
  \end{enumerate}
\end{remark}

\begin{example}
  If $(X, \OO_X)$ is an affine
  variety, we can take $\mathcal{P} = \{D(f) : f \in A(X)\}$.
  Also, if $(X, \OO_X)$ is a
  general algebraic variety, then
  we can take $\mathcal{P}$
  to be the affine open subsets of $X$ 
  (as $X$ is separated, the intersection
  of two affine open subsets is
  again affine open).
\end{example}

\begin{definition}
  A \emph{$\mathcal{P}$-sheaf of $\OO_X$-modules}
  $\mathcal{F}$
  on $X$ is the data of:
  \begin{itemize}
    \item an $\OO_X(U)$-module
      $\mathcal{F}(U)$ for
      each $U \in \mathcal{P}$,
    \item homomorphisms of abelian
      groups $\mathcal{F}(U) \to \mathcal{F}(V)$
      for all $U, V \in \mathcal{P}$
      with $V \subseteq U$
  \end{itemize}
  satisfying the following properties:
  \begin{itemize}
    \item the multiplication is
      compatible with restriction,
    \item the sheaf axiom with respect
      to open sets in $\mathcal{P}$.
  \end{itemize}
\end{definition}

\begin{example}
  If $\mathcal{F}$ is a sheaf of
  $\OO_X$-modules, then we get
  $\mathcal{F}^{\mathcal{P}}$,
  a $\mathcal{P}$-sheaf of
  $\OO_X$-modules.
\end{example}

\begin{theorem}\label{thm:P-sheaf-equivalence}
  There is an equivalence of categories
  \begin{align*}
    \mathrm{Mod}_{\OO_X}
    &\longrightarrow
    \mathrm{Mod}_{\OO_X}^{\mathcal{P}} \\
    \mathcal{F}
    &\longmapsto
    \mathcal{F}^{\mathcal{P}}.
  \end{align*}
  In particular,
  $\Hom_{\OO_X}(\mathcal{F}, \mathcal{G}) \to \Hom(\mathcal{F}^{\mathcal{P}}, \mathcal{G}^{\mathcal{P}})$
  is a bijection, and for any
  $\mathcal{H} \in \mathrm{Mod}_{\OO_X}^{\mathcal{P}}$,
  there exists some $\mathcal{F} \in \mathrm{Mod}_{\OO_X}$
  such that $\mathcal{F}^{\mathcal{P}} \cong \mathcal{H}$.
\end{theorem}

\begin{proof}
  We construct the inverse functor.
  Take
  $\mathcal{H} \in \mathrm{Mod}_{\OO_X}^{\mathcal{P}}$, and
  define $\mathcal{F} \in \mathrm{Mod}_{\OO_X}$
  by setting
  \[
    \mathcal{F}(U)
    = \{(s_V)_{V \in \mathcal{P}, V \subseteq U} : s_V|_{V \cap V'} = s_V'|_{V \cap V'} \text{ for all } V, V' \in \mathcal{P} \text{ with } V, V' \subseteq U\}.
  \]
  One can check that this defines a
  functor and an equivalence of categories.
\end{proof}

\begin{remark}
  Returning to algebraic geometry,
  let $X$ be an affine variety,
  $A = A(X)$, and $M$ an $A$-module.
\end{remark}

\begin{prop}\label{prop:P-sheaf-from-module}
  For $\mathcal{P} = \{D(f) : f \in A\}$,
  the assignment $D(f) \mapsto M_f$
  with the natural restriction maps
  defines a $\mathcal{P}$-sheaf of $\OO_X$-modules.
\end{prop}

\begin{proof}
  See Mustata Lemma 8.3.2.
  The hard part is to check the
  sheaf axiom for $\mathcal{P}$, which
  is similar to the computation
  that $\OO_X(D(f)) \cong A_f$
  for an affine variety $X$.
\end{proof}

\begin{proof}[Proof of Theorem \ref{thm:unique-OX-module}]
  Combining
  Theorem \ref{thm:P-sheaf-equivalence}
  and Proposition
  \ref{thm:P-sheaf-equivalence},
  we get an $\OO_X$-module
  $\widetilde{M}$ such that
  $\widetilde{M}(D(f)) \cong M_f$,
  and it is unique up to isomorphism.
\end{proof}

\begin{example}
  We have
  $\widetilde{A} \cong \OO_X$, since
  \[
    \widetilde{A}(D(f))
    \cong A_f \cong \OO_X(D(f)).
  \]
  Similarly, one can check that
  $\widetilde{A^{\oplus I}} \cong \widetilde{\OO_X^{\oplus I}}$.
\end{example}

\begin{exercise}
  For $Z \subseteq X$ closed and
  $I(Z) \le A(X)$, we have
  $\widetilde{I(Z)} \cong \mathcal{I}_Z$ (the ideal sheaf of $Z$).
\end{exercise}

\begin{remark}
  We have the following:
  \begin{enumerate}
    \item For $x \in X$ with
      $\p := I(\{x\}) \le A$, we have
      $\widetilde{M}_x \cong \varinjlim_{D(f) \ni x} \widetilde{M}(D(f)) = \varinjlim_{f \in A \setminus \p} M_f \cong M_\p$.
    \item For an $A$-module
      homomorphism $\varphi : M \to N$,
      we get homomorphisms
      \[
        \begin{tikzcd}
          \widetilde{M}(D(f)) \ar[r]
          \ar[d, "\cong", swap]
          & \widetilde{N}(D(f)) \ar[d, "\cong"] \\
          M_f \ar[r, "\mathrm{natural\, map}", swap]
          & N_f
        \end{tikzcd}
      \]
      which are
      compatible with restriction.
      So we get a homomorphism
      of $\OO_X$-modules
      $\widetilde{M} \to \widetilde{N}$.
      One can check that this gives a
      functor
      $\Phi : \mathrm{Mod}_A \to \mathrm{Mod}_{\OO_X}$.
  \end{enumerate}
\end{remark}

\begin{prop}\label{prop:properties-of-tilde-functor}
  We have the following:
  \begin{enumerate}
    \item $\Phi$ is exact;
    \item $\Phi$ is fully
      faithful, i.e. the following map
      is a bijection:
      \begin{align*}
        \Hom_A(M, N)
        &\longrightarrow
        \Hom_{\OO_X}(\widetilde{M}, \widetilde{N}) \\
        \varphi &\longmapsto \widetilde{\varphi}.
      \end{align*}
  \end{enumerate}
\end{prop}

\begin{proof}
  (1)  If $0 \to M' \to M \to M'' \to 0$
  is exact, then
  $0 \to M'_\p \to M_\p \to M''_\p \to 0$
  is exact for every
  $\p \le A$ maximal by
  Proposition \ref{prop:localization-properties}(3).
  So the sequence
  \[
    \begin{tikzcd}
      0 \ar[r] & \widetilde{M}'_x
      \ar[r] & \widetilde{M}_x
      \ar[r] & \widetilde{M}''_x
      \ar[r] & 0
    \end{tikzcd}
  \]
  is exact for all
  $x \in X$. So
  $0 \to \widetilde{M}' \to \widetilde{M} \to \widetilde{M}'' \to 0$
  is exact.

  (2) We want a map
  $\Hom_{\OO_X}(\widetilde{M}, \widetilde{N}) \to \Hom_A(M, N)$.
  Given an $\OO_X$-module
  homomorphism
  $\varphi : \widetilde{M} \to \widetilde{N}$,
  we get
  $f = \varphi(X) : \widetilde{M}(X) = M \to \widetilde{N}(X) = N$.
  We want to show that
  $\widetilde{f} = \varphi$. We have
  \[
    \begin{tikzcd}
      \widetilde{M}(X)
      \ar[r, "\varphi(X)"] \ar[d]
      & \widetilde{N}(X) \ar[d] \\
      \widetilde{M}(D(g))
      \ar[r, "\varphi(D(f))", swap]
      & \widetilde{N}(D(g))
    \end{tikzcd}
  \]
  and this diagram commutes.
  So we get $\widetilde{f}(D(g)) = \varphi(D(g))$.
\end{proof}

\begin{remark}
  By
  Proposition \ref{prop:properties-of-tilde-functor}(1),
  given an $A$-module
  homomorphism $f : M \to N$, we have
  \[
    \ker(\widetilde{f})
    \cong \widetilde{\ker(f)},
    \quad \coker(\widetilde{f})
    \cong \widetilde{\coker(f)},
    \quad \im(\widetilde{f})
    \cong \widetilde{\im(f)}.
  \]
\end{remark}

\begin{prop}
  For $M, N \in \mathrm{Mod}_A$,
  we have
  \begin{enumerate}
    \item $\widetilde{M} \otimes_{\OO_X} \widetilde{N}
      \cong \widetilde{M \otimes_A N}$;
    \item $\widetilde{\Hom_A(M, N)}
      \cong \Homm_{\OO_X}(\widetilde{M}, \widetilde{N})$;
    \item $\widetilde{\bigoplus_{i \in I} M_i}
      \cong \bigoplus_{i \in I} \widetilde{M}_i$.
  \end{enumerate}
\end{prop}

\begin{proof}
  (1) We have a homomorphism
  \[
    \begin{tikzcd}
    M \otimes_A N
    \ar[r, "\cong"]
    &
    \Gamma(X, \widetilde{M}\, \widetilde{\otimes}_{\OO_X}\, \widetilde{N}) \ar[d] \\
    & \Gamma(X, \widetilde{M} \otimes_{\OO_X} \widetilde{N})
    \end{tikzcd}
  \]
  This gives a homomorphism
  of $\OO_X$-modules
  $\widetilde{M \otimes_A N} \to \widetilde{M} \otimes_{\OO_X} \widetilde{N}$.
  Now at the stalks for
  $x \in X$ with $\m = I(\{x\}) \le A(X)$,
  we can see that
  \[
    (\widetilde{M \otimes_A N})_x
    \cong (M \otimes_A N) \otimes_A A_\m
    \cong M_\m \otimes_{A_\m} N_\m.
  \]
  Similarly, we have
  \[
    (\widetilde{M} \otimes_{\OO_X} \widetilde{N})_x
    \cong \widetilde{M}_x \otimes_{\OO_{X, x}} \widetilde{N}_x
    \cong M_\m \otimes_{A_\m} N_\m.
  \]
  Thus we have isomorphisms
  at the stalks.
\end{proof}

\begin{remark}
  The functor $\Phi : \mathrm{Mod}_A \to \mathrm{Mod}_{\OO_X}$
  is left adjoint to
  $\mathrm{Mod}_{\OO_X} \to \mathrm{Mod}_A$
  given by $\mathcal{F} \mapsto \mathcal{F}(X)$.
\end{remark}

  \chapter{Feb.~5 --- Coherent Sheaves, Part 2}

\section{Coherent Sheaves on Affine Varieties, Continued}

\begin{prop}
  Let $\varphi : X \to Y$ be a morphism
  of affine varieties, and let
  \[
    A = \OO_Y(Y)
    \overset{\varphi^\#}{\longrightarrow}
    \OO_X(X) = B
  \]
  be the corresponding
  ring homomorphism on the coordinate
  rings. Then:
  \begin{enumerate}
    \item For $N \in \mathrm{Mod}_B$,
      we have $\varphi_* \widetilde{N} = \widetilde{N_A}$.
    \item For $M \in \mathrm{Mod}_A$,
      we have $\varphi^* \widetilde{M} = \widetilde{M \otimes_A B}$.
  \end{enumerate}
\end{prop}

\begin{proof}
  (1) For $f \in A = \OO_Y(Y)$,
  the left-hand side is given on $D(f)$
  by
  \[
    (\varphi_* \widetilde{N})(D(f))
    = \widetilde{N}(\varphi^{-1}(D(f)))
    = \widetilde{N}(D(\varphi^\# f))
    = N_{\varphi^\# f}
    = (N_A)_f.
  \]
  So we get that
  $\varphi_* \widetilde{N} = \widetilde{N_A}$.

  (2) (Hartshorne says this holds
  by definition lol.)
  First assume $M = A^{\oplus I}$. Then
  \[
    \varphi^* \widetilde{M}
    = \varphi^*(\OO_Y^{\oplus I})
    = (\varphi^* \OO_Y)^{\oplus I}
    = \OO_X^{\oplus I}
    = \widetilde{B^{\oplus I}}
    = \widetilde{M \otimes_A B}.
  \]
  For an arbitrary $A$-module $M$,
  we use the following:
  \begin{quote}
    \textbf{Claim:} There exists an
    exact sequence
    \[
      \begin{tikzcd}
        A^{\oplus J} \ar[r, "\alpha"]
        & A^{\oplus I} \ar[r, "\beta"]
        & M \ar[r]
        & 0
      \end{tikzcd}
    \]

    \begin{proof}[Proof of claim]
      Choose generators $(m_i)_{i \in I}$ for
      $M$, and set
      $\beta(e_i) = m_i$. Choose
      generators $(n_j)_{j \in J}$
      for $\ker \beta$. Then we can set
      $\alpha(f_j) = n_j$.
    \end{proof}
  \end{quote}
  Apply $\cdot \otimes_A B$ (which is
  right exact)
  to the
  exact sequence from the claim to
  get
  \[
    \begin{tikzcd}
      B^{\oplus J} \ar[r]
      & B^{\oplus I} \ar[r]
      & M \otimes_A B \ar[r]
      & 0.
    \end{tikzcd}
  \]
  Applying $\varphi^* (\, \widetilde{\cdot}\,)$
  (note that $\widetilde{\cdot}$
  is exact and $\varphi^*$ is right exact)
  to the original sequence to get
  \[
    \begin{tikzcd}[row sep=small]
      \varphi^*(\widetilde{A^{\oplus J}}) \ar[r]
      & \varphi^*(\widetilde{A^{\oplus I}}) \ar[r]
      & \varphi^* \widetilde{M} \ar[r]
      & 0 \\
      B^{\oplus J} \ar[u, equal]
      & B^{\oplus I} \ar[u, equal]
    \end{tikzcd}
  \]
  using the previous case.
  Thus $\varphi^* \widetilde{M} = \coker(\widetilde{B^{\oplus J}} \to \widetilde{B^{\oplus I}}) = (\coker(B^{\oplus J} \to B^{\oplus I}))^{\sim} = \widetilde{M \otimes_A B}$.
\end{proof}

\section{Quasicoherent and Coherent Sheaves}
\begin{remark}
  For the rest of this lecture,
  assume $(X, \OO_X)$ is a variety
  (not necessarily affine).
\end{remark}

\begin{definition}
  An $\OO_X$-module $\mathcal{F}$
  is \emph{quasicoherent}
  if there exists an affine cover
  $X = \bigcup_{i \in I} U_i$
  such that $\mathcal{F}|_{U_i} \cong \widetilde{M}_i$
  for some $\OO_X(U_i)$-module $M_i$.
  It is \emph{coherent} if the
  $M_i$ are finitely generated
  $\OO_X(U_i)$-modules.
\end{definition}

\begin{remark}
  We have $M_i \cong \widetilde{M}_i(U_i) \cong \mathcal{F}(U_i)$,
  so we may replace the
  $\mathcal{F}|_{U_i} \cong \widetilde{M}_i$
  condition with one of the
  following:
  $\mathcal{F}|_{U_i} \cong \widetilde{\mathcal{F}(U_i)}$,
  or $\mathcal{F}(U_i)_f \to \mathcal{F}(D_{U_i}(f))$
  is an isomorphism for
  all $f \in \OO_X(U_i)$.
\end{remark}

\begin{example}
  We have the following:
  \begin{enumerate}
    \item If $\mathcal{E}$ is a
      locally free sheaf of rank $r$ 
      on $X$, then there exists an
      open cover $X = \bigcup U_i$
      such that $\mathcal{E}|_{U_i} \cong \OO_X^{\oplus r}|_{U_i}$.
      Refining the cover, we may
      assume the $U_i$ are affine,
      so
      \[
        \mathcal{E}|_{U_i}
        = \widetilde{\OO_X(U_i)^{\oplus r}}.
      \]
      Thus we see that
      $\mathcal{E}$ is coherent.
    \item For $Z \hookrightarrow X$
      a closed embedding,
      $\mathcal{I}_Z \subseteq \OO_X$
      is coherent.
    \item For $\mathcal{F}$ a
      (quasi)coherent $\OO_X$-module
      and $U \subseteq X$ open,
      $\mathcal{F}|_U$ is
      (quasi)coherent.

      To see this, use that if
      $\mathcal{F}|_{U_i} \cong \widetilde{M}_i$
      for $M_i$ an $\OO_X(U_i)$-module
      and $f \in \OO_X(U_i)$, then
      \[\mathcal{F}|_{D_{U_i}(f)} \cong \widetilde{(M_{i})_f}.\]
      Furthermore, if $M_i$
      is finitely generated, then so
      is $(M_i)_f$. So by
      refining our open affine cover in
      with principal
      opens (which form a basis)
      we may assume $U = \bigcup_{i, U_i \subseteq U} U_i$.
      This gives the result.
  \end{enumerate}
\end{example}

\begin{prop}[Key proposition]\label{prop:qc-coh-equiv}
  Let $\mathcal{F} \in \mathrm{Mod}_{\OO_X}$.
  The following are equivalent:
  \begin{enumerate}
    \item $\mathcal{F}$ is
      quasicoherent (resp. coherent).
    \item For any affine open set
      $U \subseteq X$, we have
      $\mathcal{F}|_U \cong \widetilde{\mathcal{F}(U)}$
      (with $\mathcal{F}(U)$ finitely
      generated in the coherent case).
  \end{enumerate}
\end{prop}

\begin{lemma}[Affine communication]\label{lem:affine-communication}
  If $X$ is a variety,
  $U, V \subseteq X$ affine open
  subsets, and $p \in U \cap V$,
  then there exists an open set
  $p \in W \in U \cap V$ that is a
  principal open of both $U$ and $V$.
\end{lemma}

\begin{proof}
  Choose a principal open of $U$
  with $p \in W_1 = D_U(h) \subseteq U \cap V$
  for some $h \in \OO_X(U)$,
  and choose a principal open
  of $V$ with $p \in W = D_V(g) \subseteq W_1$
  for some $g \in \OO_X(V)$. Now
  \[
    g|_{W_1}
    = \OO_X(W_1)
    = \OO_X(U)_h,
  \]
  so $g|_{W_1} = f / h^i$ for some
  $f \in \OO_X(U)$ and $i \ge 0$. Now
  $D_V(g) = W = D_{W_1}(g|_{W_1}) = D_U(f h)$.
\end{proof}

\begin{proof}[Proof of Proposition \ref{prop:qc-coh-equiv}]
  $(2 \Rightarrow 1)$ This is clear.

  $(1 \Rightarrow 2)$ Assume
  $\mathcal{F} \in \mathrm{QCoh}_X$.
  So there exists an open affine
  cover $\{U_i\}$ such that
  $\mathcal{F}|_{U_i} = \widetilde{\mathcal{F}(U_i)}$.
  Fix $U \subseteq X$ affine open. By
  refining the cover, we may assume
  that $U = \bigcup_{U_i \subseteq U} U_i$.
  Now replacing $X$ with $U$, we may
  assume that $X = U$. Using
  Lemma \ref{lem:affine-communication},
  we may assume $U_i = D(f_i)$
  for some $f_i \in \OO_X(X)$.

  So now
  $X$ is affine, $X = \bigcup_{i = 1}^r D(f_i)$,
  and $\mathcal{F}|_{D(f_i)} \cong \widetilde{\mathcal{F}(D(f_i))}$.
  We want to show that
  for any $f \in A$, the natural
  map $\mathcal{F}(X)_f \to \mathcal{F}(D(f))$
  is an isomorphism (this would imply
  $\mathcal{F} \cong \widetilde{\mathcal{F}(X)}$). Now
  \[
    \begin{tikzcd}
      0 \ar[r]
      & \mathcal{F}(X)_f \ar[r] \ar[d, "\alpha"]
      & \bigoplus_i \mathcal{F}(D(f_i))_f \ar[r] \ar[d, "\beta"]
      & \bigoplus_{i, j} \mathcal{F}(D(f_i f_j))_f \ar[d, "\gamma"] \\
      0 \ar[r]
      & \mathcal{F}(D(f)) \ar[r]
      & \bigoplus_i \mathcal{F}(D(f_i f))
      \ar[r]
      & \bigoplus_{i, j} \mathcal{F}(D(f_i f_j f))
    \end{tikzcd}
  \]
  where the first row is exact by
  the sheaf property and using that
  localization is exact, and the
  second row is exact by the sheaf
  property. Note that
  $\beta$ and $\gamma$
  are isomorphisms as
  \[
    \mathcal{F}|_{D(f_i)}
    \cong \widetilde{\mathcal{F}(D(f_i))}
    \quad \text{and} \quad
    \mathcal{F}|_{D(f_i f_j)}
    \cong \widetilde{\mathcal{F}(D(f_i f_j))}.
  \]
  So by the five lemma,
  $\alpha$ is an isomorphism, and
  thus $\mathcal{F} \cong \widetilde{\mathcal{F}(X)}$.
\end{proof}

\begin{remark}
  For the coherent case, use the
  fact that if $M$ is an $A$-module,
  $A = (f_1, \dots, f_r)$,
  and $M_{f_i}$ is finitely
  generated for $i = 1, \dots, r$, then
  $M$ is finitely generated.
\end{remark}

\begin{prop}
  We have the following:
  \begin{enumerate}
    \item If $\varphi : \mathcal{F} \to \mathcal{G}$
      is a morphism of
      (quasi)coherent sheaves, then
      $\ker \varphi,
      \im \varphi,
      \coker \varphi$
      are (quasi)coherent.
    \item If $\mathcal{F}$ and
      $\mathcal{G}$ are
      (quasi)coherent, then so
      are $\mathcal{F} \otimes \mathcal{G}$
      and $\Homm(\mathcal{F}, \mathcal{G})$.
    \item If $\mathcal{F}_i$ is
      quasicoherent for
      $i \in I$, then
      $\bigoplus_{i \in I} \mathcal{F}_i$
      is quasicoherent.
      Furthermore, if $\mathcal{F}_i$
      is coherent and $|I| < \infty$,
      then $\bigoplus_{i \in I} \mathcal{F}_i$
      is coherent.
  \end{enumerate}
\end{prop}

\begin{proof}
  (1) Choose $U \subseteq X$ affine
  open. So we can write
  $\mathcal{F}|_U = \widetilde{M}$
  and $\mathcal{G}|_U = \widetilde{N}$.
  Furthermore, $\varphi|_U = \widetilde{\alpha}$
  for some $\OO_X(U)$-module
  homomorphism $\alpha : M \to N$.
  Now
  \[
    (\ker \varphi)|_U
    = \ker \varphi|_U
    = \ker \widetilde{\alpha}
    = \widetilde{\ker \alpha}.
  \]
  Furthermore, if $\mathcal{F}$ and
  $\mathcal{G}$ are coherent, then
  $M$ and $N$ are finitely generated,
  so $\ker \alpha$ is also
  finitely generated.
  One can show the same for
  $\im \varphi$ and
  $\coker \varphi$ similarly.
\end{proof}

\begin{remark}
  We have full subcategories
  $\mathrm{Coh}_X \subseteq \mathrm{QCoh}_X \subseteq \mathrm{Mod}_{\OO_X}$,
  which are abelian.
\end{remark}

  \chapter{Feb.~10 --- Coherent Sheaves, Part 3}

\section{Quasicoherent Sheaves, Continued}

\begin{prop}\label{prop:pushforward-pullback}
  Let $f : X \to Y$ be a morphism of
  varieties. Then
  \begin{enumerate}
    \item If $\mathcal{G} \in \mathrm{QCoh}_Y$
      (resp. $\mathrm{Coh}_Y$), then $f^* \mathcal{G} \in \mathrm{QCoh}_X$
      (resp. $\mathrm{Coh}_X$).
    \item If $\mathcal{F} \in \mathrm{QCoh}_X$, then $f_* \mathcal{F} \in \mathrm{QCoh}_Y$.
  \end{enumerate}
\end{prop}

\begin{proof}
  (1) Fix $\mathcal{G} \in \mathrm{QCoh}_Y$
  and $x \in X$. There exist affine opens
  $x \in U \subseteq X$ and $f(x) \in V \subseteq Y$
  such that $f(U) \subseteq V$.
  Write $g = f|_U : U \to V$,
  then $(f^* \mathcal{G})|_U = g^* (\mathcal{G}|_V)$.
  Now letting $A = \OO_Y(V)$, $B = \OO_X(U)$,
  $M = \mathcal{G}(V)$,
  we see that
  \[
    f^* \mathcal{G}|_U
    = g^*(\mathcal{G}|_V)
    = g^* \widetilde{M}
    = \widetilde{M \otimes_A B}.
  \]
  So $f^* \mathcal{G}$ is quasicoherent.
  The coherent version follows from
  the fact that if $M$ is a finitely
  generated $A$-module, then
  $M \otimes_A B$ is a finitely generated
  $B$-module.

  (2) Fix $\mathcal{F} \in \mathrm{QCoh}_X$.
  We can check quasicoherence locally,
  so we can reduce to the case where
  $Y$ is affine (cover $Y$ by affine opens
  $U_i$ and replace $f$ with $f^{-1}(U_i) \to U_i$).
  Choose an affine cover $X = U_1 \cup \dots \cup U_r$,
  and note that $U_i \cap U_j$ is again
  affine for a variety.
  Write $\alpha_i : U_i \hookrightarrow X$
  and $\alpha_{i, j} : U_{i, j} \hookrightarrow X$.
  As $\mathcal{F}$ is a sheaf,
  we have an exact sequence
  \[
    \begin{tikzcd}
      0 \ar[r] & \mathcal{F}
      \ar[r] & \bigoplus_i (\alpha_i)_* \mathcal{F}|_{U_i}
      \ar[r] & \bigoplus_{i, j} (\alpha_{i, j})_* \mathcal{F}|_{U_{i, j}}.
    \end{tikzcd}
  \]
  Applying $f_*$, which is left exact,
  we get an exact sequence
  \[
    \begin{tikzcd}
      0 \ar[r] & f_* \mathcal{F}
      \ar[r] & \bigoplus_i(f \circ \alpha_i)_* \mathcal{F}|_{U_i}
      \ar[r] & \bigoplus_{i, j} (f \circ \alpha_{i, j})_* \mathcal{F}|_{U_{i, j}}.
    \end{tikzcd}
  \]
  Note that the last two terms
  are both quasicoherent (e.g.
  $\mathcal{F}|_{U_i}$ is
  quasicoherent and $f \circ \alpha_i$
  is a morphism of affine varieties, so
  $(f \circ \alpha_i)_* \mathcal{F}|_{U_i}$
  is quasicoherent).\footnote{If $f : X \to Y$ is a morphism of affine varieties, and $A := \OO_Y(Y)$, $B := \OO_X(X)$, then $f_* \widetilde{N} = \widetilde{N_A}$ for
  $N \in \mathrm{Mod}_B$ and $f^* \widetilde{M}$ for $M \in \mathrm{Mod}_A$. This shows that pushforwards and pullbacks of quasicoherent sheaves on \emph{affine} varieties are again quasicoherent.}
\end{proof}

\begin{remark}
  The coherent version of Proposition
  \ref{prop:pushforward-pullback}(2) fails:
  For
  \[i : \Affine^1 \setminus \{0\} \longhookrightarrow \Affine^1,\]
  the
  pushforward
  $i_* \OO_{\Affine^1 \setminus \{0\}}$
  is not coherent (but $\OO_{\Affine^1 \setminus \{0\}}$ is coherent).
\end{remark}

\begin{remark}
  If $X \to Y$ is \emph{projective},
  i.e. there exists a factorization
  \[
    \begin{tikzcd}
      X \ar[r, hook] \ar[dr] & Y \times \PP^n
      \ar[d, "\mathrm{pr}_1"] \\
                      & Y
    \end{tikzcd}
  \]
  then we do get a coherent version for
  the pushfoward $f_* \mathcal{F}$.
  One can prove this via
  sheaf cohomology.
\end{remark}

\section{Morphisms to Projective Space}

\begin{remark}
  Recall that there is a bijection
  \[
    \{\LL \to X \text{ with } s_0, \dots, s_n \in \Gamma(X, \LL) \text{ nowhere vanishing}\} / {\cong}
    \longleftrightarrow
    \{\text{morphisms } X \to \PP^n\}.
  \]
  We want to rephrase this using the
  bijection
  \[
    \{\mathcal{L} \text{ invertible sheaves of } \OO_X\text{-modules}\}
    \longleftrightarrow
    \{
      \text{line bundles } \LL \to X
    \}
  \]
  and determine when $X \to \PP^n$ is
  injective or a closed embedding.
\end{remark}

\begin{remark}
  Let $X$ be a variety and $\mathcal{L}$
  an invertible $\OO_X$-module.
  For $x \in X$, we have
  \[
    \mathcal{L}(x)
    := \mathcal{L}_x / \m_x \mathcal{L}_x
    \cong \OO_{X, x} / \m_x
    \cong k,
  \]
  where $\m_x \subseteq \OO_{X, x}$ is a
  maximal ideal.
\end{remark}

\begin{definition}
  We say $s_0, \dots, s_n \in \Gamma(X, \mathcal{L})$
  \emph{generate} $\mathcal{L}$ if
  \[
    s_0(x), \dots, s_n(x)
    \in \mathcal{L}(x) \cong k
  \]
  generate $\mathcal{L}(x)$
  as a $k$-vector space
  (i.e. at least one of $s_0(x), \dots, s_n(x)$ is nonzero)
  for all $x \in X$.
\end{definition}

\begin{prop}
  The following are equivalent:
  \begin{enumerate}
    \item $s_0, \dots, s_n \in \Gamma(X, \mathcal{L})$
      generate $\mathcal{L}$;
    \item the morphism
      $\varphi : \OO_X^{\oplus (n + 1)} \to \mathcal{L}$
      given by $e_i \mapsto s_i$ is
      surjective;
    \item for every $U \subseteq X$ open
      affine,
      $s_0|_U, \dots, s_n|_U$ generate
      $\mathcal{L}(U)$ as an
      $\OO_X(U)$-module.
  \end{enumerate}
\end{prop}

\begin{proof}
  By commutative algebra (in particular
  Nakayama's lemma),
  $s_0(x), \dots, s_n(x)$ span
  $\mathcal{L}(x)$ (1) if and only if
  $(s_0)_x, \dots, (s_n)_x$ generate $\mathcal{L}_x$ as an $\OO_{X, x}$-module.
  This happens if and only if
  $\varphi_x$ is surjective, which happens
  if and only if $\varphi$ is surjective (2).
  This happens if and only if
  $\varphi(U)$ is surjective for all
  $U \subseteq X$ open affine (3)
  (note that $\coker(\widetilde{M} \to \widetilde{N} = (\coker(M \to N))^{\sim}$).
\end{proof}

  \chapter{Feb.~12 --- Morphisms to Projective Space}

\section{Morphisms to Projective Space, Continued}

\begin{definition}
  When any of the conditions
  in Proposition \ref{prop:generate} hold, we say that
  $\mathcal{L}$ is \emph{generated}
  by $s_0, \dots, s_n$. Furthermore,
  we say $\mathcal{L}$
  is \emph{globally generated} if
  there exist $s_0, \dots, s_n$ generating
  $\mathcal{L}$.
\end{definition}

\begin{example}
  $\OO_X$ is globally generated,
  e.g. by $1 \in \Gamma(X, \OO_X)$.
\end{example}

\begin{example}
  For $X = \PP^n$, the invertible sheaf
  $\OO_{\PP^n}(1)$
  on $\PP^n$ defined by transition data
  $(U_i, x_j / x_i)$
  is globally generated. There are
  isomorphisms
  $\alpha_{i} : \OO_{\PP^n}(1)|_{U_i} \to \OO_{U_i}$
  such that $\alpha_i \circ \alpha_j^{-1} =$
  multiplication by $x_j / x_i$.
  Note that we have an isomorphism
  \begin{align*}
    k[x_0, \dots, x_n]_1
    &\longrightarrow \Gamma(\PP^n, \OO_{\PP^n}(1)) \\
    f &\longmapsto s_f,
  \end{align*}
  where $\alpha_i(s_f|_{U_i}) = f / x_i$
  (check that this is well-defined).

  Then $x_0, \dots, x_n \in \Gamma(\PP^n, \OO_{\PP^n}(1))$
  generate $\OO_{\PP^n}(1)$.
  To see this, note that on $U_i$, we have
  \[
    \alpha_i(x_i) = 1 \in \OO_{U_i},
  \]
  so $x_i|_{U_i}$ generates
  $\OO_{\PP^n}(1)(U_i)$ as
  an $\OO_{\PP^n}(U_i)$-module
  by Proposition \ref{prop:generate}(4).
  So the $x_0, \dots, x_n$
  generate $\OO_{\PP^n}(1)$. Alternatively,
  one can note that for
  $a = [a_0 : \cdots : a_n] \in \PP^n$,
  we have $[x_0(a) : \cdots : x_n(a)] = [a_0 : \cdots : a_n]$,
  so the $x_i$ generate $\OO_{\PP^n}(1)$.
\end{example}

\begin{example}
  Let $f : X \to \PP^n$ be a morphism.
  Then $\mathcal{L} = f^* \OO_{\PP^n}(1)$
  is generated by
  \[f^* x_0, \dots, f^* x_n \in \Gamma(X, \mathcal{L}).\]
\end{example}

\begin{remark}
  If $\mathcal{L}$ is generated by
  $s_0, \dots, s_n$, then we get a map
  \begin{align*}
    X &\longrightarrow \PP^n \\
    x &\longmapsto [s_0(x) : \cdots : s_n(x)].
  \end{align*}
  This is a morphism: If we fix $a \in X$ 
  and pick $a \in U \subseteq X$ open such
  that
  \[
    \alpha : \mathcal{L}|_U \overset{\cong}{\longrightarrow} \OO_U,
  \]
  then $f|_U(x) = [\alpha(s_0|_U)(x) : \cdots : \alpha(s_n|_U)(x)]$.
  Since each
  $\alpha(s_i)$ is a regular function,
  $f|_U$ is a morphism.
  Thus we see that $f$ is a morphism.
\end{remark}

\begin{remark}
  Similar to before, we have a bijection
  \[
    \{\text{morphisms } X \to \PP^n\}
    \longleftrightarrow
    \{\text{invertible sheaves $\mathcal{L}$ on $X$ with } s_0, \dots, s_n \in \Gamma(X, \mathcal{L} \text{ generators}\} / {\cong}
  \]
  The maps are given by
  $f \mapsto \mathcal{L} = f^* \OO_{\PP^n}(1)$
  with $s_i = f^* x_i$ with
  inverse $(\mathcal{L}, s_i) \mapsto [x \mapsto [s(x)]]$.
\end{remark}

\begin{example}
  Recall the \emph{Veronese embedding}
  given by
  \begin{align*}
    \PP^n &\longrightarrow \PP^{Nd} \\
    a &\longmapsto [x^I(a) : x^I \text{ monomial of degree $d$ in $x_0, \dots, x_n$}].
  \end{align*}
  This map corresponds to
  $\OO_{\PP^n}(d)$ with sections
  $x^I \in \Gamma(\PP^n, \OO_{\PP^n}(d))$.
\end{example}

\section{Injectivity and Closed Embeddings}

\begin{remark}
  When is $f : X \to \PP^n$ injective or a
  closed embedding?
\end{remark}

\begin{definition}
  Let $V = \Span\{s_0, \dots, s_n\} \subseteq \Gamma(X, \mathcal{L})$
  such that $s_0, \dots, s_n$ generate
  $\mathcal{L}$. We say that
  \begin{enumerate}
    \item $s_0, \dots, s_n$
      \emph{separate points} of $X$ if
      for all $x \ne y$, there exists
      $s \in V$ such that $s(x) = 0 \ne s(y)$.\footnote{Note that $s(x) = 0$ if and only if $s \in \m_p \mathcal{L}_p$.}
    \item $s_0, \dots, s_n$
      \emph{separate tangent vectors}
      if for each $p \in X$,
      \[
        \{s_p : s \in V \text{ and } s(p) = 0\}
      \]
      generates $\m_p \mathcal{L}_p / \m_p^2 \mathcal{L}_p$ as a $k$-vector space.\footnote{Here $\m_p \le \OO_{X, p}$ is its unique maximal ideal. Recall that $(T_p X)^* \cong \m_p / \m_p^2 \cong \m_p \mathcal{L}_p / \m_p^2 \mathcal{L}_p$.}
  \end{enumerate}
\end{definition}

\begin{prop}
  Assume that $X$ is complete and
  $\mathcal{L}$ an invertible sheaf on
  $X$ generated by $s_0, \dots, s_n \in \Gamma(X, Y)$.
  Write
  \begin{align*}
    f : X &\longrightarrow \PP^n \\
    x &\longmapsto [s_0(x) : \cdots : s_n(x)].
  \end{align*}
  Then we have the following:
  \begin{enumerate}
    \item $f$ is injective if and only if
      $s_0, \dots, s_n$ separate points;
    \item $f$ is a closed embedding
      if and only if
      $s_0, \dots, s_n$ separate points and
      tangent vectors.
  \end{enumerate}
\end{prop}

\begin{proof}
  (1) For $s = \sum_{i = 0}^n a_i s_i$
  with $a_i \in k$, we have
  \[
    \{x \in X : s(x) = 0\}
    = f^{-1}(H_a)
  \]
  where $H_a = V(\sum a_i x_i)$. Now
  $f$ is injective if and only if
  for all $x \ne y \in X$, there exists
  a hyperplane $H \subseteq \PP^n$ such that
  $f(x) \in H$ and $f(y) \not\in H$. This
  happens if and only if
  $s_0, \dots, s_n$ separate points.

  (2) Assume that $f$ is injective 
  (equivalently, $s_0, \dots, s_n$ separate points by (1)).
  For $p \in X$, we get a local ring
  homomorphism
  $\OO_{\PP^n, f(p)} \to \OO_{X, p}$,
  which induces a map on cotangent spaces
  \[
    \begin{tikzcd}
      \m_{f(p)} / \m_{f(p)}^2
      \arrow[r] \arrow[d, "\cong", swap] &
      \m_p / \m_p^2 \arrow[d, "\cong"] \\
      \m_{f(p)} \OO_{\PP^n}(1)_{f(p)} / \m^2_{f(p)} \OO_{\PP^n}(1)_{f(p)} &
      \m_p \mathcal{L}_p / \m_p^2 \mathcal{L}_p
    \end{tikzcd}
  \]
  (as $\mathcal{L}$ is invertible, it is
  locally trivial), where the bottom map
  is given by
  \[
    \left[\sum {a_i x_i}\right]
    \longmapsto \left[\sum a_i s_i\right].
  \]
  So we get that $f$ separates tangent
  vectors at $p$ if and only if
  $\m_{f(p)} / \m_{f(p)}^2 \to \m_p / \m_p^2$
  is surjective. Since
  $f \hookrightarrow \PP^n$ is injective
  and a closed map (as $X$ is complete),
  we get $X \hookrightarrow f(X) \subseteq \PP^n$ is a
  homeomorphism (we know it is a bijection,
  preimages of open sets are open, and
  images of closed sets are closed), and
  $f(X) \subseteq \PP^n$ is
  closed. Using this, $f$ is a
  closed embedding if and only if
  $\OO_{\PP^n} \to f_* \OO_X$ is
  surjective (HW) which happens if and only
  if $\OO_{\PP^n, f(p)} \to \OO_{X, p}$ is
  surjective for all $p \in X$. This
  happens if and only if
  \[
    \m_{f(p)} / \m_{f(p)}^2
    \longrightarrow \m_p / \m_p^2
  \]
  is surjective for all $p \in X$
  (see Mustata's notes, uses that
  $f_* \OO_X$ is coherent).
\end{proof}

\begin{definition}
  Let $\mathcal{L}$ be an invertible sheaf
  on a complete variety $X$.
  \begin{enumerate}
    \item $\mathcal{L}$ is
      \emph{very ample} if there exists
      a closed embedding
      $f : X \hookrightarrow \PP^n$
      and $\mathcal{L} \cong \OO_X(1) := f^* \OO_{\PP^n}(1)$.
    \item $\mathcal{L}$ is
      \emph{ample} if there exists $m > 0$ 
      such that $\mathcal{L}^{\otimes m}$
      is very ample.
  \end{enumerate}
\end{definition}

\begin{example}
  Let $X = \PP^n$ (for $n \ge 1$), then $\mathcal{L} = \OO_{\PP^n}(d)$ if and
  only if $d > 0$, if and only if
  $\OO_{\PP^n}(d)$ is ample (though this is
  not true in general).
\end{example}

\begin{exercise}
  If $\mathcal{F}$ is a coherent sheaf on
  a complete variety $X$ and $\OO_X(1)$ is
  then show that there exists an exact
  sequence:
  \[
    \begin{tikzcd}
    \bigoplus_{i = 1}^s \OO_X(n_i)
    \ar[r] & \bigoplus_{i = 1}^r \OO_X(m_i)
    \ar[r] & \mathcal{F} \ar[r] & 0
    \end{tikzcd}
  \]
\end{exercise}

\section{Divisors}

\begin{remark}
  For the rest of the lecture,
  let $X$ be an irreducible variety.
\end{remark}

\begin{definition}
  A \emph{prime divisor} on $X$ is a
  closed irreducible subvariety
  $D \subseteq X$ of codimension $1$.
\end{definition}

\begin{example}
  We have the following:
  \begin{enumerate}
    \item Prime divisors on $\PP^n$
      correspond to irreducible hypersurfaces.
    \item Prime divisors
      on a curve $C$ correspond to points.
  \end{enumerate}
\end{example}

\begin{definition}
  A \emph{(Weil) divisor} on $X$ is a
  formal sum
  \[
    D = \sum_{i = 1}^r a_i D_i,
  \]
  such $a_i \in \Z$ and the $D_i$ are
  distinct prime divisors. Write
  \[
    \Div X
    = \{\text{divisors on $X$}\}
    = \bigoplus_{\substack{E \subseteq X \\ \text{prime divisor}}} \Z[E].
  \]
\end{definition}

\begin{remark}
  We will see that when $X$ is smooth,
  there is a surjective group homomorphism
  \begin{align*}
    \Phi : \Div X &\longrightarrow \Pic X
    = \{\text{invertible sheaves on } X\} / {\cong} \\
    D &\longmapsto \OO_X(D).
  \end{align*}
  Furthermore, $\ker \Phi = \{\divv \varphi : \varphi \in K(X)^\times\}$,
  where $\divv \varphi$ is the divisor
  of zeros and poles of $\varphi$.
\end{remark}

\pagebreak

\begin{remark}[Local rings]
  For a closed irreducible subset
  $Z \subseteq X$, we can define a local
  ring
  \begin{align*}
    \OO_{X, Z}
    &= \{
      (\varphi, U) : \varphi \in \OO_X(U),\,
      U \subseteq X \text{ open},\,
      Z \cap U \ne \varnothing
    \} / {\sim} \\
    &= \varinjlim_{\substack{U \subseteq X \text{ open} \\ Z \cap U \ne \varnothing}} \OO_X(U),
  \end{align*}
  where $(\varphi, U) \sim (\psi, V)$
  if there exists $W \subseteq U \cap V$
  open with $Z \cap W \ne \varnothing$
  such that $\varphi|_W = \psi|_W$.
\end{remark}

  \chapter{Feb.~17 --- Divisors}

\section{Divisors, Continued}
\begin{prop}
   We have the following:
   \begin{enumerate}
     \item If $U \subseteq X$ is affine open,
       then
       we have an isomorphism
       \[
         \OO_X(U)_{\mathcal{I}_Z(U)}
         \overset{\cong}{\longrightarrow}
         \OO_{X, Z}
       \]
       when $Z \cap U \ne \varnothing$.
     \item $\dim \OO_{X, Z} = \codim_X Z$.
     \item $\OO_{X, Z}$ is a local ring.
     \item Let $\m_Z$ be the maximal
       ideal of $\OO_{X, Z}$.
       If $X$ is smooth, then
       $\OO_{X, Z}$ is regular, i.e.
       \[
         \dim \OO_{X, Z}
         = \dim \m_Z / \m_Z^2,
       \]
       where the right-hand side is the
       dimension as a vector space over
       $\OO_{X, Z} / \m_Z$.
   \end{enumerate}
\end{prop}

\begin{proof}
  (1) The proof is the same as when
  $Z = \{x\}$.

  (2) We can reduce to the case where
  $X$ is affine. Then we have a bijection
  \[
    \begin{tikzcd}
      \{\text{prime ideals $\p \le A$ such that $\p \cap (A \setminus I_Z) = \varnothing$}\} \ar[d]
    \ar[r, "\q \mapsto \q \OO_{X, Z}"]
    &
    \{\text{prime ideals of $\OO_{X, Z}$}\} \ar[l] \\
    \{\text{closed irreducible subvarieties $Z \subseteq Y \subseteq X$}\} \ar[u]
    \end{tikzcd}
  \]
  where $A = \OO_X(X)$.
  This proves the claim.

  (3) We use (1). Since
  $\mathcal{I}_Z(U) \subseteq \OO_X(U)$
  is prime, $\OO_{X, Z}$ is a local
  ring with maximal ideal
  \[
    \m_Z := \mathcal{I}_Z(U) \OO_{X, Z}
    = \{(\varphi, U) \in \OO_{X, Z} : \varphi|_{Z \cap U} = 0\} \subseteq \OO_{X, Z}.
  \]

  (4) In Algebraic Geometry I, we have
  seen that this holds when $Z$ is a point.
  Choose $x \in Z$, then
  \[
    \OO_{X, Z} \cong (\OO_{X, x})_{\{\varphi \in \OO_{X, x}\, :\, \varphi|_Z = 0\}}.
  \]
  As $\OO_{X, x}$ is regular, so is
  $\OO_{X, Z}$ by commutative algebra.
\end{proof}

\begin{remark}
  For the rest of this
  lecture, assume that $X$ is a smooth
  irreducible variety (so that the local
  rings are regular and we can talk about
  the function field of $X$).
\end{remark}

\begin{remark}
  Let $E \subseteq X$ be a prime divisor.
  Then $\OO_{X, E}$ is a regular local ring
  of dimension $1$. By results from
  commutative algebra,
  $\OO_{X, E}$ is a
  \emph{discrete valuation ring} (DVR).
  In particular, this implies that
  $\OO_{X, E}$ is an integral domain
  which is a UFD with a unique irreducible
  element up to multiplication by units.

  So there exist an irreducible
  element $\pi \in \OO_{X, E}$ such that
  for any $0 \ne f \in \Frac(\OO_{X, E}) = K(X)$,
  \[
    f = u \pi^d
  \]
  for some $u \in \OO_{X, E}^\times$ and
  $d \in \Z$ (with $d \ge 0$ if and only if
  $f \in \OO_{X, E}$).
  Such a $\pi$ is called a \emph{uniformizer}.

  We will write $\ord_E(f) = d$ when
  $f = u \pi^d$, which we view as the
  multiplicity of vanishing of $f$ along $E$.
\end{remark}

\begin{remark}
  Note the following:
  \begin{itemize}
    \item $\ord_E(fg) = \ord_E(f) + \ord_E(g)$.
    \item $\ord_E(f + g) \ge \min\{\ord_E(f), \ord_E(g)\}$.
  \end{itemize}
  Along with some other properties,
  the above implies that $\ord_E$ is a
  \emph{valuation} on $K(X)$.
\end{remark}

\begin{example}
  Consider $0 \in \Affine^1$. Then we have
  \[
    \OO_{\Affine^1, 0}
    \cong \left\{\frac{f}{g} : f, g \in k[x],\, g(0) \ne 0\right\}.
  \]
  For $\varphi \in \OO_{\Affine^1, 0}$,
  we can write
  \[
    \varphi = u x^m
  \]
  such that $u \in \OO_{\Affine^1, 0}^\times = \{f / g : f, g \in k[x],\, f(0), g(0)  \ne 0\}$.
  So $\ord_0 \varphi = m$.
\end{example}

\begin{remark}
  We can also think of
  $\OO_{X, E}
    = \{\varphi \in K(X) : \ord_E(\varphi) \ge 0 \}$.
\end{remark}

\begin{definition}
  For $0 \ne \varphi \in K(X)$, its
  \emph{divisor of zeros and poles} is
  \[
    \divv(\varphi) = \divv_X(\varphi)
    := \sum_{\substack{E \subseteq X \\ \text{prime}}} \ord_E(\varphi) E.
  \]
\end{definition}

\begin{prop}
  The divisor of zeros and poles
  $\divv(\varphi)$ is a divisor, i.e. we have
  $\ord_E(\varphi) = 0$ for all but finitely
  many prime divisors $E \subseteq X$.
\end{prop}

\begin{proof}
  For an open affine $\varnothing \ne U \subseteq X$.
  Since $X \setminus U$ contains at most
  finitely many prime divisors, it suffices
  to show that $\divv_U(\varphi)$
  is a divisor. Write
  $\varphi = f / g$ with
  $f, g \in \OO_X(U) = A$. Since
  \[
    \divv_U(\varphi) = \divv_U(f) - \divv_U(g),
  \]
  it suffices to consider the case when
  $\varphi \in \OO_X(U)$. Now for
  $E \subseteq X$ prime with
  $E \cap U \ne \varnothing$, we have
  \begin{align*}
    \ord_E(\varphi) > 0
    &\iff \varphi \in \m_E \subseteq \OO_{X, E} \\
    &\iff \varphi \text{ vanishes on } E \cap U \\
    &\iff E \subseteq V(\varphi).
  \end{align*}
  As $V(\varphi)$ contains finitely many
  prime divisors,
  we get that $\divv_U(\varphi)$
  is a divisor.
\end{proof}

\begin{example}
  Let $f \in k[x_1, \dots, x_n] \in \OO_{\Affine^n}(\Affine^n)$.
  We can write
  \[
    f = cf_1^{a_1} \cdots f_r^{a_r}
  \]
  with $f_i$ irreducible and  $a_i \ge 0$.
  Then we have
  \[
    \divv_{\Affine^n}(f) = \sum_{i = 1}^r a_i V(f_i).
  \]
  This follows since $\ord_{V(f_i)} f_i = 1$,
  which in turns follows from
  $\OO_{\Affine^n, V(f_i)} \cong k[x_1, \dots, x_n]_{(f_i)}$,
  which has maximal ideal
  $(f_i)$ and hence uniformizer $f_i$.
\end{example}

\begin{prop}[Algebraic Hartog's lemma]
  For $0 \ne \varphi \in K(X)$, we have
  $\divv \varphi \ge 0$ (i.e.
  $\ord_E \varphi \ge 0$ for all prime
  divisors $E \subseteq X$)
  if and only if $\varphi \in \OO_X(X)$.
\end{prop}

\begin{proof}
  It suffices to show the statement on
  an affine cover. So we may assume that
  $X$ is affine. Write $A = \OO_X(X)$.
  Hartog's lemma in commutative algebra
  implies that
  \[
    A = \bigcap_{\substack{\p \le A \text{ prime} \\ \text{height $1$}}} A_\p
    \subseteq \Frac(A).
  \]
  Thus for $\varphi \in \Frac(A) \cong K(X)$,
  we have
  $\varphi \in A$ if and only if
  $\varphi \in A_\p$ for all height $1$
  prime ideals $\p \le A$, which happens
  if and only if $\ord_E(\varphi) \ge 0$
  for all prime divisors $E \subseteq X$, i.e.
  $\divv_X(\varphi) \ge 0$.
\end{proof}

\section{Class Groups}
\begin{definition}
  A divisor $D \in \Div X$ is \emph{principal}
  if $D = \divv_X(\varphi)$ for some
  $0 \ne \varphi \in K(X)$.
\end{definition}

\begin{example}
  Any divisor on $\Affine^n$ is principal
  by the previous examples (this is
  also true when $\Affine^n$ is replaced
  by an affine variety $X$ such that
  $\OO_X(X)$ is a UFD).
\end{example}

\begin{remark}
  Note that $\PDiv X = \{\divv \varphi : 0 \ne \varphi \in K(X)\} \subseteq \Div X$
  is a subgroup.
\end{remark}

\begin{definition}
  We say that $D_1, D_2 \in \Div X$ are
  \emph{linearly equivalent} if
  $D_1 - D_2$ is principal. The
  \emph{class group} of $X$ is
  \[
    \Cl(X) = \Div(X) / \PDiv(X)
    = \text{group of divisors up to $\sim$}.
  \]
\end{definition}

\begin{example}
  We have the following:
  \begin{enumerate}
    \item $\Cl(\Affine^n) = 0$.
    \item $\Cl(\PP^n) = \Z[H]$, where
      $H \subseteq \PP^n$ is a hyperplane.

      To see this, recall that we have
      a bijection
      \[
        \{\text{prime divisors in $\PP^n$}\}
        \longleftrightarrow
        \{\text{irreducible hypersurfaces}\}.
      \]
      Consider $\deg : \Div(\PP^n) \to \Z$
      which sends
      $\sum a_i E_i = \sum a_i \deg E_i$,
      which is clearly surjective. For
      $\varphi \in K(\PP^n)$, write
      $\varphi = f / g$ where
      $f, g \in k[x_0, \dots, x_n]$
      are homogeneous of the same degree.
      So
      \[
        f = f_1^{a_1} \cdots f_r^{a_r}
        \quad\text{and}\quad
        g = g_1^{b_1} \cdots g_s^{b_s}
      \]
      with $f_i, g_i$ homogeneous and
      $a_i, b_i \in \Z_{\ge 0}$, and
      $\sum a_i \deg f_i = \sum b_i \deg g_i$.
      Now
      \[
        \divv \varphi
        = \sum_{i = 1}^r a_i V(f_i)
        - \sum_{i = 1}^s b_i V(g_i).
      \]
      So $\deg \varphi = 0$.
      This shows that
      $\PDiv(\PP^n) \subseteq \ker \deg$.
      One can show the reverse inclusion as well,
      so $\PDiv(\PP^n) = \ker \deg$.
      Thus we have
      $\Cl(\PP^n) = \Div(\PP^n) / \PDiv(\PP^n) \cong \Z$.
  \end{enumerate}
\end{example}

\section{Cartier Divisors}

\begin{definition}
  A divisor $D$ on $X$ is \emph{Cartier}
  if it is locally principal, i.e. at
  each $x \in X$, there exists an open
  neighborhood $x \in U_x \subseteq X$
  and $0 \ne \varphi_x \in K(X)$ such that
  $\divv_{U_x}(\varphi_x) = D|_{U_x}$.
\end{definition}

\begin{example}
  Principal divisors are Cartier
  (in fact, they are globally principal).
\end{example}

\begin{example}
  Let $H = V(x_0) \subseteq \PP^n$.
  Note $H \nsim 0$, so
  $H$ is not principal. But on
  $U_i = \{x_i \ne 0\} \subseteq \PP^n$,
  \[
    H|_{U_i} = \divv_{U_i}(x_0 / x_i),
  \]
  so $H$ is Cartier (alternatively,
  $\PP^n$ has an open cover by the $U_i$,
  and each $\Cl(U_i) = 0$).
\end{example}

\begin{prop}\label{prop:cartier}
  Every divisor $D$ on $X$ is Cartier.
\end{prop}

\begin{proof}
  If $D_1, D_2$ are Cartier, then so is
  $D_1 + D_2$. So it suffices to consider
  the case when $D = E$ for some prime
  divisor $E$ on $X$. Now fix $x \in X$.
  If $x \notin E$, then set
  $U_x = X \setminus E$ and $\varphi_x = 1$.
  Now assume $x \in E$. Since
  $\OO_{X, x}$ is a regular local ring,
  the Auslander-Buchsbaum theorem
  implies that $\OO_{X, x}$ is a UFD.
  Now we have a bijection
  \[
    \{\text{prime ideals of $\OO_{X, x}$}\}
    \longleftrightarrow
    \{\text{irreducible closed subvarieties } x \in Y \subseteq X\}.
  \]
  Write $J \subseteq \OO_{X, x}$
  for the prime ideal corresponding to $E$.
  As $J$ is a height $1$ prime ideal and
  $\OO_{X, x}$ is a UFD, there exists
  $f \in \OO_{X, x}$ such that $J = (f)$.
  So there exists an isomorphism
  \[
    \varphi : (\OO_{X, x})_{(f)}
    \overset{\cong}{\longrightarrow}
    \OO_{X, E}
  \]
  such that $\varphi(f) \OO_{X, E} = \m_E$.
  So $\ord_E(f) = 1$. One then checks that
  $\divv_{U_x}(f) = E|_{U_x}$ for some
  $U_x$.
\end{proof}

\begin{remark}
  Proposition \ref{prop:cartier}
  crucially uses our assumption
  that $X$ is smooth. It may fail in
  general.
\end{remark}

\end{document}
